
The general form of $j$th derivative of function $f$ is 
$$\frac{\partial^{(j)}{f}(z)}{\partial{z}} = \frac{(-1)^{j}j!}{z^{j+1}}$$ 
therefore
\begin{displaymath}
\begin{split}
  g(z) &= \sum_{j=1}^{m}{ \left. \frac{\partial^{(j-1)}{f}}{\partial{z}} \right|_{z=\lambda_{1}}\Phi_{1,j}(z)} \\
       &= \sum_{j=1}^{m}{ \left. \frac{(-1)^{j-1}(j-1)!}{z^{j}} \right|_{z=1}\Phi_{1,j}(z)} \\
       %&= \sum_{j=1}^{m}{ \left. \frac{1}{z^{j}} \right|_{z=1}(-1)^{j-1}(j-1)!\Phi_{1,j}(z)} \\
       &= \sum_{j=1}^{m}{\sum_{k=0}^{j-1}{(-1)^{j-1}(j-1)!\frac{(-1)^{j-1-k}}{(j-1-k)!}\frac{z^{k}}{k!}}} \\
\end{split}
\end{displaymath}
yielding equation
\begin{equation}
  g(z) = \sum_{j=1}^{m}{\sum_{k=0}^{j-1}{{{j-1}\choose{k}}(-z)^{k}}} 
\end{equation}

To swap summations we reason on the relation $k=j-1$, which holds by the inner summation upper limit:
$k\in \lbrace 0,\ldots,m-1 \rbrace$ because last value for $j$ is $m$. Finally, to cover the same set 
of pairs $(j, k)$, when $k=0$ then $j\in \lbrace 1,\ldots,m \rbrace$ and when $k=m-1$ then 
$j\in \lbrace m \rbrace$, therefore $j\in \lbrace k+1, \ldots, m \rbrace$ is required, in the general case.
Binomial manipulations yield:
\begin{displaymath}
  g(z) = \sum_{k=0}^{m-1}{\left(\sum_{j=k+1}^{m}{{{j-1}\choose{k}}}\right)(-z)^{k}}
       = \sum_{k=0}^{m-1}{\left(\sum_{j=k}^{m-1}{{{j}\choose{k}}}\right)(-z)^{k}}
\end{displaymath}
the inner sum admits a closed expression, yielding equation
\begin{equation}
  g(z) = \sum_{k=0}^{m-1}{{{m}\choose{k+1}}(-z)^{k}} \\
\end{equation}
namely, $\mathcal{R}_{m}^{-1}=g(\mathcal{R}_{m})$ for any proper, finite Riordan matrix $\mathcal{R}_{m}$.

\iffalse
Polynomial $g$ can also be written in closed form, assuming convergence condition $|z|\leq 1$:
\begin{displaymath}
g(z) = \frac{1- \left(1- z \right)^{m} }{z}
\end{displaymath}
\fi

For the sake of clarity, polynomial $g$ when $m=8$ and relaxing the condition $\lambda=1$, is defined according to 
\begin{displaymath}
\begin{split}
g{\left (z \right )} &= - \frac{z^{7}}{\lambda^{8}} \\
&+ z^{6} \left(\frac{1}{\lambda^{7}} + \frac{7}{\lambda^{8}}\right) \\
&+ z^{5} \left(- \frac{1}{\lambda^{6}} - \frac{6}{\lambda^{7}} - \frac{21}{\lambda^{8}}\right) \\
&+ z^{4} \left(\frac{1}{\lambda^{5}} + \frac{5}{\lambda^{6}} + \frac{15}{\lambda^{7}} + \frac{35}{\lambda^{8}}\right) \\
&+ z^{3} \left(- \frac{1}{\lambda^{4}} - \frac{4}{\lambda^{5}} - \frac{10}{\lambda^{6}} - \frac{20}{\lambda^{7}} - \frac{35}{\lambda^{8}}\right) \\
&+ z^{2} \left(\frac{1}{\lambda^{3}} + \frac{3}{\lambda^{4}} + \frac{6}{\lambda^{5}} + \frac{10}{\lambda^{6}} + \frac{15}{\lambda^{7}} + \frac{21}{\lambda^{8}}\right) \\
&+ z \left(- \frac{1}{\lambda^{2}} - \frac{2}{\lambda^{3}} - \frac{3}{\lambda^{4}} - \frac{4}{\lambda^{5}} - \frac{5}{\lambda^{6}} - \frac{6}{\lambda^{7}} - \frac{7}{\lambda^{8}}\right) \\
&+ \frac{1}{\lambda} + \frac{1}{\lambda^{2}} + \frac{1}{\lambda^{3}} + \frac{1}{\lambda^{4}} + \frac{1}{\lambda^{5}} + \frac{1}{\lambda^{6}} + \frac{1}{\lambda^{7}} + \frac{1}{\lambda^{8}}
\end{split}
\end{displaymath}
restoring $\lambda=1$ yields \[g{\left (z \right )} = - z^{7} + 8 z^{6} - 28 z^{5} + 56 z^{4} - 70 z^{3} + 56 z^{2} - 28 z + 8\]

A computation observation concerns the evaluation of $g(\mathcal{R}_{m})$,
which should be carried out as
\begin{displaymath}
g(z) = z \left(z \left(z \left(z \left(z \left(z \left(- z + 8\right) - 28\right) + 56\right) - 70\right) + 56\right) - 28\right) + 8
\end{displaymath}
namely, according to the Horner rule for polynomials, interpreting each
coefficient $c\in\mathbb{R}$ as $cI$ where $I\in\mathbb{C}^{m\times m}$ is the
identity matrix. Such approach requires $m-2$ matrix products and $m-1$
additions; finally, we implicitly use this scheme in all subsequent evaluation
of a polynomial $g$ to a matrix $A$.

Using Riordan array characterization we have 
\begin{displaymath}
D_{\frac{1}{z}}E_{\lambda_{1}} = \left[\begin{matrix}\frac{1}{\lambda_{1}} & 0 & 0 & 0 & 0 & 0 & 0 & 0\\\frac{1}{\lambda_{1}} & - \frac{1}{\lambda_{1}^{2}} & 0 & 0 & 0 & 0 & 0 & 0\\\frac{1}{\lambda_{1}} & - \frac{2}{\lambda_{1}^{2}} & \frac{2}{\lambda_{1}^{3}} & 0 & 0 & 0 & 0 & 0\\\frac{1}{\lambda_{1}} & - \frac{3}{\lambda_{1}^{2}} & \frac{6}{\lambda_{1}^{3}} & - \frac{6}{\lambda_{1}^{4}} & 0 & 0 & 0 & 0\\\frac{1}{\lambda_{1}} & - \frac{4}{\lambda_{1}^{2}} & \frac{12}{\lambda_{1}^{3}} & - \frac{24}{\lambda_{1}^{4}} & \frac{24}{\lambda_{1}^{5}} & 0 & 0 & 0\\\frac{1}{\lambda_{1}} & - \frac{5}{\lambda_{1}^{2}} & \frac{20}{\lambda_{1}^{3}} & - \frac{60}{\lambda_{1}^{4}} & \frac{120}{\lambda_{1}^{5}} & - \frac{120}{\lambda_{1}^{6}} & 0 & 0\\\frac{1}{\lambda_{1}} & - \frac{6}{\lambda_{1}^{2}} & \frac{30}{\lambda_{1}^{3}} & - \frac{120}{\lambda_{1}^{4}} & \frac{360}{\lambda_{1}^{5}} & - \frac{720}{\lambda_{1}^{6}} & \frac{720}{\lambda_{1}^{7}} & 0\\\frac{1}{\lambda_{1}} & - \frac{7}{\lambda_{1}^{2}} & \frac{42}{\lambda_{1}^{3}} & - \frac{210}{\lambda_{1}^{4}} & \frac{840}{\lambda_{1}^{5}} & - \frac{2520}{\lambda_{1}^{6}} & \frac{5040}{\lambda_{1}^{7}} & - \frac{5040}{\lambda_{1}^{8}}\end{matrix}\right]
\end{displaymath}
generated by the production matrix
\begin{displaymath}
\left[\begin{matrix}1 & - \frac{1}{\lambda_{1}} & 0 & 0 & 0 & 0 & 0\\0 & 1 & - \frac{2}{\lambda_{1}} & 0 & 0 & 0 & 0\\0 & 0 & 1 & - \frac{3}{\lambda_{1}} & 0 & 0 & 0\\0 & 0 & 0 & 1 & - \frac{4}{\lambda_{1}} & 0 & 0\\0 & 0 & 0 & 0 & 1 & - \frac{5}{\lambda_{1}} & 0\\0 & 0 & 0 & 0 & 0 & 1 & - \frac{6}{\lambda_{1}}\\0 & 0 & 0 & 0 & 0 & 0 & 1\end{matrix}\right]
\end{displaymath}
so the matrix satisfies the recurrence relation 
\begin{displaymath}
\begin{split}
d_{0,0}&=\frac{1}{\lambda_{1}}\\
d_{n,0}&=d_{n-1, 0}, \quad n>0 \\
d_{n,k}&=-\frac{k}{\lambda_{1}}d_{n-1, k-1} + d_{n-1,k}, \quad n,k > 0\\
\end{split}
\end{displaymath}
finally,
\begin{displaymath}
D_{\frac{1}{z}}E_{\lambda_{1}}\boldsymbol{z} = \left[\begin{matrix}\frac{1}{\lambda_{1}}\\- \frac{1}{\lambda_{1}^{2}} \left(z - \lambda_{1}\right)\\\frac{1}{\lambda_{1}^{3}} \left(z - \lambda_{1}\right)^{2}\\- \frac{1}{\lambda_{1}^{4}} \left(z - \lambda_{1}\right)^{3}\\\frac{1}{\lambda_{1}^{5}} \left(z - \lambda_{1}\right)^{4}\\- \frac{1}{\lambda_{1}^{6}} \left(z - \lambda_{1}\right)^{5}\\\frac{1}{\lambda_{1}^{7}} \left(z - \lambda_{1}\right)^{6}\\- \frac{1}{\lambda_{1}^{8}} \left(z - \lambda_{1}\right)^{7}\end{matrix}\right]
\end{displaymath}
therefore restoring $\lambda_{1}=1$ yields the polynomial
\[g{\left (z \right )} = \boldsymbol{1}^{T}D_{\frac{1}{z}}E_{\lambda_{1}}\boldsymbol{z} = - \left(z - 1\right)^{7} + \left(z - 1\right)^{6} - \left(z - 1\right)^{5} + \left(z - 1\right)^{4} - \left(z - 1\right)^{3} + \left(z - 1\right)^{2} - (z-1) + 1\]
hence we generalize for $m\in\mathbb{N}$:
\begin{displaymath}
\mathcal{R}_{m}^{-1} = g{\left (\mathcal{R}_{m} \right )} = \sum_{j=1}^{m}{\left(-Z_{1,2}^{[\mathcal{R}_{m}]}\right)^{j-1}}
\end{displaymath}

