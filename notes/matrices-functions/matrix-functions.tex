
Let $A\in\mathbb{C}^{m\times m}$ be a matrix and denote with $\sigma(A)$ the
spectrum of $A$, namely the set of $A$'s eigenvalues
\begin{displaymath}
\sigma(A) = \left\lbrace \lambda_{i}:
A\boldsymbol{v}_{i}=\lambda_{i}\boldsymbol{v}_{i} \, \text{for} \, 1\leq i\leq \nu\wedge
\boldsymbol{v}_{i}\in\mathbb{C}^{m}\right\rbrace 
\end{displaymath}
with multiplicities $\lbrace m_{i}: 1\leq i\leq \nu\rbrace$ respectively, such
that $ \sum_{i=1}^{\nu}{m_{i}}=m$.

Let $p(\lambda)=\det{\left(A-\lambda I\right)}=\prod_{i=1}^{\nu}{(\lambda -
\lambda_{i})^{m_{i}}}$ be the \textit{characteristic polynomial} of
matrix $A$ of degree $m$. For any polynomial $h$ of degree greater than $m$ it
is possible to perform the following division:
\begin{displaymath}
h(\lambda) = q(\lambda)p(\lambda)+r(\lambda), \quad \deg{r(\lambda) < m}
\end{displaymath}
for some polynomial $q$, so $h(A) = r(A)$ because $p(A)=0$; eventually, both
    polynomials (of \textit{different degrees}) $h$ and $r$ yield the same
    matrix when applied to $A$.
Since $\left. \frac{\partial^{(j)}{p}}{\partial{\lambda}}
\right|_{\lambda=\lambda_{i}}=0$ then
\begin{displaymath}
\left.\frac{\partial^{(j)}\left(h(\lambda) - r(\lambda)\right)}{\partial\lambda}\right|_{\lambda=\lambda_{i}} =
\left.\frac{\partial^{(j)}\left(q(\lambda)p(\lambda)\right)}{\partial\lambda}\right|_{\lambda=\lambda_{i}} = 0
\end{displaymath}
therefore, polynomials $h$ and $r$ satisfy 
\begin{displaymath}
h(A)=r(A) \leftrightarrow
\left.\frac{\partial^{(j)}h}{\partial\lambda}=\frac{\partial^{(j)}r}{\partial\lambda}\right|_{\lambda=\lambda_{i}}
\end{displaymath}
for $i\in \lbrace 1, \ldots, \nu \rbrace$ for $j \in \lbrace 0, \ldots, m_{i}-1 \rbrace$;
in words, \textit{polynomials} $h$ \textit{and} $r$ \textit{take the same values on} $\sigma(A)$.

Let $f:\mathbb{C}\rightarrow \mathbb{C}$ be a function on the formal variable
$z$. We say that $f$ \textit{is defined on $\sigma(A)$} if exists
\begin{displaymath}
    \left. \frac{\partial^{(j)}{f}}{\partial{z}} \right|_{z=\lambda_{i}},\quad
    i\in \lbrace 1, \ldots, \nu \rbrace \wedge j \in \lbrace 0, \ldots, m_{i}-1
    \rbrace
\end{displaymath}

Given a function $f$ defined on $\sigma(A)$, we aim to define a polynomial $g$
such that $f$ and $g$ take the same values on $\sigma(A)$, to compute $f(A)$
via $g(A)$; moreover, $g$ is an \emph{Hermite interpolating polynomial} which
can be written using the base of \textit{generalized Lagrange polynomials}
$\left\lbrace \Phi_{i,j}\in\prod_{m-1} \right\rbrace$. Formally:
\begin{displaymath}
g(z) = \sum_{i=1}^{\nu}{\sum_{j=1}^{m_{i}}{ \left.
\frac{\partial^{(j-1)}{f}}{\partial{z}} \right|_{z=\lambda_{i}}\Phi_{i,j}(z) }}
\end{displaymath}
where each polynomial $\Phi_{i,j}$ is implicitly defined as the solution of the
system with $m$ constraints:
\begin{displaymath}
    \left. \frac{\partial^{(r-1)}{\Phi_{i,j}}}{\partial{z}} \right|_{z=\lambda_{l}} = \delta_{i,l}\delta_{j,r}
\end{displaymath}
for $l\in \lbrace 1, \ldots, \nu \rbrace$ for $r \in \lbrace 1, \ldots, m_{l}
\rbrace$, where $\delta$ is the Kroneker delta, defined as $\delta_{i,j}=1
\leftrightarrow i=j$.  Observe that if $m_{l}=1$  then $\left\lbrace \Phi_{i,j}\in\prod_{m-1} \right\rbrace$
reduces to the usual Lagrange base, for all $l\in\lbrace 1, \ldots, \nu\rbrace$;
for the sake of clarity, if $\nu=4$ then the polynomials $\left\lbrace \Phi_{i,1}\in\prod_{3}:i\in\lbrace1,\ldots,4\rbrace \right\rbrace$
defined as follows
\begin{displaymath}
\begin{split}
\Phi_{ 1, 1 }{\left (z \right )} &= \frac{\left(z - \lambda_{2}\right)
\left(z - \lambda_{3}\right) \left(z - \lambda_{4}\right)}{\left(\lambda_{1} -
\lambda_{2}\right) \left(\lambda_{1} - \lambda_{3}\right) \left(\lambda_{1} -
\lambda_{4}\right)} \\ 
\Phi_{ 2, 1 }{\left (z \right )} &= - \frac{\left(z -
\lambda_{1}\right) \left(z - \lambda_{3}\right) \left(z -
\lambda_{4}\right)}{\left(\lambda_{1} - \lambda_{2}\right) \left(\lambda_{2} -
\lambda_{3}\right) \left(\lambda_{2} - \lambda_{4}\right)} \\ 
\Phi_{ 3, 1 }{\left (z \right )} &= \frac{\left(z - \lambda_{1}\right) \left(z -
\lambda_{2}\right) \left(z - \lambda_{4}\right)}{\left(\lambda_{1} -
\lambda_{3}\right) \left(\lambda_{2} - \lambda_{3}\right) \left(\lambda_{3} -
\lambda_{4}\right)} \\ 
\Phi_{ 4, 1 }{\left (z \right )} &= - \frac{\left(z -
\lambda_{1}\right) \left(z - \lambda_{2}\right) \left(z -
\lambda_{3}\right)}{\left(\lambda_{1} - \lambda_{4}\right) \left(\lambda_{2} -
\lambda_{4}\right) \left(\lambda_{3} - \lambda_{4}\right)}\\
\end{split}
\end{displaymath}
are a Lagrange base respect to eigenvalues $\lambda_{1}, \lambda_{2},\lambda_{3},\lambda_{4}$.
On the other hand, if $\nu=1$ then there is one eigenvalue $\lambda_{1}$ only, with algebraic multiplicity $m_{1}=m$; for the sake of clarity again,
for $m=8$ the polynomials $\left\lbrace \Phi_{1,j}\in\prod_{7}:j\in\lbrace1,\ldots,8\rbrace \right\rbrace$ defined as follows
\begin{equation}
\begin{array}{c}
\Phi_{ 1, 1 }{\left (z \right )} = 1 \\ 
\Phi_{ 1, 2 }{\left (z \right )} = z - \lambda_{1} \\ 
\Phi_{ 1, 3 }{\left (z \right )} = \frac{z^{2}}{2} - z \lambda_{1} + \frac{\lambda_{1}^{2}}{2}\\ 
\Phi_{ 1, 4 }{\left (z \right )} = \frac{z^{3}}{6} - \frac{z^{2} \lambda_{1}}{2} + \frac{z \lambda_{1}^{2}}{2} - \frac{\lambda_{1}^{3}}{6} \\ 
\Phi_{ 1, 5 }{\left (z \right )} = \frac{z^{4}}{24} - \frac{z^{3} \lambda_{1}}{6} + \frac{z^{2} \lambda_{1}^{2}}{4} - \frac{z \lambda_{1}^{3}}{6} + \frac{\lambda_{1}^{4}}{24} \\ 
\Phi_{ 1, 6 }{\left (z \right )} = \frac{z^{5}}{120} - \frac{z^{4} \lambda_{1}}{24} + \frac{z^{3} \lambda_{1}^{2}}{12} - \frac{z^{2} \lambda_{1}^{3}}{12} + \frac{z \lambda_{1}^{4}}{24} - \frac{\lambda_{1}^{5}}{120} \\
\Phi_{ 1, 7 }{\left (z \right )} = \frac{z^{6}}{720} - \frac{z^{5} \lambda_{1}}{120} + \frac{z^{4} \lambda_{1}^{2}}{48} - \frac{z^{3} \lambda_{1}^{3}}{36} + \frac{z^{2} \lambda_{1}^{4}}{48} - \frac{z \lambda_{1}^{5}}{120} + \frac{\lambda_{1}^{6}}{720} \\ 
\Phi_{ 1, 8 }{\left (z \right )} = \frac{z^{7}}{5040} - \frac{z^{6} \lambda_{1}}{720} + \frac{z^{5} \lambda_{1}^{2}}{240} - \frac{z^{4} \lambda_{1}^{3}}{144} + \frac{z^{3} \lambda_{1}^{4}}{144} - \frac{z^{2} \lambda_{1}^{5}}{240} + \frac{z \lambda_{1}^{6}}{720} - \frac{\lambda_{1}^{7}}{5040}\\
\end{array}
\label{eq:generalized-Lagrange-base}
\end{equation}
are a \textit{generalized} Lagrange base respect to eigenvalue $\lambda_{1}$.
Evaluating polynomial $g$ on matrix $A$ yield:
\begin{displaymath}
g(A) = \sum_{i=1}^{\nu}{\sum_{j=1}^{m_{i}}{ \left.  \frac{\partial^{(j-1)}{f}}{\partial{z}} \right|_{z=\lambda_{i}}\Phi_{i,j}(A) }}
     = \sum_{i=1}^{\nu}{\sum_{j=1}^{m_{i}}{ \left.  \frac{\partial^{(j-1)}{f}}{\partial{z}} \right|_{z=\lambda_{i}}Z_{ij}^{[A]} }}
\end{displaymath}
where matrix $Z_{ij}^{[A]}=\Phi_{i,j}(A)$, for $i\in \lbrace 1, \ldots, \nu \rbrace$
and $j \in \lbrace 0, \ldots, m_{i}-1 \rbrace$, is a \textit{component matrix}
of $A$. Moreover, we can rewrite it according to facts reported in the appendix:
\begin{displaymath}
g(A) = \sum_{i=1}^{\nu}{\sum_{j=1}^{m_{i}}{ \left.  \frac{\partial^{(j-1)}{f}}{\partial{z}} \right|_{z=\lambda_{i}}\frac{1}{(j-1)!}{Z_{i1}^{[A]}(A-\lambda_{i}I)^{j-1}} }}
\end{displaymath}
