
Let $m=8$ and define
\begin{displaymath}
%\mathcal{P}_{m}=\left[\begin{matrix}1 & 0 & 0 & 0 & 0 & 0 & 0 & 0\\1 & 1 & 0 & 0 & 0 & 0 & 0 & 0\\1 & 2 & 1 & 0 & 0 & 0 & 0 & 0\\1 & 3 & 3 & 1 & 0 & 0 & 0 & 0\\1 & 4 & 6 & 4 & 1 & 0 & 0 & 0\\1 & 5 & 10 & 10 & 5 & 1 & 0 & 0\\1 & 6 & 15 & 20 & 15 & 6 & 1 & 0\\1 & 7 & 21 & 35 & 35 & 21 & 7 & 1\end{matrix}\right]
\mathcal{P}_{m}=\left[\begin{matrix}1 &   &   &   &   &   &   &  \\1 & 1 &   &   &   &   &   &  \\1 & 2 & 1 &   &   &   &   &  \\1 & 3 & 3 & 1 &   &   &   &  \\1 & 4 & 6 & 4 & 1 &   &   &  \\1 & 5 & 10 & 10 & 5 & 1 &   &  \\1 & 6 & 15 & 20 & 15 & 6 & 1 &  \\1 & 7 & 21 & 35 & 35 & 21 & 7 & 1\end{matrix}\right]
\end{displaymath}
therefore $\mathcal{P}_{m}^{-1}$ is
\begin{displaymath}
%\left[\begin{matrix}1 & 0 & 0 & 0 & 0 & 0 & 0 & 0\\-1 & 1 & 0 & 0 & 0 & 0 & 0 & 0\\1 & -2 & 1 & 0 & 0 & 0 & 0 & 0\\-1 & 3 & -3 & 1 & 0 & 0 & 0 & 0\\1 & -4 & 6 & -4 & 1 & 0 & 0 & 0\\-1 & 5 & -10 & 10 & -5 & 1 & 0 & 0\\1 & -6 & 15 & -20 & 15 & -6 & 1 & 0\\-1 & 7 & -21 & 35 & -35 & 21 & -7 & 1\end{matrix}\right]
\left[\begin{matrix}1 &   &   &   &   &   &   &  \\-1 & 1 &   &   &   &   &   &  \\1 & -2 & 1 &   &   &   &   &  \\-1 & 3 & -3 & 1 &   &   &   &  \\1 & -4 & 6 & -4 & 1 &   &   &  \\-1 & 5 & -10 & 10 & -5 & 1 &   &  \\1 & -6 & 15 & -20 & 15 & -6 & 1 &  \\-1 & 7 & -21 & 35 & -35 & 21 & -7 & 1\end{matrix}\right]
\end{displaymath}
$\mathcal{P}_{m}^{r}$ is
\begin{displaymath}
%\left[\begin{matrix}1 & 0 & 0 & 0 & 0 & 0 & 0 & 0\\r & 1 & 0 & 0 & 0 & 0 & 0 & 0\\r^{2} & 2 r & 1 & 0 & 0 & 0 & 0 & 0\\r^{3} & 3 r^{2} & 3 r & 1 & 0 & 0 & 0 & 0\\r^{4} & 4 r^{3} & 6 r^{2} & 4 r & 1 & 0 & 0 & 0\\r^{5} & 5 r^{4} & 10 r^{3} & 10 r^{2} & 5 r & 1 & 0 & 0\\r^{6} & 6 r^{5} & 15 r^{4} & 20 r^{3} & 15 r^{2} & 6 r & 1 & 0\\r^{7} & 7 r^{6} & 21 r^{5} & 35 r^{4} & 35 r^{3} & 21 r^{2} & 7 r & 1\end{matrix}\right]
\left[\begin{matrix}1 &   &   &   &   &   &   &  \\r & 1 &   &   &   &   &   &  \\r^{2} & 2 r & 1 &   &   &   &   &  \\r^{3} & 3 r^{2} & 3 r & 1 &   &   &   &  \\r^{4} & 4 r^{3} & 6 r^{2} & 4 r & 1 &   &   &  \\r^{5} & 5 r^{4} & 10 r^{3} & 10 r^{2} & 5 r & 1 &   &  \\r^{6} & 6 r^{5} & 15 r^{4} & 20 r^{3} & 15 r^{2} & 6 r & 1 &  \\r^{7} & 7 r^{6} & 21 r^{5} & 35 r^{4} & 35 r^{3} & 21 r^{2} & 7 r & 1\end{matrix}\right]
\end{displaymath}
$\sqrt{\mathcal{P}_{m}}$ is
\begin{displaymath}
%\left[\begin{matrix}1 & 0 & 0 & 0 & 0 & 0 & 0 & 0\\\frac{1}{2} & 1 & 0 & 0 & 0 & 0 & 0 & 0\\\frac{1}{4} & 1 & 1 & 0 & 0 & 0 & 0 & 0\\\frac{1}{8} & \frac{3}{4} & \frac{3}{2} & 1 & 0 & 0 & 0 & 0\\\frac{1}{16} & \frac{1}{2} & \frac{3}{2} & 2 & 1 & 0 & 0 & 0\\\frac{1}{32} & \frac{5}{16} & \frac{5}{4} & \frac{5}{2} & \frac{5}{2} & 1 & 0 & 0\\\frac{1}{64} & \frac{3}{16} & \frac{15}{16} & \frac{5}{2} & \frac{15}{4} & 3 & 1 & 0\\\frac{1}{128} & \frac{7}{64} & \frac{21}{32} & \frac{35}{16} & \frac{35}{8} & \frac{21}{4} & \frac{7}{2} & 1\end{matrix}\right]
\left[\begin{matrix}1 &   &   &   &   &   &   &  \\\frac{1}{2} & 1 &   &   &   &   &   &  \\\frac{1}{4} & 1 & 1 &   &   &   &   &  \\\frac{1}{8} & \frac{3}{4} & \frac{3}{2} & 1 &   &   &   &  \\\frac{1}{16} & \frac{1}{2} & \frac{3}{2} & 2 & 1 &   &   &  \\\frac{1}{32} & \frac{5}{16} & \frac{5}{4} & \frac{5}{2} & \frac{5}{2} & 1 &   &  \\\frac{1}{64} & \frac{3}{16} & \frac{15}{16} & \frac{5}{2} & \frac{15}{4} & 3 & 1 &  \\\frac{1}{128} & \frac{7}{64} & \frac{21}{32} & \frac{35}{16} & \frac{35}{8} & \frac{21}{4} & \frac{7}{2} & 1\end{matrix}\right]
\end{displaymath}
finally, $e^{\mathcal{P}_{m}}$ is
\begin{displaymath}
%e \left[\begin{matrix}1 & 0 & 0 & 0 & 0 & 0 & 0 & 0\\1 & 1 & 0 & 0 & 0 & 0 & 0 & 0\\2 & 2 & 1 & 0 & 0 & 0 & 0 & 0\\5 & 6 & 3 & 1 & 0 & 0 & 0 & 0\\15 & 20 & 12 & 4 & 1 & 0 & 0 & 0\\52 & 75 & 50 & 20 & 5 & 1 & 0 & 0\\203 & 312 & 225 & 100 & 30 & 6 & 1 & 0\\877 & 1421 & 1092 & 525 & 175 & 42 & 7 & 1\end{matrix}\right]
e \left[\begin{matrix}1 &   &   &   &   &   &   &  \\1 & 1 &   &   &   &   &   &  \\2 & 2 & 1 &   &   &   &   &  \\5 & 6 & 3 & 1 &   &   &   &  \\15 & 20 & 12 & 4 & 1 &   &   &  \\52 & 75 & 50 & 20 & 5 & 1 &   &  \\203 & 312 & 225 & 100 & 30 & 6 & 1 &  \\877 & 1421 & 1092 & 525 & 175 & 42 & 7 & 1\end{matrix}\right]
\end{displaymath}
such matrix is known as $A056857$ in the OEIS, there we found
the interesting relation $e^{\mathcal{P}_{m}}=e\cdot\left(\mathcal{S}_{m}\cdot \mathcal{P}_{m}\cdot \mathcal{S}_{m}^{-1}\right)$,
where $\mathcal{S}_{m}$ is the matrix of Stirling numbers of the second kind, defined in a later case study.
Also, it is interesting the application of function $f(z)=e^{\alpha z}$ to the matrix $H_{m}$ defined as
\begin{displaymath}
%H = \left[\begin{matrix}0 & 0 & 0 & 0 & 0 & 0 & 0 & 0\\1 & 0 & 0 & 0 & 0 & 0 & 0 & 0\\0 & 2 & 0 & 0 & 0 & 0 & 0 & 0\\0 & 0 & 3 & 0 & 0 & 0 & 0 & 0\\0 & 0 & 0 & 4 & 0 & 0 & 0 & 0\\0 & 0 & 0 & 0 & 5 & 0 & 0 & 0\\0 & 0 & 0 & 0 & 0 & 6 & 0 & 0\\0 & 0 & 0 & 0 & 0 & 0 & 7 & 0\end{matrix}\right]
H_{m} = \left[\begin{matrix} 0 &   &   &   &   &   &   &  \\1 & 0   &   &   &   &   &   &  \\  & 2 &  0  &   &   &   &   &  \\  &   & 3 &  0  &   &   &   &  \\  &   &   & 4 &  0  &   &   &  \\  &   &   &   & 5 &  0  &   &  \\  &   &   &   &   & 6 &  0  &  \\  &   &   &   &   &   & 7 &  0 \end{matrix}\right]
\end{displaymath}
It requires the generalized Lagrange base
\begin{displaymath}
\left\{\Phi_{ 1, 1 }{\left (z \right )} = 1, \Phi_{ 1, 2 }{\left (z \right )} = z, \Phi_{ 1, 3 }{\left (z \right )} = \frac{z^{2}}{2},\right.
\Phi_{ 1, 4 }{\left (z \right )} = \frac{z^{3}}{6}, \Phi_{ 1, 5 }{\left (z \right )} = \frac{z^{4}}{24}, \Phi_{ 1, 6 }{\left (z \right )} = \frac{z^{5}}{120},
\left.\Phi_{ 1, 7 }{\left (z \right )} = \frac{z^{6}}{720}, \Phi_{ 1, 8 }{\left (z \right )} = \frac{z^{7}}{5040}\right\}
\end{displaymath}
to define the interpolating polynomial $g$ as
\begin{displaymath}
g{\left (z \right )} = \frac{\alpha^{7} z^{7}}{5040} + \frac{\alpha^{6} z^{6}}{720} + \frac{\alpha^{5} z^{5}}{120} + \frac{\alpha^{4} z^{4}}{24} + \frac{\alpha^{3} z^{3}}{6} + \frac{\alpha^{2} z^{2}}{2} + \alpha z + 1
\end{displaymath}
therefore $g(H_{m})=e^{\alpha H_{m}}$ is the matrix
\begin{displaymath}
%\left[\begin{matrix}1 & 0 & 0 & 0 & 0 & 0 & 0 & 0\\\alpha & 1 & 0 & 0 & 0 & 0 & 0 & 0\\\alpha^{2} & 2 \alpha & 1 & 0 & 0 & 0 & 0 & 0\\\alpha^{3} & 3 \alpha^{2} & 3 \alpha & 1 & 0 & 0 & 0 & 0\\\alpha^{4} & 4 \alpha^{3} & 6 \alpha^{2} & 4 \alpha & 1 & 0 & 0 & 0\\\alpha^{5} & 5 \alpha^{4} & 10 \alpha^{3} & 10 \alpha^{2} & 5 \alpha & 1 & 0 & 0\\\alpha^{6} & 6 \alpha^{5} & 15 \alpha^{4} & 20 \alpha^{3} & 15 \alpha^{2} & 6 \alpha & 1 & 0\\\alpha^{7} & 7 \alpha^{6} & 21 \alpha^{5} & 35 \alpha^{4} & 35 \alpha^{3} & 21 \alpha^{2} & 7 \alpha & 1\end{matrix}\right]
\left[\begin{matrix}1 &   &   &   &   &   &   &  \\\alpha & 1 &   &   &   &   &   &  \\\alpha^{2} & 2 \alpha & 1 &   &   &   &   &  \\\alpha^{3} & 3 \alpha^{2} & 3 \alpha & 1 &   &   &   &  \\\alpha^{4} & 4 \alpha^{3} & 6 \alpha^{2} & 4 \alpha & 1 &   &   &  \\\alpha^{5} & 5 \alpha^{4} & 1  \alpha^{3} & 1  \alpha^{2} & 5 \alpha & 1 &   &  \\\alpha^{6} & 6 \alpha^{5} & 15 \alpha^{4} & 2  \alpha^{3} & 15 \alpha^{2} & 6 \alpha & 1 &  \\\alpha^{7} & 7 \alpha^{6} & 21 \alpha^{5} & 35 \alpha^{4} & 35 \alpha^{3} & 21 \alpha^{2} & 7 \alpha & 1\end{matrix}\right]
\end{displaymath}
yielding the identity $e^{\alpha H_{m}} = \mathcal{P}_{m}^{\alpha}$;
moreover, $e^{(\alpha+\beta) H_{m}} = \mathcal{P}_{m}^{\alpha+\beta} = \mathcal{P}_{m}^{\alpha}\mathcal{P}_{m}^{\beta} $ 
also holds, as well as $\log{\mathcal{P}_{m}}=H_{m}$.

We define a new matrix $\hat{H}_{m}$ having the same shape of $H_{m}$, making
coefficients symbolic values:
\begin{displaymath}
%\left[\begin{matrix}0 & 0 & 0 & 0 & 0 & 0 & 0 & 0\\h_{1} & 0 & 0 & 0 & 0 & 0 & 0 & 0\\0 & h_{2} & 0 & 0 & 0 & 0 & 0 & 0\\0 & 0 & h_{3} & 0 & 0 & 0 & 0 & 0\\0 & 0 & 0 & h_{4} & 0 & 0 & 0 & 0\\0 & 0 & 0 & 0 & h_{5} & 0 & 0 & 0\\0 & 0 & 0 & 0 & 0 & h_{6} & 0 & 0\\0 & 0 & 0 & 0 & 0 & 0 & h_{7} & 0\end{matrix}\right]
\hat{H}_{m} = \left[\begin{matrix}  &   &   &   &   &   &   &  \\h_{1} &   &   &   &   &   &   &  \\  & h_{2} &   &   &   &   &   &  \\  &   & h_{3} &   &   &   &   &  \\  &   &   & h_{4} &   &   &   &  \\  &   &   &   & h_{5} &   &   &  \\  &   &   &   &   & h_{6} &   &  \\  &   &   &   &   &   & h_{7} &  \end{matrix}\right]
\end{displaymath}
such that $e^{\alpha\hat{H}_{m}}=\mathcal{P}^{\alpha} \leftrightarrow h_{i}=\frac{i}{\alpha}$, 
for $i\in \lbrace 1,\ldots,7 \rbrace$, generalizing previous argument; 
for completeness, we report the whole matrix $e^{\alpha\hat{H}_{m}}$ in the appendix.
