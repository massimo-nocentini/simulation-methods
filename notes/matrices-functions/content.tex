
\section*{Basic definitions}

Let $A\in\mathbb{C}^{m\times m}$ be a matrix and denote with $\sigma$ the spectre of $A$, namely the set of $A$ eigenvalues, 
formally $\sigma(A) = \lbrace \lambda_{i}: 1\leq i\leq \nu\rbrace$,
with multiplicities $\lbrace m_{i}: 1\leq i\leq \nu\rbrace$, respectively, such that
\begin{displaymath}
\sum_{i=1}^{\nu}{m_{i}}=m
\end{displaymath}

Let $f$ be a function on the variable $z$. We say that function $f$ 
\emph{is defined on the spectre $\sigma$ of matrix $A$} if exists
    \begin{displaymath}
        \left. \frac{\partial^{(j)}{f}}{\partial{z}} \right|_{z=\lambda_{i}}
    \end{displaymath}
for $i\in \lbrace 1, \ldots, \nu \rbrace$, for $j \in \lbrace 0, \ldots, m_{i}-1 \rbrace$.    

Given a function $f$ defined on the spectre of a matrix $A$, define the polynomial $g$ such that
it takes the same values of $f$ on the spectre of $A$, formally:
    \begin{displaymath}
        \left. \frac{\partial^{(j)}{f}}{\partial{z}} \right|_{z=\lambda_{i}} =
        \left. \frac{\partial^{(j)}{g}}{\partial{z}} \right|_{z=\lambda_{i}}
    \end{displaymath}
for $i\in \lbrace 1, \ldots, \nu \rbrace$, for $j \in \lbrace 0, \ldots, m_{i}-1 \rbrace$,
then $f(A) = g(A)$; finally, polynomial $g$ is an \emph{interpolating Hermite polynomial} which
can be defined using the generalized Lagrange base $ \lbrace \Phi_{ij} \rbrace$, formally:
\begin{displaymath}
g(z) = \sum_{i=1}^{\nu}{\sum_{j=1}^{m_{i}}{
        \left. \frac{\partial^{(j-1)}{f}}{\partial{z}} \right|_{z=\lambda_{i}}\Phi_{ij}(z)
    }}
\end{displaymath}
where each polynomial $\Phi_{ij}$, of degree $m-1$, is defined implicitly as the solution of the system:
\begin{displaymath}
    \left. \frac{\partial^{(r-1)}{\Phi_{ij}}}{\partial{z}} \right|_{z=\lambda_{l}} = \delta_{il}\delta_{jr}
\end{displaymath}
for $l\in \lbrace 1, \ldots, \nu \rbrace$, for $r \in \lbrace 1, \ldots, m_{l} \rbrace$, where
$\delta$ is the Kroneker delta, defined as $\delta_{ij}=1 \leftrightarrow i=j$.

The aim of this work is to study functions of \emph{Riordan matrices}. A Riordan matrix 
$\mathcal{R}=(d_{nk})_{n,k\in\mathbb{N}}$ is an infinite, lower triangular matrix
such that there exist two generating functions $d$ and $h$, with $h(0)=0 \wedge h^{\prime}(0)\neq0$, satisfying
$R_{k}(t) = d(t)h(t)^{k}$
where $R_{k}(t) = \sum_{n\in\mathbb{N}}{d_{nk}t^{n}}$ is the formal power series with coefficients lying 
on column $k$; finally, we write $\mathcal{R}$ as a pair, namely $\mathcal{R}=(d, h)$. 

In the rest of this document we assume that $\mathcal{R}_{m}\in\mathbb{C}^{m\times m}$ is a \emph{finite Riordan matrix}, 
hence $\sigma(\mathcal{R}_{m})= \lbrace \lambda_{1} \rbrace$, so $\nu=1$, where $\lambda_{1}=1$ 
with multiplicity $m_{1}=m$. 


Therefore, the generalized Lagrange base is composed of polynomials
\begin{displaymath}
  \Phi_{1j}(z) = \sum_{k=0}^{j-1}{\frac{(-1)^{j-1-k}}{(j-1-k)!}\frac{z^{k}}{k!}}
\end{displaymath}

For the sake of clarity we show polynomials $\Phi_{1j}$ for $j\in \lbrace 1,\ldots,m \rbrace$ relative to any finite Riordan
matrix $\mathcal{R}_{m}$ where $m=8$:
\begin{displaymath}
\begin{array}{c}
 \Phi_{ 1, 1 }{\left (z \right )} = 1\\
 \Phi_{ 1, 2 }{\left (z \right )} = z - 1\\
 \Phi_{ 1, 3 }{\left (z \right )} = \frac{z^{2}}{2} - z + \frac{1}{2}\\
 \Phi_{ 1, 4 }{\left (z \right )} = \frac{z^{3}}{6} - \frac{z^{2}}{2} + \frac{z}{2} - \frac{1}{6}\\
 \Phi_{ 1, 5 }{\left (z \right )} = \frac{z^{4}}{24} - \frac{z^{3}}{6} + \frac{z^{2}}{4} - \frac{z}{6} + \frac{1}{24}\\
 \Phi_{ 1, 6 }{\left (z \right )} = \frac{z^{5}}{120} - \frac{z^{4}}{24} + \frac{z^{3}}{12} - \frac{z^{2}}{12} + \frac{z}{24} - \frac{1}{120}\\
 \Phi_{ 1, 7 }{\left (z \right )} = \frac{z^{6}}{720} - \frac{z^{5}}{120} + \frac{z^{4}}{48} - \frac{z^{3}}{36} + \frac{z^{2}}{48} - \frac{z}{120} + \frac{1}{720}\\
 \Phi_{ 1, 8 }{\left (z \right )} = \frac{z^{7}}{5040} - \frac{z^{6}}{720} + \frac{z^{5}}{240} - \frac{z^{4}}{144} + \frac{z^{3}}{144} - \frac{z^{2}}{240} + \frac{z}{720} - \frac{1}{5040}\\
 %\Phi_{ 1, 9 }{\left (z \right )} = \frac{z^{8}}{40320} - \frac{z^{7}}{5040} + \frac{z^{6}}{1440} - \frac{z^{5}}{720} + \frac{z^{4}}{576} - \frac{z^{3}}{720} + \frac{z^{2}}{1440} - \frac{z}{5040} + \frac{1}{40320}\\
 %\Phi_{ 1, 10 }{\left (z \right )} = \frac{z^{9}}{362880} - \frac{z^{8}}{40320} + \frac{z^{7}}{10080} - \frac{z^{6}}{4320} + \frac{z^{5}}{2880} - \frac{z^{4}}{2880} + \frac{z^{3}}{4320} - \frac{z^{2}}{10080} + \frac{z}{40320} - \frac{1}{362880}\\
\end{array}
\end{displaymath}
Above equations can be compressed in matrix notation:
\begin{displaymath}
\left[\begin{matrix}1 & 0 & 0 & 0 & 0 & 0 & 0 & 0\\-1 & 1 & 0 & 0 & 0 & 0 & 0 & 0\\\frac{1}{2} & -1 & 1 & 0 & 0 & 0 & 0 & 0\\- \frac{1}{6} & \frac{1}{2} & -1 & 1 & 0 & 0 & 0 & 0\\\frac{1}{24} & - \frac{1}{6} & \frac{1}{2} & -1 & 1 & 0 & 0 & 0\\- \frac{1}{120} & \frac{1}{24} & - \frac{1}{6} & \frac{1}{2} & -1 & 1 & 0 & 0\\\frac{1}{720} & - \frac{1}{120} & \frac{1}{24} & - \frac{1}{6} & \frac{1}{2} & -1 & 1 & 0\\- \frac{1}{5040} & \frac{1}{720} & - \frac{1}{120} & \frac{1}{24} & - \frac{1}{6} & \frac{1}{2} & -1 & 1\end{matrix}\right] 
\end{displaymath}
multiplied on the right by the vector $\left[\begin{matrix}1\,z\,\frac{z^{2}}{2!}\,\frac{z^{3}}{3!}\,\frac{z^{4}}{4!}\,\frac{z^{5}}{5!}\,\frac{z^{6}}{6!}\,\frac{z^{7}}{7!}\end{matrix}\right]^{T}$
; the coefficient matrix is a \emph{Toeplitz matrix}.


\section{$f(z)=\frac{1}{z}$}

The general form of $j$th derivative of function $f$ is 
$$\frac{\partial^{(j)}{f}(z)}{\partial{z}} = \frac{(-1)^{j}j!}{z^{j+1}}$$ 
therefore
\begin{displaymath}
\begin{split}
  g(z) &= \sum_{j=1}^{m}{ \left. \frac{\partial^{(j-1)}{f}}{\partial{z}} \right|_{z=\lambda_{1}}\Phi_{1j}(z)} \\
       &= \sum_{j=1}^{m}{ \left. \frac{(-1)^{j-1}(j-1)!}{z^{j}} \right|_{z=\lambda_{1}}\Phi_{1j}(z)} \\
       &= \sum_{j=1}^{m}{ \left. \frac{1}{z^{j}} \right|_{z=1}(-1)^{j-1}(j-1)!\Phi_{1j}(z)} \\
       &= \sum_{j=1}^{m}{\sum_{k=0}^{j-1}{(-1)^{j-1}(j-1)!\frac{(-1)^{j-1-k}}{(j-1-k)!}\frac{z^{k}}{k!}}} \\
\end{split}
\end{displaymath}
yielding equation
\begin{equation}
  g(z) = \sum_{j=1}^{m}{\sum_{k=0}^{j-1}{{{j-1}\choose{k}}(-z)^{k}}} 
\end{equation}

To swap summations we reason on the relation $k=j-1$, which holds by the inner summation upper limit:
$k\in \lbrace 0,\ldots,m-1 \rbrace$ because last value for $j$ is $m$. Finally, to cover the same set 
of pairs $(j, k)$, when $k=0$ then $j\in \lbrace 1,\ldots,m \rbrace$ and when $k=m-1$ then 
$j\in \lbrace m \rbrace$, therefore $j\in \lbrace k+1, \ldots, m \rbrace$ is required, in the general case.
Binomial manipulations yield:
\begin{displaymath}
\begin{split}
  g(z) &= \sum_{k=0}^{m-1}{\left(\sum_{j=k+1}^{m}{{{j-1}\choose{k}}}\right)(-z)^{k}} \\
       &= \sum_{k=0}^{m-1}{\left(\sum_{j=k}^{m-1}{{{j}\choose{k}}}\right)(-z)^{k}} \\
\end{split}
\end{displaymath}
the inner sum admits a closed expression, yielding equation
\begin{equation}
  g(z) = \sum_{k=0}^{m-1}{{{m}\choose{k+1}}(-z)^{k}} \\
\end{equation}
namely, $\mathcal{R}_{m}^{-1}=g(\mathcal{R}_{m})$ for any proper, finite Riordan matrix $\mathcal{R}_{m}$.

Polynomial $g$ can also be written in closed form, assuming convergence condition $|z|\leq 1$:
\begin{displaymath}
g(z) = \frac{1- \left(1- z \right)^{m} }{z}
\end{displaymath}

For the sake of clarity, polynomial $g$ when $m=8$ and relaxing the condition $\lambda=1$, is defined according to 
\begin{displaymath}
\begin{split}
g{\left (z \right )} &= - \frac{z^{7}}{\lambda^{8}} \\
&+ z^{6} \left(\frac{1}{\lambda^{7}} + \frac{7}{\lambda^{8}}\right) \\
&+ z^{5} \left(- \frac{1}{\lambda^{6}} - \frac{6}{\lambda^{7}} - \frac{21}{\lambda^{8}}\right) \\
&+ z^{4} \left(\frac{1}{\lambda^{5}} + \frac{5}{\lambda^{6}} + \frac{15}{\lambda^{7}} + \frac{35}{\lambda^{8}}\right) \\
&+ z^{3} \left(- \frac{1}{\lambda^{4}} - \frac{4}{\lambda^{5}} - \frac{10}{\lambda^{6}} - \frac{20}{\lambda^{7}} - \frac{35}{\lambda^{8}}\right) \\
&+ z^{2} \left(\frac{1}{\lambda^{3}} + \frac{3}{\lambda^{4}} + \frac{6}{\lambda^{5}} + \frac{10}{\lambda^{6}} + \frac{15}{\lambda^{7}} + \frac{21}{\lambda^{8}}\right) \\
&+ z \left(- \frac{1}{\lambda^{2}} - \frac{2}{\lambda^{3}} - \frac{3}{\lambda^{4}} - \frac{4}{\lambda^{5}} - \frac{5}{\lambda^{6}} - \frac{6}{\lambda^{7}} - \frac{7}{\lambda^{8}}\right) \\
&+ \frac{1}{\lambda} + \frac{1}{\lambda^{2}} + \frac{1}{\lambda^{3}} + \frac{1}{\lambda^{4}} + \frac{1}{\lambda^{5}} + \frac{1}{\lambda^{6}} + \frac{1}{\lambda^{7}} + \frac{1}{\lambda^{8}}
\end{split}
\end{displaymath}
restoring $\lambda=1$ yields \[g{\left (z \right )} = - z^{7} + 8 z^{6} - 28 z^{5} + 56 z^{4} - 70 z^{3} + 56 z^{2} - 28 z + 8\]

A computation observation concerns the evaluation of $g(\mathcal{R}_{m})$, which should be carried out as
\begin{displaymath}
g(z) = z \left(z \left(z \left(z \left(z \left(z \left(- z + 8\right) - 28\right) + 56\right) - 70\right) + 56\right) - 28\right) + 8
\end{displaymath}
namely, according to the Horner rule for polynomials, interpreting each coefficient $c\in\mathbb{R}$ as $cI$ where $I\in\mathbb{C}^{m\times m}$ is the
identity matrix. Such approach requires $m-2$ matrix products and $m-1$ additions; finally, we implicitly use this scheme in all subsequent evaluation of a polynomial $g$ to a matrix $A$.



\section{$f(z)=z^{r}$}

The general form of $j$th derivative of function $f$ is 
$$\frac{\partial^{(j)}{f}(z)}{\partial{z}} = (r)_{(j)} z^{r-j}$$ 
where $(r)_{(j)} = r(r-1)\cdots(r-j+1)$ is the falling factorial, therefore
\begin{displaymath}
\begin{split}
  g(z) &= \sum_{j=1}^{m}{ \left. \frac{\partial^{(j-1)}{f}}{\partial{z}} \right|_{z=\lambda_{1}}\Phi_{1j}(z)} \\
       &= \sum_{j=1}^{m}{ \left. (r)_{(j-1)} z^{r-j+1} \right|_{z=\lambda_{1}}\Phi_{1j}(z)} \\
       &= \sum_{j=1}^{m}{ \left. z^{r-j+1} \right|_{z=\lambda_{1}}(r)_{(j-1)} \Phi_{1j}(z)} \\
       &= \sum_{j=1}^{m}{\sum_{k=0}^{j-1}{\frac{(r)_{(j-1)}}{(j-1)_{(j-1)}}\frac{(j-1)!(-1)^{j-1-k}}{(j-1-k)!}\frac{z^{k}}{k!}}} \\
\end{split}
\end{displaymath}
yielding equation
\begin{equation}
  g(z) = \sum_{j=1}^{m}{\sum_{k=0}^{j-1}{(-1)^{j-1}{{r}\choose{j-1}}{{j-1}\choose{k}}(-z)^{k}}} 
\end{equation}
We swap summations holding the same argument explained in previous section:
\begin{displaymath}
  g(z) = \sum_{k=0}^{m-1}{\left(\sum_{j=k+1}^{m}{(-1)^{j-1}{{r}\choose{j-1}}{{j-1}\choose{k}}}\right)(-z)^{k}}
\end{displaymath}
yielding equation
\begin{equation}
  g(z) = \sum_{k=0}^{m-1}{\left(\sum_{j=k}^{m-1}{(-1)^{j}{{r}\choose{j}}{{j}\choose{k}}}\right)(-z)^{k}}
\end{equation}
The above expression holds without conditions on both $r$ and $m$; however it is possible 
to find a closed expression for the inner sum:
\begin{eqnarray}
  g(z) &= \sum_{k=0}^{m-1}{\left(\left(-1\right)^{m}\frac{ k - m }{r-k}{\binom{m}{k}} {\binom{r}{m}}\right)(-z)^{k}}\\
       &= \sum_{k=0}^{m-1}{\left(\frac{ k - m }{r-k}{\binom{m}{k}} {\binom{m-r-1}{m}}\right)(-z)^{k}}
\end{eqnarray}
Last expression is defined unless $r=k$, namely $\mathcal{R}_{m}^{r}=g(\mathcal{R}_{m})$ for any proper, 
finite Riordan matrix $\mathcal{R}_{m}$, where $m\leq r$. On the other hand, for $m>r$ we have 
$g(z)=z^{r}$, which agrees with intuition because computing $g(\mathcal{R}_{m})$, for $m$ much bigger than $r$, 
requires computing combination of powers $\mathcal{R}_{m}^{j}$ for $j\in \lbrace 0,\ldots,r \rbrace$ at least,
which outperforms computing $\mathcal{R}_{m}^{r}$ as a whole.

For the sake of clarity, let $m=8$ to define polynomial $g$:
\begin{displaymath}
\begin{split}
g{\left (z \right )} &= z^{7} {\binom{r}{7}} \\
&+ z^{6} \left({\binom{r}{6}} - 7 {\binom{r}{7}}\right) \\
&+ z^{5} \left({\binom{r}{5}} - 6 {\binom{r}{6}} + 21 {\binom{r}{7}}\right) \\
&+ z^{4} \left({\binom{r}{4}} - 5 {\binom{r}{5}} + 15 {\binom{r}{6}} - 35 {\binom{r}{7}}\right) \\
&+ z^{3} \left({\binom{r}{3}} - 4 {\binom{r}{4}} + 10 {\binom{r}{5}} - 20 {\binom{r}{6}} + 35 {\binom{r}{7}}\right) \\
&+ z^{2} \left({\binom{r}{2}} - 3 {\binom{r}{3}} + 6 {\binom{r}{4}} - 10 {\binom{r}{5}} + 15 {\binom{r}{6}} - 21 {\binom{r}{7}}\right) \\
&+ z \left({\binom{r}{1}} - 2 {\binom{r}{2}} + 3 {\binom{r}{3}} - 4 {\binom{r}{4}} + 5 {\binom{r}{5}} - 6 {\binom{r}{6}} + 7 {\binom{r}{7}}\right) \\
&- {\binom{r}{1}} + {\binom{r}{2}} - {\binom{r}{3}} + {\binom{r}{4}} - {\binom{r}{5}} + {\binom{r}{6}} - {\binom{r}{7}} + 1 \\
\end{split}
\end{displaymath}
expansion of inner binomial coefficients yields
\begin{displaymath}
\begin{split}
g{\left (z \right )} &= - \frac{r^{7}}{5040} + \frac{r^{6}}{180} - \frac{23 r^{5}}{360} + \frac{7 r^{4}}{18} - \frac{967 r^{3}}{720} + \frac{469 r^{2}}{180} - \frac{363 r}{140} \\
&+ z^{7} \left(\frac{r^{7}}{5040} - \frac{r^{6}}{240} + \frac{5 r^{5}}{144} - \frac{7 r^{4}}{48} + \frac{29 r^{3}}{90} - \frac{7 r^{2}}{20} + \frac{r}{7}\right) \\
&+ z^{6} \left(- \frac{r^{7}}{720} + \frac{11 r^{6}}{360} - \frac{19 r^{5}}{72} + \frac{41 r^{4}}{36} - \frac{1849 r^{3}}{720} + \frac{1019 r^{2}}{360} - \frac{7 r}{6}\right) \\
&+ z^{5} \left(\frac{r^{7}}{240} - \frac{23 r^{6}}{240} + \frac{69 r^{5}}{80} - \frac{185 r^{4}}{48} + \frac{134 r^{3}}{15} - \frac{201 r^{2}}{20} + \frac{21 r}{5}\right) \\
&+ z^{4} \left(- \frac{r^{7}}{144} + \frac{r^{6}}{6} - \frac{113 r^{5}}{72} + \frac{22 r^{4}}{3} - \frac{2545 r^{3}}{144} + \frac{41 r^{2}}{2} - \frac{35 r}{4}\right) \\
&+ z^{3} \left(\frac{r^{7}}{144} - \frac{25 r^{6}}{144} + \frac{247 r^{5}}{144} - \frac{1219 r^{4}}{144} + \frac{389 r^{3}}{18} - \frac{949 r^{2}}{36} + \frac{35 r}{3}\right) \\
&+ z^{2} \left(- \frac{r^{7}}{240} + \frac{13 r^{6}}{120} - \frac{9 r^{5}}{8} + \frac{71 r^{4}}{12} - \frac{3929 r^{3}}{240} + \frac{879 r^{2}}{40} - \frac{21 r}{2}\right) \\
&+ z \left(\frac{r^{7}}{720} - \frac{3 r^{6}}{80} + \frac{59 r^{5}}{144} - \frac{37 r^{4}}{16} + \frac{319 r^{3}}{45} - \frac{223 r^{2}}{20} + 7 r\right) + 1
\end{split}
\end{displaymath}
Moreover, using last two equations and requiring $r \geq 8$, we have:
\begin{displaymath}
\begin{split}
g{\left (z \right )} &= 8 {\binom{r}{8}} \left( \frac{z^{7}}{r - 7} - \frac{7 z^{6}}{r - 6} + \frac{21 z^{5}}{r - 5}\right. \\
& \left. - \frac{35 z^{4}}{r - 4} + \frac{35 z^{3}}{r - 3} - \frac{21 z^{2}}{r - 2} + \frac{7 z}{r - 1} - \frac{1}{r} \right) \\
&= 8 {\binom{7-r}{8}} \left( \frac{z^{7}}{r - 7} - \frac{7 z^{6}}{r - 6} + \frac{21 z^{5}}{r - 5}\right. \\
& \left. - \frac{35 z^{4}}{r - 4} + \frac{35 z^{3}}{r - 3} - \frac{21 z^{2}}{r - 2} + \frac{7 z}{r - 1} - \frac{1}{r} \right) \\
\end{split}
\end{displaymath}
respectively.


\section{$f(z)=\sqrt{z}$}

The general form of $j$th derivative of function $f$ is 
$$\frac{\partial^{(j)}{f}(z)}{\partial{z}} =\frac{(-1)^{j-1}}{2}\frac{(j-1)!}{4^{j-1}}{{2(j-1)}\choose{j-1}}\frac{1}{z^{\frac{2(j-1)+1}{2}}} $$ 
for $j \geq 1$, therefore
\begin{displaymath}
\begin{split}
  g(z) &= \sum_{j=1}^{m}{ \left. \frac{\partial^{(j-1)}{f}}{\partial{z}} \right|_{z=\lambda_{1}}\Phi_{1j}(z)} \\
       &= \sum_{j=0}^{m-1}{ \left. \frac{\partial^{(j)}{f}}{\partial{z}} \right|_{z=\lambda_{1}}\Phi_{1,j+1}(z)} \\
       &= f(1)\Phi_{11}(z) + \sum_{j=1}^{m-1}{ \left. \frac{\partial^{(j)}{f}}{\partial{z}} \right|_{z=\lambda_{1}}\Phi_{1,j+1}(z)} \\
\end{split}
\end{displaymath}
Observing that $\Phi_{11}(z)=1$ and
\begin{displaymath}
  \Phi_{1,j+1}(z) = \sum_{k=0}^{j}{\frac{(-1)^{j-k}}{(j-k)!}\frac{z^{k}}{k!}}
\end{displaymath}
it allows us to state the equation
\begin{equation}
  g(z) = 1 +\frac{1}{2} \sum_{j=1}^{m-1}{\sum_{k=0}^{j}{-\frac{1}{j 4^{j-1}} {{2(j-1)}\choose{j-1}}{{j}\choose{k}} (-z)^{k}}}
\end{equation}
To swap summations, for covering the same set of pairs $(j, k)$ it is required that 
$j\in \lbrace 1, \ldots, m-1 \rbrace$ for both $k=0$ and $k=1$, therefore we split
the outer sum over $k$ extracting the very first term due to $k=0$:
\begin{equation}
\begin{split}
  g(z) &= 1 +\frac{1}{2} \sum_{k=1}^{m-1}{\sum_{j=k}^{m-1}{-\frac{1}{j 4^{j-1}} {{2(j-1)}\choose{j-1}}{{j}\choose{k}} (-z)^{k}}} \\
       &+ \frac{1}{2}\sum_{j=1}^{m-1}{-\frac{1}{j 4^{j-1}} {{2(j-1)}\choose{j-1}}}
\end{split}
\end{equation}
compactly using Kroneker delta in the inner sum's starting:
\begin{equation}
\begin{split}
  g(z) &= 1 +\frac{1}{2} \sum_{k=0}^{m-1}{\sum_{j=k+\delta_{k0}}^{m-1}{-\frac{1}{j 4^{j-1}} {{2(j-1)}\choose{j-1}}{{j}\choose{k}} (-z)^{k}}} \\
\end{split}
\end{equation}

\iffalse %  {{{
Finding closed expressions for the inner sum and independent coefficient yields:
\begin{equation}
\begin{split}
  g(z) &= \frac{2 }{4^{m}(2 m - 1)}\binom{2 m}{m}\sum_{k=1}^{m-1}{\frac{ \left(k - m\right) }{\left(2 k - 1\right) }{\binom{m}{k}} { (-z)^{k}}} \\
       &+ \frac{2 m }{4^{m}(2 m - 1)}{\binom{2 m}{m}}
\end{split}
\end{equation}
we can put the coefficient back inside sum, so the final expression is:
\begin{displaymath}
\begin{split}
  g(z) &= \frac{2 }{4^{m}(2 m - 1)}\binom{2 m}{m}\sum_{k=0}^{m-1}{\frac{ \left(k - m\right) }{\left(2 k - 1\right) }{\binom{m}{k}} { (-z)^{k}}} \\
\end{split}
\end{displaymath}
namely $\sqrt{\mathcal{R}_{m}}=g(\mathcal{R}_{m})$ for any proper, finite Riordan matrix $\mathcal{R}_{m}$.

\fi
% }}}

For the sake of clarity, here is $g$ polynomial where $m=8$ and relaxing $\lambda=1$ hypothesis:
\begin{displaymath}
\begin{split}
g{\left (z \right )} &= \frac{33 z^{7}}{2048 \lambda^{\frac{13}{2}}} \\
&+ z^{6} \left(- \frac{21}{1024 \lambda^{\frac{11}{2}}} - \frac{231}{2048 \lambda^{\frac{13}{2}}}\right) \\
&+ z^{5} \left(\frac{7}{256 \lambda^{\frac{9}{2}}} + \frac{63}{512 \lambda^{\frac{11}{2}}} + \frac{693}{2048 \lambda^{\frac{13}{2}}}\right) \\
&+ z^{4} \left(- \frac{5}{128 \lambda^{\frac{7}{2}}} - \frac{35}{256 \lambda^{\frac{9}{2}}} - \frac{315}{1024 \lambda^{\frac{11}{2}}} - \frac{1155}{2048 \lambda^{\frac{13}{2}}}\right) \\
&+ z^{3} \left(\frac{1}{16 \lambda^{\frac{5}{2}}} + \frac{5}{32 \lambda^{\frac{7}{2}}} + \frac{35}{128 \lambda^{\frac{9}{2}}} + \frac{105}{256 \lambda^{\frac{11}{2}}} + \frac{1155}{2048 \lambda^{\frac{13}{2}}}\right) \\
&+ z^{2} \left(- \frac{1}{8 \lambda^{\frac{3}{2}}} - \frac{3}{16 \lambda^{\frac{5}{2}}} - \frac{15}{64 \lambda^{\frac{7}{2}}} - \frac{35}{128 \lambda^{\frac{9}{2}}} - \frac{315}{1024 \lambda^{\frac{11}{2}}} - \frac{693}{2048 \lambda^{\frac{13}{2}}}\right) \\
&+ z \left(\frac{1}{2 \sqrt{\lambda}} + \frac{1}{4 \lambda^{\frac{3}{2}}} + \frac{3}{16 \lambda^{\frac{5}{2}}} + \frac{5}{32 \lambda^{\frac{7}{2}}} + \frac{35}{256 \lambda^{\frac{9}{2}}} \right. \\
    &+ \left. \frac{63}{512 \lambda^{\frac{11}{2}}} + \frac{231}{2048 \lambda^{\frac{13}{2}}}\right) \\
&+ \sqrt{\lambda} - \frac{1}{2 \sqrt{\lambda}} - \frac{1}{8 \lambda^{\frac{3}{2}}} - \frac{1}{16 \lambda^{\frac{5}{2}}} - \frac{5}{128 \lambda^{\frac{7}{2}}} - \frac{7}{256 \lambda^{\frac{9}{2}}} \\
&- \frac{21}{1024 \lambda^{\frac{11}{2}}} - \frac{33}{2048 \lambda^{\frac{13}{2}}}
\end{split}
\end{displaymath}
restoring $\lambda=1$:
\begin{displaymath}
\begin{split}
g{\left (z \right )} &= \frac{33 z^{7}}{2048} - \frac{273 z^{6}}{2048} + \frac{1001 z^{5}}{2048} - \frac{2145 z^{4}}{2048} + \frac{3003 z^{3}}{2048} \\
&- \frac{3003 z^{2}}{2048} + \frac{3003 z}{2048} + \frac{429}{2048}
\end{split}
\end{displaymath}

\section{$f(z)=e^{\alpha z}$}

The general form of $j$th derivative of function $f$ is 
$$\frac{\partial^{(j)}{f}(z)}{\partial{z}} = \alpha^{j} e^{\alpha z}$$ 
therefore
\begin{displaymath}
\begin{split}
  g(z) &= \sum_{j=1}^{m}{ \left. \frac{\partial^{(j-1)}{f}}{\partial{z}} \right|_{z=\lambda_{1}}\Phi_{1j}(z)} \\
       &= \sum_{j=1}^{m}{ \left. \alpha^{j-1} e^{\alpha z} \right|_{z=\lambda_{1}}\Phi_{1j}(z)} \\
       &= \sum_{j=1}^{m}{ \left. e^{\alpha z} \right|_{z=\lambda_{1}}\alpha^{j-1} \Phi_{1j}(z)} \\
       &= e^{\alpha}\sum_{j=1}^{m}{\sum_{k=0}^{j-1}{\frac{\alpha^{j-1}}{(j-1)!}  \frac{(j-1)!(-1)^{j-1-k}}{(j-1-k)!}\frac{z^{k}}{k!}}}\\
\end{split}
\end{displaymath}
yielding equation
\begin{equation}
  g(z) = e^{\alpha}\sum_{j=1}^{m}{\sum_{k=0}^{j-1}{\frac{(-\alpha)^{j-1}}{(j-1)!}{{j-1}\choose{k}}(-z)^{k}}} 
\end{equation}
We swap summations holding the same argument explained in previous section:
\begin{displaymath}
  g(z) = e^{\alpha}\sum_{k=0}^{m-1}{\left(\sum_{j=k+1}^{m}{\frac{(-\alpha)^{j-1}}{(j-1)!}{{j-1}\choose{k}}}\right)(-z)^{k}}
\end{displaymath}
yielding equation
\begin{equation}
  g(z) = e^{\alpha}\sum_{k=0}^{m-1}{\left(\sum_{j=k}^{m-1}{\frac{(-\alpha)^{j}}{j!}{{j}\choose{k}}}\right)(-z)^{k}}
\end{equation}
namely $e^{\alpha\mathcal{R}_{m}}=g(\mathcal{R}_{m})$ for any proper, 
finite Riordan matrix $\mathcal{R}_{m}$.

For the sake of clarity, let $m=8$ to define polynomial $g$:
\begin{displaymath}
\begin{split}
g{\left (z \right )} &= e^{\alpha}\left(\frac{\alpha^{7} z^{7}}{5040}\right. \\
&+ \frac{\alpha^{6} z^{6}}{720} \left(- \alpha + 1\right) \\
&+ \frac{\alpha^{5} z^{5}}{240} \left(\alpha^{2} - 2 \alpha + 2\right) \\
&+ \frac{\alpha^{4} z^{4}}{144} \left(- \alpha^{3} + 3 \alpha^{2} - 6 \alpha + 6\right) \\
&+ \frac{\alpha^{3} z^{3}}{144} \left(\alpha^{4} - 4 \alpha^{3} + 12 \alpha^{2} - 24 \alpha + 24\right) \\
&+ \frac{\alpha^{2} z^{2}}{240} \left(- \alpha^{5} + 5 \alpha^{4} - 20 \alpha^{3} + 60 \alpha^{2} - 120 \alpha + 120\right) \\
&+ \frac{\alpha z}{720} \left(\alpha^{6} - 6 \alpha^{5} + 30 \alpha^{4} - 120 \alpha^{3} + 360 \alpha^{2} - 720 \alpha + 720\right) \\
&- \left.\frac{\alpha^{7}}{5040} + \frac{\alpha^{6}}{720} - \frac{\alpha^{5}}{120} + \frac{\alpha^{4}}{24} - \frac{\alpha^{3}}{6} + \frac{\alpha^{2}}{2} -\alpha + 1\right) 
\end{split}
\end{displaymath}
choosing $\alpha=1$ yields:
$$ g{\left (z \right )} = e \left(\frac{z^{7}}{5040} + \frac{z^{5}}{240} + \frac{z^{4}}{72} + \frac{z^{3}}{16} + \frac{11 z^{2}}{60} + \frac{53 z}{144} + \frac{103}{280}\right) $$
on the other hand, choosing $\alpha=-1$ yields:
$$ e\,g{\left (z \right )} = - \frac{z^{7}}{5040} + \frac{z^{6}}{360} - \frac{z^{5}}{48} + \frac{z^{4}}{9} - \frac{65 z^{3}}{144} + \frac{163 z^{2}}{120} - \frac{1957 z}{720} + \frac{685}{252} $$

\section{Case studies}

\subsection{Pascal array $\mathcal{P}$}

Let $m=8$ and define
\begin{displaymath}
\mathcal{P}_{m}=\left[\begin{matrix}1 & 0 & 0 & 0 & 0 & 0 & 0 & 0\\1 & 1 & 0 & 0 & 0 & 0 & 0 & 0\\1 & 2 & 1 & 0 & 0 & 0 & 0 & 0\\1 & 3 & 3 & 1 & 0 & 0 & 0 & 0\\1 & 4 & 6 & 4 & 1 & 0 & 0 & 0\\1 & 5 & 10 & 10 & 5 & 1 & 0 & 0\\1 & 6 & 15 & 20 & 15 & 6 & 1 & 0\\1 & 7 & 21 & 35 & 35 & 21 & 7 & 1\end{matrix}\right]
\end{displaymath}
therefore $\mathcal{P}_{m}^{-1}$ is
\begin{displaymath}
\left[\begin{matrix}1 & 0 & 0 & 0 & 0 & 0 & 0 & 0\\-1 & 1 & 0 & 0 & 0 & 0 & 0 & 0\\1 & -2 & 1 & 0 & 0 & 0 & 0 & 0\\-1 & 3 & -3 & 1 & 0 & 0 & 0 & 0\\1 & -4 & 6 & -4 & 1 & 0 & 0 & 0\\-1 & 5 & -10 & 10 & -5 & 1 & 0 & 0\\1 & -6 & 15 & -20 & 15 & -6 & 1 & 0\\-1 & 7 & -21 & 35 & -35 & 21 & -7 & 1\end{matrix}\right]
\end{displaymath}
$\mathcal{P}_{m}^{r}$ is
\begin{displaymath}
\left[\begin{matrix}1 & 0 & 0 & 0 & 0 & 0 & 0 & 0\\r & 1 & 0 & 0 & 0 & 0 & 0 & 0\\r^{2} & 2 r & 1 & 0 & 0 & 0 & 0 & 0\\r^{3} & 3 r^{2} & 3 r & 1 & 0 & 0 & 0 & 0\\r^{4} & 4 r^{3} & 6 r^{2} & 4 r & 1 & 0 & 0 & 0\\r^{5} & 5 r^{4} & 10 r^{3} & 10 r^{2} & 5 r & 1 & 0 & 0\\r^{6} & 6 r^{5} & 15 r^{4} & 20 r^{3} & 15 r^{2} & 6 r & 1 & 0\\r^{7} & 7 r^{6} & 21 r^{5} & 35 r^{4} & 35 r^{3} & 21 r^{2} & 7 r & 1\end{matrix}\right]
\end{displaymath}
$\sqrt{\mathcal{P}_{m}}$ is
\begin{displaymath}
\indent\indent\indent \left[\begin{matrix}1 & 0 & 0 & 0 & 0 & 0 & 0 & 0\\\frac{1}{2} & 1 & 0 & 0 & 0 & 0 & 0 & 0\\\frac{1}{4} & 1 & 1 & 0 & 0 & 0 & 0 & 0\\\frac{1}{8} & \frac{3}{4} & \frac{3}{2} & 1 & 0 & 0 & 0 & 0\\\frac{1}{16} & \frac{1}{2} & \frac{3}{2} & 2 & 1 & 0 & 0 & 0\\\frac{1}{32} & \frac{5}{16} & \frac{5}{4} & \frac{5}{2} & \frac{5}{2} & 1 & 0 & 0\\\frac{1}{64} & \frac{3}{16} & \frac{15}{16} & \frac{5}{2} & \frac{15}{4} & 3 & 1 & 0\\\frac{1}{128} & \frac{7}{64} & \frac{21}{32} & \frac{35}{16} & \frac{35}{8} & \frac{21}{4} & \frac{7}{2} & 1\end{matrix}\right]
\end{displaymath}
finally, $e^{\mathcal{P}_{m}}$ is
\begin{displaymath}
e \left[\begin{matrix}1 & 0 & 0 & 0 & 0 & 0 & 0 & 0\\1 & 1 & 0 & 0 & 0 & 0 & 0 & 0\\2 & 2 & 1 & 0 & 0 & 0 & 0 & 0\\5 & 6 & 3 & 1 & 0 & 0 & 0 & 0\\15 & 20 & 12 & 4 & 1 & 0 & 0 & 0\\52 & 75 & 50 & 20 & 5 & 1 & 0 & 0\\203 & 312 & 225 & 100 & 30 & 6 & 1 & 0\\877 & 1421 & 1092 & 525 & 175 & 42 & 7 & 1\end{matrix}\right]
\end{displaymath}
such matrix is known as $A056857$ in the OEIS, there we found
the interesting relation $e^{\mathcal{P}_{m}}=e\left(S_{m}\cdot P\cdot S_{m}^{-1}\right)$,
where $S_{m}$ is the matrix of Stirling numbers of the second kind, defined in a later case study.

Also, it is interesting the application of function $f(z)=e^{\alpha z}$ to the matrix $H$ defined as
\begin{displaymath}
H = \left[\begin{matrix}0 & 0 & 0 & 0 & 0 & 0 & 0 & 0\\1 & 0 & 0 & 0 & 0 & 0 & 0 & 0\\0 & 2 & 0 & 0 & 0 & 0 & 0 & 0\\0 & 0 & 3 & 0 & 0 & 0 & 0 & 0\\0 & 0 & 0 & 4 & 0 & 0 & 0 & 0\\0 & 0 & 0 & 0 & 5 & 0 & 0 & 0\\0 & 0 & 0 & 0 & 0 & 6 & 0 & 0\\0 & 0 & 0 & 0 & 0 & 0 & 7 & 0\end{matrix}\right]
\end{displaymath}
It requires the generalized Langrange base
\begin{displaymath}
\begin{split}
&\left\{\Phi_{ 1, 1 }{\left (z \right )} = 1, \Phi_{ 1, 2 }{\left (z \right )} = z, \Phi_{ 1, 3 }{\left (z \right )} = \frac{z^{2}}{2},\right. \\
&\Phi_{ 1, 4 }{\left (z \right )} = \frac{z^{3}}{6}, \Phi_{ 1, 5 }{\left (z \right )} = \frac{z^{4}}{24}, \Phi_{ 1, 6 }{\left (z \right )} = \frac{z^{5}}{120}, \\
&\left.\Phi_{ 1, 7 }{\left (z \right )} = \frac{z^{6}}{720}, \Phi_{ 1, 8 }{\left (z \right )} = \frac{z^{7}}{5040}\right\}
\end{split}
\end{displaymath}
to define the interpolating polynomial $g$ as
\begin{displaymath}
\begin{split}
g{\left (z \right )} &= \frac{\alpha^{7} z^{7}}{5040} + \frac{\alpha^{6} z^{6}}{720} + \frac{\alpha^{5} z^{5}}{120} + \frac{\alpha^{4} z^{4}}{24} \\
&+ \frac{\alpha^{3} z^{3}}{6} + \frac{\alpha^{2} z^{2}}{2} + \alpha z + 1
\end{split}
\end{displaymath}
therefore $g(H)=e^{\alpha H}$ is the matrix
\begin{displaymath}
\left[\begin{matrix}1 & 0 & 0 & 0 & 0 & 0 & 0 & 0\\\alpha & 1 & 0 & 0 & 0 & 0 & 0 & 0\\\alpha^{2} & 2 \alpha & 1 & 0 & 0 & 0 & 0 & 0\\\alpha^{3} & 3 \alpha^{2} & 3 \alpha & 1 & 0 & 0 & 0 & 0\\\alpha^{4} & 4 \alpha^{3} & 6 \alpha^{2} & 4 \alpha & 1 & 0 & 0 & 0\\\alpha^{5} & 5 \alpha^{4} & 10 \alpha^{3} & 10 \alpha^{2} & 5 \alpha & 1 & 0 & 0\\\alpha^{6} & 6 \alpha^{5} & 15 \alpha^{4} & 20 \alpha^{3} & 15 \alpha^{2} & 6 \alpha & 1 & 0\\\alpha^{7} & 7 \alpha^{6} & 21 \alpha^{5} & 35 \alpha^{4} & 35 \alpha^{3} & 21 \alpha^{2} & 7 \alpha & 1\end{matrix}\right]
\end{displaymath}
yielding the identity $e^{\alpha H} = \mathcal{P}_{m}^{\alpha}$;
moreover $e^{(\alpha+\beta) H} = \mathcal{P}_{m}^{\alpha}\mathcal{P}_{m}^{\beta} = \mathcal{P}_{m}^{\alpha+\beta}$ also holds.


\subsection{Catalan array $\mathcal{C}$}

Let $m=8$ and define
\begin{displaymath}
\mathcal{C}_{m}=\left[\begin{matrix}1 & 0 & 0 & 0 & 0 & 0 & 0 & 0\\1 & 1 & 0 & 0 & 0 & 0 & 0 & 0\\2 & 2 & 1 & 0 & 0 & 0 & 0 & 0\\5 & 5 & 3 & 1 & 0 & 0 & 0 & 0\\14 & 14 & 9 & 4 & 1 & 0 & 0 & 0\\42 & 42 & 28 & 14 & 5 & 1 & 0 & 0\\132 & 132 & 90 & 48 & 20 & 6 & 1 & 0\\429 & 429 & 297 & 165 & 75 & 27 & 7 & 1\end{matrix}\right]
\end{displaymath}
therefore $\mathcal{C}_{m}^{-1}$ is
\begin{displaymath}
\left[\begin{matrix}1 & 0 & 0 & 0 & 0 & 0 & 0 & 0\\-1 & 1 & 0 & 0 & 0 & 0 & 0 & 0\\0 & -2 & 1 & 0 & 0 & 0 & 0 & 0\\0 & 1 & -3 & 1 & 0 & 0 & 0 & 0\\0 & 0 & 3 & -4 & 1 & 0 & 0 & 0\\0 & 0 & -1 & 6 & -5 & 1 & 0 & 0\\0 & 0 & 0 & -4 & 10 & -6 & 1 & 0\\0 & 0 & 0 & 1 & -10 & 15 & -7 & 1\end{matrix}\right]
\end{displaymath}
since $\mathcal{C}_{m}^{r}$ is too big we report $\mathcal{C}_{m}^{r}\textbf{e}_{1}$ 
\begin{displaymath}
%\left[\begin{matrix}1 & 0 & 0 & 0 & 0 & 0 & 0 & 0\\r & 1 & 0 & 0 & 0 & 0 & 0 & 0\\r \left(r + 1\right) & 2 r & 1 & 0 & 0 & 0 & 0 & 0\\\frac{r}{2} \left(2 r^{2} + 5 r + 3\right) & r \left(3 r + 2\right) & 3 r & 1 & 0 & 0 & 0 & 0\\\frac{r}{3} \left(3 r^{3} + 13 r^{2} + 18 r + 8\right) & r \left(4 r^{2} + 7 r + 3\right) & 3 r \left(2 r + 1\right) & 4 r & 1 & 0 & 0 & 0\\\frac{r}{12} \left(12 r^{4} + 77 r^{3} + 178 r^{2} + 175 r + 62\right) & \frac{r}{3} \left(15 r^{3} + 47 r^{2} + 48 r + 16\right) & \frac{r}{2} \left(20 r^{2} + 27 r + 9\right) & 2 r \left(5 r + 2\right) & 5 r & 1 & 0 & 0\\\frac{r}{30} \left(30 r^{5} + 261 r^{4} + 875 r^{3} + 1405 r^{2} + 1075 r + 314\right) & \frac{r}{6} \left(36 r^{4} + 171 r^{3} + 298 r^{2} + 225 r + 62\right) & r \left(15 r^{3} + 37 r^{2} + 30 r + 8\right) & 2 r \left(10 r^{2} + 11 r + 3\right) & 5 r \left(3 r + 1\right) & 6 r & 1 & 0\\\frac{r}{60} \left(60 r^{6} + 669 r^{5} + 3002 r^{4} + 6900 r^{3} + 8510 r^{2} + 5301 r + 1298\right) & \frac{r}{60} \left(420 r^{5} + 2754 r^{4} + 7075 r^{3} + 8860 r^{2} + 5375 r + 1256\right) & \frac{r}{4} \left(84 r^{4} + 319 r^{3} + 448 r^{2} + 275 r + 62\right) & \frac{r}{3} \left(105 r^{3} + 214 r^{2} + 144 r + 32\right) & \frac{5 r}{2} \left(14 r^{2} + 13 r + 3\right) & 3 r \left(7 r + 2\right) & 7 r & 1\end{matrix}\right]
\left[\begin{matrix}1 \\r \\r \left(r + 1\right) \\\frac{r}{2} \left(2 r^{2} + 5 r + 3\right) \\\frac{r}{3} \left(3 r^{3} + 13 r^{2} + 18 r + 8\right) \\\frac{r}{12} \left(12 r^{4} + 77 r^{3} + 178 r^{2} + 175 r + 62\right) \\\frac{r}{30} \left(30 r^{5} + 261 r^{4} + 875 r^{3} + 1405 r^{2} + 1075 r + 314\right) \\\frac{r}{60} \left(60 r^{6} + 669 r^{5} + 3002 r^{4} + 6900 r^{3} + 8510 r^{2} + 5301 r + 1298\right) \end{matrix}\right]
\end{displaymath}
$\sqrt{\mathcal{C}_{m}}$ is
\begin{displaymath}
\left[\begin{matrix}1 & 0 & 0 & 0 & 0 & 0 & 0 & 0\\\frac{1}{2} & 1 & 0 & 0 & 0 & 0 & 0 & 0\\\frac{3}{4} & 1 & 1 & 0 & 0 & 0 & 0 & 0\\\frac{3}{2} & \frac{7}{4} & \frac{3}{2} & 1 & 0 & 0 & 0 & 0\\\frac{55}{16} & \frac{15}{4} & 3 & 2 & 1 & 0 & 0 & 0\\\frac{545}{64} & \frac{143}{16} & \frac{55}{8} & \frac{9}{2} & \frac{5}{2} & 1 & 0 & 0\\\frac{709}{32} & \frac{727}{32} & \frac{273}{16} & 11 & \frac{25}{4} & 3 & 1 & 0\\\frac{15249}{256} & \frac{3855}{64} & \frac{2853}{64} & \frac{455}{16} & \frac{65}{4} & \frac{33}{4} & \frac{7}{2} & 1\end{matrix}\right]
\end{displaymath}
finally, $e^{\mathcal{C}_{m}}$ is
\begin{displaymath}
e \left[\begin{matrix}1 & 0 & 0 & 0 & 0 & 0 & 0 & 0\\1 & 1 & 0 & 0 & 0 & 0 & 0 & 0\\3 & 2 & 1 & 0 & 0 & 0 & 0 & 0\\\frac{23}{2} & 8 & 3 & 1 & 0 & 0 & 0 & 0\\\frac{154}{3} & 37 & 15 & 4 & 1 & 0 & 0 & 0\\\frac{1027}{4} & \frac{572}{3} & \frac{163}{2} & 24 & 5 & 1 & 0 & 0\\\frac{7046}{5} & \frac{6439}{6} & 478 & 150 & 35 & 6 & 1 & 0\\\frac{502481}{60} & \frac{390899}{60} & \frac{12005}{4} & \frac{2965}{3} & \frac{495}{2} & 48 & 7 & 1\end{matrix}\right]
\end{displaymath}
evaluating corresponding $g$ polynomials on $\mathcal{C}_{m}$, as required.

\subsection{Stirling array $\mathcal{S}$}

Let $m=8$ and define
\begin{displaymath}
\mathcal{S}_{m} = \left[\begin{matrix}1 & 0 & 0 & 0 & 0 & 0 & 0 & 0\\0 & 1 & 0 & 0 & 0 & 0 & 0 & 0\\0 & 1 & 1 & 0 & 0 & 0 & 0 & 0\\0 & 1 & 3 & 1 & 0 & 0 & 0 & 0\\0 & 1 & 7 & 6 & 1 & 0 & 0 & 0\\0 & 1 & 15 & 25 & 10 & 1 & 0 & 0\\0 & 1 & 31 & 90 & 65 & 15 & 1 & 0\\0 & 1 & 63 & 301 & 350 & 140 & 21 & 1\end{matrix}\right]
\end{displaymath}
therefore $\mathcal{S}_{m}^{-1}$ is
\begin{displaymath}
%\quad\quad\quad\quad\quad\quad
\left[\begin{matrix}1 & 0 & 0 & 0 & 0 & 0 & 0 & 0\\0 & 1 & 0 & 0 & 0 & 0 & 0 & 0\\0 & -1 & 1 & 0 & 0 & 0 & 0 & 0\\0 & 2 & -3 & 1 & 0 & 0 & 0 & 0\\0 & -6 & 11 & -6 & 1 & 0 & 0 & 0\\0 & 24 & -50 & 35 & -10 & 1 & 0 & 0\\0 & -120 & 274 & -225 & 85 & -15 & 1 & 0\\0 & 720 & -1764 & 1624 & -735 & 175 & -21 & 1\end{matrix}\right]
\end{displaymath}
since $\mathcal{S}_{m}^{r}$ is too big we report $\mathcal{S}_{m}^{r}\textbf{e}_{2}$ 
\begin{displaymath}
%\left[\begin{matrix}1 & 0 & 0 & 0 & 0 & 0 & 0 & 0\\0 & 1 & 0 & 0 & 0 & 0 & 0 & 0\\0 & r & 1 & 0 & 0 & 0 & 0 & 0\\0 & \frac{r}{2} \left(3 r - 1\right) & 3 r & 1 & 0 & 0 & 0 & 0\\0 & \frac{r}{2} \left(6 r^{2} - 5 r + 1\right) & r \left(9 r - 2\right) & 6 r & 1 & 0 & 0 & 0\\0 & \frac{r}{6} \left(45 r^{3} - 65 r^{2} + 30 r - 4\right) & \frac{5 r}{2} \left(12 r^{2} - 7 r + 1\right) & 5 r \left(6 r - 1\right) & 10 r & 1 & 0 & 0\\0 & \frac{r}{24} \left(540 r^{4} - 1155 r^{3} + 890 r^{2} - 273 r + 22\right) & \frac{r}{2} \left(225 r^{3} - 235 r^{2} + 80 r - 8\right) & \frac{15 r}{2} \left(20 r^{2} - 9 r + 1\right) & 5 r \left(15 r - 2\right) & 15 r & 1 & 0\\0 & \frac{r}{24} \left(1890 r^{5} - 5481 r^{4} + 6125 r^{3} - 3129 r^{2} + 637 r - 18\right) & \frac{7 r}{24} \left(1620 r^{4} - 2565 r^{3} + 1490 r^{2} - 351 r + 22\right) & \frac{7 r}{2} \left(225 r^{3} - 185 r^{2} + 50 r - 4\right) & \frac{35 r}{2} \left(30 r^{2} - 11 r + 1\right) & \frac{35 r}{2} \left(9 r - 1\right) & 21 r & 1\end{matrix}\right]
\left[\begin{matrix} 0 \\ 1 \\ r \\ \frac{r}{2} \left(3 r - 1\right) \\ \frac{r}{2} \left(6 r^{2} - 5 r + 1\right) \\ \frac{r}{6} \left(45 r^{3} - 65 r^{2} + 30 r - 4\right) \\ \frac{r}{24} \left(540 r^{4} - 1155 r^{3} + 890 r^{2} - 273 r + 22\right) \\ \frac{r}{24} \left(1890 r^{5} - 5481 r^{4} + 6125 r^{3} - 3129 r^{2} + 637 r - 18\right) \end{matrix}\right]
\end{displaymath}
$\sqrt{\mathcal{S}_{m}}$ is
\begin{displaymath}
\left[\begin{matrix}1 & 0 & 0 & 0 & 0 & 0 & 0 & 0\\0 & 1 & 0 & 0 & 0 & 0 & 0 & 0\\0 & \frac{1}{2} & 1 & 0 & 0 & 0 & 0 & 0\\0 & \frac{1}{8} & \frac{3}{2} & 1 & 0 & 0 & 0 & 0\\0 & 0 & \frac{5}{4} & 3 & 1 & 0 & 0 & 0\\0 & \frac{1}{32} & \frac{5}{8} & 5 & 5 & 1 & 0 & 0\\0 & - \frac{7}{128} & \frac{11}{32} & \frac{45}{8} & \frac{55}{4} & \frac{15}{2} & 1 & 0\\0 & \frac{1}{128} & - \frac{7}{128} & \frac{161}{32} & \frac{105}{4} & \frac{245}{8} & \frac{21}{2} & 1\end{matrix}\right]
\end{displaymath}
finally, $e^{\mathcal{S}_{m}}$ is
\begin{displaymath}
e \left[\begin{matrix}1 & 0 & 0 & 0 & 0 & 0 & 0 & 0\\0 & 1 & 0 & 0 & 0 & 0 & 0 & 0\\0 & 1 & 1 & 0 & 0 & 0 & 0 & 0\\0 & \frac{5}{2} & 3 & 1 & 0 & 0 & 0 & 0\\0 & \frac{21}{2} & 16 & 6 & 1 & 0 & 0 & 0\\0 & \frac{203}{3} & \frac{235}{2} & 55 & 10 & 1 & 0 & 0\\0 & \frac{14681}{24} & 1176 & \frac{1245}{2} & 140 & 15 & 1 & 0\\0 & \frac{22018}{3} & \frac{367745}{24} & 8911 & \frac{4515}{2} & \frac{595}{2} & 21 & 1\end{matrix}\right]
\end{displaymath}
evaluating corresponding $g$ polynomials on $\mathcal{S}_{m}$, as required.

\subsection{Fibonacci numbers}

In order to have a comparison with a matrix $\mathcal{F}$ having two eigenvalues $\lambda_{1}\neq \lambda_{2}$,
defined as
\begin{displaymath}
\mathcal{F} = \left[\begin{matrix}1 & 1\\1 & 0\end{matrix}\right],
\quad  \lambda_{1} =  \frac{1}{2}- \frac{\sqrt{5}}{2} ,
\quad \lambda_{2} = \frac{1}{2} + \frac{\sqrt{5}}{2}
\end{displaymath}
respectively, we need to use an augmented base composed of polynomials
\begin{displaymath}
 \Phi_{ 1, 1 }{\left (z \right )} = \frac{z}{\lambda_{1} - \lambda_{2}} - \frac{\lambda_{2}}{\lambda_{1} - \lambda_{2}}, \quad  \Phi_{ 2, 1 }{\left (z \right )} = - \frac{z}{\lambda_{1} - \lambda_{2}} + \frac{\lambda_{1}}{\lambda_{1} - \lambda_{2}}
\end{displaymath}
to define polynomial $g$ representing $f(z)=z^{r}$:
\begin{displaymath}
g{\left (z \right )} = z \left(\frac{\lambda_{1}^{r}}{\lambda_{1} - \lambda_{2}} - \frac{\lambda_{2}^{r}}{\lambda_{1} - \lambda_{2}}\right) + \frac{\lambda_{1} \lambda_{2}^{r}}{\lambda_{1} - \lambda_{2}} - \frac{\lambda_{1}^{r} \lambda_{2}}{\lambda_{1} - \lambda_{2}}
\end{displaymath}
therefore $\mathcal{F}^{r}$ is $g(\mathcal{F})$, in matrix notation
\begin{displaymath}
\left[\begin{matrix}\frac{1}{\lambda_{1} - \lambda_{2}} \left(\lambda_{1} \lambda_{2}^{r} - \lambda_{1}^{r} \lambda_{2} + \lambda_{1}^{r} - \lambda_{2}^{r}\right) & \frac{\lambda_{1}^{r} - \lambda_{2}^{r}}{\lambda_{1} - \lambda_{2}}\\\frac{\lambda_{1}^{r} - \lambda_{2}^{r}}{\lambda_{1} - \lambda_{2}} & \frac{\lambda_{1} \lambda_{2}^{r} - \lambda_{1}^{r} \lambda_{2}}{\lambda_{1} - \lambda_{2}}\end{matrix}\right]
\end{displaymath}

For the sake of clarity, choose $r=8$ to obtain
    \begin{displaymath}
    \mathcal{F}^{8} = \left[\begin{matrix}f_{9} & f_{8}\\f_{8} & f_{7}\end{matrix}\right] = \left[\begin{matrix}34 & 21\\21 & 13\end{matrix}\right]
    \end{displaymath}
where $f_{n}$ is the $n$-th Fibonacci number within sequence A000045 in the OEIS.

\section*{Facts about Riordan matrices}

Defining structure of Riordan matrices entails many useful properties, such as:
\begin{itemize} 
\item matrix-vector product $\mathcal{R}\cdot \textbf{a}$ is defined as 
    $ d(t)A(h(t))$, where $A$ is a formal power series
    with coefficients belonging to vector $\textbf{a}$ and $\mathcal{R}=(d, h)$;
\item matrix-matrix product $\mathcal{R}\cdot \mathcal{W}$ is defined as 
    $(d(t)g(h(t)), l(h(t)))$, where $\mathcal{R}=(d, h)$ and $\mathcal{W}=(g, l)$;
\item identity matrix $I$ is $(1, t)$;
\item the inverse $\mathcal{R}^{-1}$ of $\mathcal{R}=(d, h)$ is defined as 
    $\left(\frac{1}{d(\bar{h}(t))}, \bar{h}\right)$, where $\bar{h}$ is the \emph{compositional inverse} 
    of $h$, namely $\bar{h}(h(t))=t$;
\item the set of these matrices is a \emph{group} with matrix-matrix product as operation.    
\end{itemize}

