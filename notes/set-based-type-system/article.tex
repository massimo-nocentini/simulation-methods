
\documentclass[a4paper,12pt]{article}

\usepackage{inputenc}
\usepackage{euler}
\usepackage[T1]{fontenc}
\usepackage{fontspec}
%\usepackage{libertine}
\usepackage{lipsum}
\usepackage{fancyvrb}
\usepackage{url}
%\usepackage[english]{babel}
\usepackage{amsmath}
\usepackage{amsthm}
\usepackage{amssymb}
\usepackage{acronym}
\usepackage{hyperref}
\usepackage{tabu}
\usepackage{rotating}
\usepackage{mathdots}
\usepackage{minted}
\usepackage{units}
\usepackage{float}
\usepackage{bbold}
%\usepackage{cprotect}
\usepackage{stmaryrd}
\usepackage[sort&compress,square,comma,authoryear]{natbib}

\fvset{fontsize=\small}
\setmonofont[Scale=0.8]{Menlo}
\usemintedstyle{xcode}
\hypersetup{colorlinks}

\newtheorem{theorem}{Theorem}
\newtheorem{lemma}[theorem]{Lemma}
\newtheorem{proposition}[theorem]{Proposition}
\newtheorem{corollary}[theorem]{Corollary}
\newtheorem{definition}[theorem]{Definition}
\newtheorem{remark}[theorem]{Remark}
\newtheorem{example}[theorem]{Example}

\newcommand{\semantic}[1]{\left\llbracket\,#1\,\right\rrbracket}
\newcommand{\tydef}[3]{#1\indent\triangleq\indent#2\indent\textit{(#3)}}
\newcommand{\typarens}[1]{\left(\,#1\,\right)}
\newcommand{\tyunit}[0]{\textbf{unit}}
\newcommand{\tytruth}[0]{\textbf{truth}}
\newcommand{\tyfalsehood}[0]{\textbf{falsehood}}
\newcommand{\tyboolean}[0]{\textbf{boolean}}
\newcommand{\tydu}[2]{#1\,\cup\,#2}
\newcommand{\tycp}[2]{#1\,\times\,#2}
\newcommand{\tymaybe}[1]{\textbf{maybe}\,#1}
\newcommand{\rec}[2]{\textbf{rec}\,#1.\,#2}
\newcommand{\tyvar}[1]{\textbf{tyvar}\,#1}
\newcommand{\lst}[1]{\textbf{lst}\,#1}
\newcommand{\derangements}[1]{\textbf{derangements}\,#1}

\iffalse
\newenvironment{pyeval}[1]%
{\cprotEnv\begin{minted}[baselinestretch=0.8]{python}#1\end{minted}\begin{displaymath}\it}%
{\end{displaymath}}

\newenvironment{pycode}%
{\cprotEnv\begin{minted}[baselinestretch=0.8]{python}\it}%
{\end{minted}}

\newenvironment{code}[1]
{\VerbatimEnvironment
\begin{figure}
\centering
\caption{simple caption}
\begin{minted}[frame=lines]{\it}}
{\end{minted}\end{figure}}

\newenvironment{code1}[1]
{\VerbatimEnvironment
\begin{center}
\begin{minted}[frame=lines,framesep=2mm,linenos]{python}#1\end{minted}
\end{center}
\begin{displaymath}}
{\end{displaymath}}

\newenvironment{qsi}[1]%
{\begin{Verbatim}#1 wrote\end{Verbatim}\it}%
{\end{quote}}
\fi


\author{Massimo Nocentini}

\title{A naive type system based on sets and its generating functions}

\begin{document}

\maketitle

\begin{abstract}
This note collects a naive type system based on sets, with an automatic
computation of their generating functions.
\end{abstract}


\section{Basic definitions}


Let $A\in\mathbb{C}^{m\times m}$ be a matrix and denote with $\sigma(A)$ the
spectrum of $A$, namely the set of $A$'s eigenvalues
\begin{displaymath}
\sigma(A) = \left\lbrace \lambda_{i}:
A\boldsymbol{v}_{i}=\lambda_{i}\boldsymbol{v}_{i} \, \text{for} \, 1\leq i\leq \nu\wedge
\boldsymbol{v}_{i}\in\mathbb{C}^{m}\right\rbrace 
\end{displaymath}
with multiplicities $\lbrace m_{i}: 1\leq i\leq \nu\rbrace$ respectively, such
that $ \sum_{i=1}^{\nu}{m_{i}}=m$.

Let $p(\lambda)=\det{\left(A-\lambda I\right)}=\prod_{i=1}^{\nu}{(\lambda -
\lambda_{i})^{m_{i}}}$ be the \textit{characteristic polynomial} of
matrix $A$ of degree $m$. For any polynomial $h$ of degree greater than $m$ it
is possible to perform the following division:
\begin{displaymath}
h(\lambda) = q(\lambda)p(\lambda)+r(\lambda), \quad \deg{r(\lambda) < m}
\end{displaymath}
for some polynomial $q$, so $h(A) = r(A)$ because $p(A)=0$; eventually, both
    polynomials (of \textit{different degrees}) $h$ and $r$ yield the same
    matrix when applied to $A$.
Since $\left. \frac{\partial^{(j)}{p}}{\partial{\lambda}}
\right|_{\lambda=\lambda_{i}}=0$ then
\begin{displaymath}
\left.\frac{\partial^{(j)}\left(h(\lambda) - r(\lambda)\right)}{\partial\lambda}\right|_{\lambda=\lambda_{i}} =
\left.\frac{\partial^{(j)}\left(q(\lambda)p(\lambda)\right)}{\partial\lambda}\right|_{\lambda=\lambda_{i}} = 0
\end{displaymath}
therefore, polynomials $h$ and $r$ satisfy 
\begin{displaymath}
h(A)=r(A) \leftrightarrow
\left.\frac{\partial^{(j)}h}{\partial\lambda}=\frac{\partial^{(j)}r}{\partial\lambda}\right|_{\lambda=\lambda_{i}}
\end{displaymath}
for $i\in \lbrace 1, \ldots, \nu \rbrace$ for $j \in \lbrace 0, \ldots, m_{i}-1 \rbrace$;
in words, \textit{polynomials} $h$ \textit{and} $r$ \textit{take the same values on} $\sigma(A)$.

Let $f:\mathbb{C}\rightarrow \mathbb{C}$ be a function on the formal variable
$z$. We say that $f$ \textit{is defined on $\sigma(A)$} if exists
\begin{displaymath}
    \left. \frac{\partial^{(j)}{f}}{\partial{z}} \right|_{z=\lambda_{i}},\quad
    i\in \lbrace 1, \ldots, \nu \rbrace \wedge j \in \lbrace 0, \ldots, m_{i}-1
    \rbrace
\end{displaymath}

Given a function $f$ defined on $\sigma(A)$, we aim to define a polynomial $g$
such that $f$ and $g$ take the same values on $\sigma(A)$, to compute $f(A)$
via $g(A)$; moreover, $g$ is an \emph{Hermite interpolating polynomial} which
can be written using the base of \textit{generalized Lagrange polynomials}
$\left\lbrace \Phi_{i,j}\in\prod_{m-1} \right\rbrace$. Formally:
\begin{displaymath}
g(z) = \sum_{i=1}^{\nu}{\sum_{j=1}^{m_{i}}{ \left.
\frac{\partial^{(j-1)}{f}}{\partial{z}} \right|_{z=\lambda_{i}}\Phi_{i,j}(z) }}
\end{displaymath}
where each polynomial $\Phi_{i,j}$ is implicitly defined as the solution of the
system with $m$ constraints:
\begin{displaymath}
    \left. \frac{\partial^{(r-1)}{\Phi_{i,j}}}{\partial{z}} \right|_{z=\lambda_{l}} = \delta_{i,l}\delta_{j,r}
\end{displaymath}
for $l\in \lbrace 1, \ldots, \nu \rbrace$ for $r \in \lbrace 1, \ldots, m_{l}
\rbrace$, where $\delta$ is the Kroneker delta, defined as $\delta_{i,j}=1
\leftrightarrow i=j$.  Observe that if $m_{l}=1$  then $\left\lbrace \Phi_{i,j}\in\prod_{m-1} \right\rbrace$
reduces to the usual Lagrange base, for all $l\in\lbrace 1, \ldots, \nu\rbrace$;
for the sake of clarity, if $\nu=4$ then the polynomials $\left\lbrace \Phi_{i,1}\in\prod_{3}:i\in\lbrace1,\ldots,4\rbrace \right\rbrace$
defined as follows
\begin{displaymath}
\begin{split}
\Phi_{ 1, 1 }{\left (z \right )} &= \frac{\left(z - \lambda_{2}\right)
\left(z - \lambda_{3}\right) \left(z - \lambda_{4}\right)}{\left(\lambda_{1} -
\lambda_{2}\right) \left(\lambda_{1} - \lambda_{3}\right) \left(\lambda_{1} -
\lambda_{4}\right)} \\ 
\Phi_{ 2, 1 }{\left (z \right )} &= - \frac{\left(z -
\lambda_{1}\right) \left(z - \lambda_{3}\right) \left(z -
\lambda_{4}\right)}{\left(\lambda_{1} - \lambda_{2}\right) \left(\lambda_{2} -
\lambda_{3}\right) \left(\lambda_{2} - \lambda_{4}\right)} \\ 
\Phi_{ 3, 1 }{\left (z \right )} &= \frac{\left(z - \lambda_{1}\right) \left(z -
\lambda_{2}\right) \left(z - \lambda_{4}\right)}{\left(\lambda_{1} -
\lambda_{3}\right) \left(\lambda_{2} - \lambda_{3}\right) \left(\lambda_{3} -
\lambda_{4}\right)} \\ 
\Phi_{ 4, 1 }{\left (z \right )} &= - \frac{\left(z -
\lambda_{1}\right) \left(z - \lambda_{2}\right) \left(z -
\lambda_{3}\right)}{\left(\lambda_{1} - \lambda_{4}\right) \left(\lambda_{2} -
\lambda_{4}\right) \left(\lambda_{3} - \lambda_{4}\right)}\\
\end{split}
\end{displaymath}
are a Lagrange base respect to eigenvalues $\lambda_{1}, \lambda_{2},\lambda_{3},\lambda_{4}$.
On the other hand, if $\nu=1$ then there is one eigenvalue $\lambda_{1}$ only, with algebraic multiplicity $m_{1}=m$; for the sake of clarity again,
for $m=8$ the polynomials $\left\lbrace \Phi_{1,j}\in\prod_{7}:j\in\lbrace1,\ldots,8\rbrace \right\rbrace$ defined as follows
\begin{equation}
\begin{array}{c}
\Phi_{ 1, 1 }{\left (z \right )} = 1 \\ 
\Phi_{ 1, 2 }{\left (z \right )} = z - \lambda_{1} \\ 
\Phi_{ 1, 3 }{\left (z \right )} = \frac{z^{2}}{2} - z \lambda_{1} + \frac{\lambda_{1}^{2}}{2}\\ 
\Phi_{ 1, 4 }{\left (z \right )} = \frac{z^{3}}{6} - \frac{z^{2} \lambda_{1}}{2} + \frac{z \lambda_{1}^{2}}{2} - \frac{\lambda_{1}^{3}}{6} \\ 
\Phi_{ 1, 5 }{\left (z \right )} = \frac{z^{4}}{24} - \frac{z^{3} \lambda_{1}}{6} + \frac{z^{2} \lambda_{1}^{2}}{4} - \frac{z \lambda_{1}^{3}}{6} + \frac{\lambda_{1}^{4}}{24} \\ 
\Phi_{ 1, 6 }{\left (z \right )} = \frac{z^{5}}{120} - \frac{z^{4} \lambda_{1}}{24} + \frac{z^{3} \lambda_{1}^{2}}{12} - \frac{z^{2} \lambda_{1}^{3}}{12} + \frac{z \lambda_{1}^{4}}{24} - \frac{\lambda_{1}^{5}}{120} \\
\Phi_{ 1, 7 }{\left (z \right )} = \frac{z^{6}}{720} - \frac{z^{5} \lambda_{1}}{120} + \frac{z^{4} \lambda_{1}^{2}}{48} - \frac{z^{3} \lambda_{1}^{3}}{36} + \frac{z^{2} \lambda_{1}^{4}}{48} - \frac{z \lambda_{1}^{5}}{120} + \frac{\lambda_{1}^{6}}{720} \\ 
\Phi_{ 1, 8 }{\left (z \right )} = \frac{z^{7}}{5040} - \frac{z^{6} \lambda_{1}}{720} + \frac{z^{5} \lambda_{1}^{2}}{240} - \frac{z^{4} \lambda_{1}^{3}}{144} + \frac{z^{3} \lambda_{1}^{4}}{144} - \frac{z^{2} \lambda_{1}^{5}}{240} + \frac{z \lambda_{1}^{6}}{720} - \frac{\lambda_{1}^{7}}{5040}\\
\end{array}
\label{eq:generalized-Lagrange-base}
\end{equation}
are a \textit{generalized} Lagrange base respect to eigenvalue $\lambda_{1}$.
Evaluating polynomial $g$ on matrix $A$ yield:
\begin{displaymath}
g(A) = \sum_{i=1}^{\nu}{\sum_{j=1}^{m_{i}}{ \left.  \frac{\partial^{(j-1)}{f}}{\partial{z}} \right|_{z=\lambda_{i}}\Phi_{i,j}(A) }}
     = \sum_{i=1}^{\nu}{\sum_{j=1}^{m_{i}}{ \left.  \frac{\partial^{(j-1)}{f}}{\partial{z}} \right|_{z=\lambda_{i}}Z_{ij}^{[A]} }}
\end{displaymath}
where matrix $Z_{ij}^{[A]}=\Phi_{i,j}(A)$, for $i\in \lbrace 1, \ldots, \nu \rbrace$
and $j \in \lbrace 0, \ldots, m_{i}-1 \rbrace$, is a \textit{component matrix}
of $A$. Moreover, we can rewrite it according to facts reported in the appendix:
\begin{displaymath}
g(A) = \sum_{i=1}^{\nu}{\sum_{j=1}^{m_{i}}{ \left.  \frac{\partial^{(j-1)}{f}}{\partial{z}} \right|_{z=\lambda_{i}}\frac{1}{(j-1)!}{Z_{i1}^{[A]}(A-\lambda_{i}I)^{j-1}} }}
\end{displaymath}


We are now ready to apply this framework to matrices in the Riordan group.

\section{Riordan matrices}


In the rest we study functions of \emph{Riordan matrices}. A Riordan matrix
$\mathcal{R}=(d_{n,k}\in\mathbb{C})_{n,k\in\mathbb{N}}$ is an infinite, lower
triangular matrix defined by two generating functions $d$ and $h$, with $h(0)=0
\wedge h^{\prime}(0)\neq0$, where $R_{k}(t) = d(t)h(t)^{k} =
\sum_{n\in\mathbb{N}}{d_{n,k}t^{n}}$ is the formal power series enumerating
coefficients that lie on column $k$; in parallel, $\mathcal{R}$ is denoted by $(d,
h)$. 

From here on, we assume that $\mathcal{R}_{m}\in\mathbb{C}^{m\times m}$ is a
\emph{finite Riordan matrix}, hence $\sigma(\mathcal{R}_{m})= \lbrace
\lambda_{1} \rbrace$, therefore $\nu=1$ and eigenvalue $\lambda_{1}$ gets
multiplicity $m_{1}=m$; finally, function $h$ satisfies $h'(0)=1$ usually, so
$\lambda_{1}=1$ and $\mathcal{R}_{m}$ is called \textit{proper}.
We proceed by relaxing the condition $\lambda_{1}=1$ in order to use
$\lambda_{1}$ as a pure symbol: this allows us to spot structures with respect
to $\lambda_{1}$ and, eventually, perform the substitution to specialize
symbolic terms.

Polynomials in \autoref{eq:generalized-Lagrange-base} can be rewritten in matrix form:
\begin{displaymath}
E_{\lambda_{1}}\boldsymbol{z} = \left[\begin{matrix}1 & 0 & 0 & 0 & 0 & 0 & 0 & 0\\- \lambda_{1} & 1 & 0 & 0 & 0 & 0 & 0 & 0\\\frac{\lambda_{1}^{2}}{2} & - \lambda_{1} & 1 & 0 & 0 & 0 & 0 & 0\\- \frac{\lambda_{1}^{3}}{6} & \frac{\lambda_{1}^{2}}{2} & - \lambda_{1} & 1 & 0 & 0 & 0 & 0\\\frac{\lambda_{1}^{4}}{24} & - \frac{\lambda_{1}^{3}}{6} & \frac{\lambda_{1}^{2}}{2} & - \lambda_{1} & 1 & 0 & 0 & 0\\- \frac{\lambda_{1}^{5}}{120} & \frac{\lambda_{1}^{4}}{24} & - \frac{\lambda_{1}^{3}}{6} & \frac{\lambda_{1}^{2}}{2} & - \lambda_{1} & 1 & 0 & 0\\\frac{\lambda_{1}^{6}}{720} & - \frac{\lambda_{1}^{5}}{120} & \frac{\lambda_{1}^{4}}{24} & - \frac{\lambda_{1}^{3}}{6} & \frac{\lambda_{1}^{2}}{2} & - \lambda_{1} & 1 & 0\\- \frac{\lambda_{1}^{7}}{5040} & \frac{\lambda_{1}^{6}}{720} & - \frac{\lambda_{1}^{5}}{120} & \frac{\lambda_{1}^{4}}{24} & - \frac{\lambda_{1}^{3}}{6} & \frac{\lambda_{1}^{2}}{2} & - \lambda_{1} & 1\end{matrix}\right] \left[\begin{matrix}1\\z\\\frac{z^{2}}{2!}\\\frac{z^{3}}{3!}\\\frac{z^{4}}{4!}\\\frac{z^{5}}{5!}\\\frac{z^{6}}{6!}\\\frac{z^{7}}{7!}\end{matrix}\right] = \left[\begin{matrix}\phi_{ 1, 1 }{\left (z \right )}\\\phi_{ 1, 2 }{\left (z \right )}\\\phi_{ 1, 3 }{\left (z \right )}\\\phi_{ 1, 4 }{\left (z \right )}\\\phi_{ 1, 5 }{\left (z \right )}\\\phi_{ 1, 6 }{\left (z \right )}\\\phi_{ 1, 7 }{\left (z \right )}\\\phi_{ 1, 8 }{\left (z \right )}\end{matrix}\right]
\end{displaymath}
where the generic matrix element $d_{nk}$ at row $n$ and column $k$ is defined by 
\begin{displaymath}
    d_{n,k} = \frac{\left(-\lambda_{1}\right)^{n-k}}{\left(n-k\right)!}, \quad k\leq n
\end{displaymath}
therefore, the generalized Lagrange base is composed of polynomials
\begin{displaymath}
  \Phi_{1,j}(z) = \sum_{k=0}^{j-1}{\frac{(-\lambda_{1})^{j-1-k}}{(j-1-k)!}\frac{z^{k}}{k!}}
\end{displaymath}

For the sake of clarity, restoring the condition $\lambda_{1}=1$ we have the following polynomials
\begin{displaymath}
\begin{array}{c}
 \Phi_{ 1, 1 }{\left (z \right )} = 1\\
 \Phi_{ 1, 2 }{\left (z \right )} = z - 1\\
 \Phi_{ 1, 3 }{\left (z \right )} = \frac{z^{2}}{2} - z + \frac{1}{2}\\
 \Phi_{ 1, 4 }{\left (z \right )} = \frac{z^{3}}{6} - \frac{z^{2}}{2} + \frac{z}{2} - \frac{1}{6}\\
 \Phi_{ 1, 5 }{\left (z \right )} = \frac{z^{4}}{24} - \frac{z^{3}}{6} + \frac{z^{2}}{4} - \frac{z}{6} + \frac{1}{24}\\
 \Phi_{ 1, 6 }{\left (z \right )} = \frac{z^{5}}{120} - \frac{z^{4}}{24} + \frac{z^{3}}{12} - \frac{z^{2}}{12} + \frac{z}{24} - \frac{1}{120}\\
 \Phi_{ 1, 7 }{\left (z \right )} = \frac{z^{6}}{720} - \frac{z^{5}}{120} + \frac{z^{4}}{48} - \frac{z^{3}}{36} + \frac{z^{2}}{48} - \frac{z}{120} + \frac{1}{720}\\
 \Phi_{ 1, 8 }{\left (z \right )} = \frac{z^{7}}{5040} - \frac{z^{6}}{720} + \frac{z^{5}}{240} - \frac{z^{4}}{144} + \frac{z^{3}}{144} - \frac{z^{2}}{240} + \frac{z}{720} - \frac{1}{5040}\\
\end{array}
\end{displaymath}
for \textit{any} proper Riordan array $\mathcal{R}_{8}$.


\subsection{A Riordan array characterization of Hermite interpolating polynomials}


Observe that matrix $E_{\lambda_{1}}$ is the \textit{ordinary} Riordan array
$\left(e^{-\lambda_{1}t}, t\right)$, a minor of dimension $m$, precisely.
Since the multiplying vector is a chunk of the series expansion of $e^{zt}$, by
the fundamental theorem of Riordan arrays
\begin{displaymath}
\left(e^{-\lambda_{1}t}, t\right)e^{zt} = e^{-\lambda_{1}t} \left(e^{zt}\circ_{t} t \right) =  e^{-\lambda_{1}t} e^{zt} = e^{t(z - \lambda_{1})} = \Phi_{1}(t, z)
\end{displaymath}
where $\Phi_{1}(t, z)$ expands with respect to $t$ as follows
\begin{displaymath}
\begin{split}
\Phi_{1}(t, z) &= 1 \\
               &+ t \left(z - \lambda_{1}\right) \\
               &+ t^{2} \left(\frac{z^{2}}{2} - z \lambda_{1} + \frac{\lambda_{1}^{2}}{2}\right) \\
               &+ t^{3} \left(\frac{z^{3}}{6} - \frac{z^{2} \lambda_{1}}{2} + \frac{z \lambda_{1}^{2}}{2} - \frac{\lambda_{1}^{3}}{6}\right) \\
               &+ t^{4} \left(\frac{z^{4}}{24} - \frac{z^{3} \lambda_{1}}{6} + \frac{z^{2} \lambda_{1}^{2}}{4} - \frac{z \lambda_{1}^{3}}{6} + \frac{\lambda_{1}^{4}}{24}\right) \\
               &+ t^{5} \left(\frac{z^{5}}{120} - \frac{z^{4} \lambda_{1}}{24} + \frac{z^{3} \lambda_{1}^{2}}{12} - \frac{z^{2} \lambda_{1}^{3}}{12} + \frac{z \lambda_{1}^{4}}{24} - \frac{\lambda_{1}^{5}}{120}\right)\\
               &+ t^{6} \left(\frac{z^{6}}{720} - \frac{z^{5} \lambda_{1}}{120} + \frac{z^{4} \lambda_{1}^{2}}{48} - \frac{z^{3} \lambda_{1}^{3}}{36} + \frac{z^{2} \lambda_{1}^{4}}{48} - \frac{z \lambda_{1}^{5}}{120} + \frac{\lambda_{1}^{6}}{720}\right)\\
               &+ t^{7} \left(\frac{z^{7}}{5040} - \frac{z^{6} \lambda_{1}}{720} + \frac{z^{5} \lambda_{1}^{2}}{240} - \frac{z^{4} \lambda_{1}^{3}}{144} + \frac{z^{3} \lambda_{1}^{4}}{144} - \frac{z^{2} \lambda_{1}^{5}}{240} + \frac{z \lambda_{1}^{6}}{720} - \frac{\lambda_{1}^{7}}{5040}\right) \\
               &+ \mathcal{O}\left(t^{8}\right)\\
\end{split}
\end{displaymath}
so $\Phi_{1, j}(z) = [t^{j}]\Phi_{1}(t, z)$ for $j \in  \lbrace 1,\ldots, m_{1} \rbrace$.
In this setting, the formal definition of an interpolating polynomial $g$ specializes to
\begin{displaymath}
g(z) = \sum_{j=1}^{m_{1}}{ \left.  \frac{\partial^{(j-1)}{f}}{\partial{z}} \right|_{z=\lambda_{1}}\Phi_{1,j}(z) } = \boldsymbol{1}^{T}\,D_{f}\, E_{\lambda_{1}} \,\boldsymbol{z}
\end{displaymath}
where $D_{f}$ is a matrix with derivatives of function $f$ on the main diagonal
\begin{displaymath}
D_{f} = 
\left[
    \begin{array}{ccccc}
        \left.f(t)\right|_{t=\lambda_{1}} & \\
                                          &  \ddots \\
                                          &         & \left.\frac{\partial^{i}}{\partial t^{i}}f(t)\right|_{t=\lambda_{1}} \\
                                          &         &                                                                       & \ddots \\
                                          &         &                                                                       &        &  \left.\frac{\partial^{m_{1}-1}}{\partial t^{m_{1}-1}}f(t)\right|_{t=\lambda_{1}} \\
    \end{array}
\right]
\end{displaymath}
for $i\in  \lbrace 1,\ldots,m_{1}-2 \rbrace$.


\subsection{A component matrices characterization of Hermite interpolating polynomials}


Polynomials $\Phi_{ 1, 1 }$ and $\Phi_{ 1, 2 }$ have interesting properties
when evaluated at a Riordan array $\mathcal{R}_{m}$, formally
\begin{displaymath}
 Z_{1,1}^{[\mathcal{R}_{m}]} = \Phi_{ 1, 1 }{\left (\mathcal{R}_{m} \right )} = I \quad\quad\quad
 Z_{1,2}^{[\mathcal{R}_{m}]} = \Phi_{ 1, 2 }{\left (\mathcal{R}_{m} \right )} = \mathcal{R}_{m} - I
\end{displaymath}
According to these facts, consider again the definition of polynomial $g$ that takes the same values of a function $f$:
\begin{displaymath}
\begin{split}
    g(\mathcal{R}_{m}) &= \sum_{j=1}^{m}{ \left. \frac{\partial^{(j-1)}{f}}{\partial{z}} \right|_{z=\lambda_{1}}\frac{1}{(j-1)!}{Z_{1,1}^{[\mathcal{R}_{m}]} (\mathcal{R}_{m}-\lambda_{1}I)^{j-1}} }\\
                       &= \sum_{j=1}^{m}{ \left. \frac{\partial^{(j-1)}{f}}{\partial{z}} \right|_{z=1}\frac{1}{(j-1)!}{(\mathcal{R}_{m}-I)^{j-1}} }\\
                       &= \sum_{j=1}^{m}{ \left. \frac{\partial^{(j-1)}{f}}{\partial{z}} \right|_{z=1}\frac{1}{(j-1)!}{\left(Z_{1,2}^{[\mathcal{R}_{m}]}\right)^{j-1}} }\\
                       &= g_{e}\left(Z_{1,2}^{[\mathcal{R}_{m}]}\right)\\
\end{split}
\end{displaymath}
where polynomial $g_{e}$ is a kind of exponential generating function
\begin{displaymath}
    g_{e}\left(z\right) = \sum_{j=1}^{m}{ \left. \frac{\partial^{(j-1)}{f}}{\partial{z}} \right|_{z=1}\frac{z^{j-1}}{(j-1)!}}
\end{displaymath}
here the difficult part lies on the nature of matrix $\mathcal{R}_{m}-I$
because \textit{subtraction} is not a well defined operation in the Riordan
group; therefore, how can it be defined?  Moreover, is it a Riordan matrix in
all cases?


\section{Functions}

\subsection{$f(z)=z^{r}$}


The general form of $j$th derivative of function $f$ is 
$$\frac{\partial^{(j)}{f}(z)}{\partial{z}} = (r)_{(j)} z^{r-j}$$ 
where $(r)_{(j)} = r(r-1)\cdots(r-j+1)$ is the falling factorial, therefore
\begin{displaymath}
\begin{split}
  g(z) &= \sum_{j=1}^{m}{ \left. \frac{\partial^{(j-1)}{f}}{\partial{z}} \right|_{z=\lambda_{1}}\Phi_{1,j}(z)} \\
       &= \sum_{j=1}^{m}{ \left. (r)_{(j-1)} z^{r-j+1} \right|_{z=1}\Phi_{1,j}(z)} \\
       %&= \sum_{j=1}^{m}{ \left. z^{r-j+1} \right|_{z=\lambda_{1}}(r)_{(j-1)} \Phi_{1,j}(z)} \\
       &= \sum_{j=1}^{m}{\sum_{k=0}^{j-1}{\frac{(r)_{(j-1)}}{(j-1)_{(j-1)}}\frac{(j-1)!(-1)^{j-1-k}}{(j-1-k)!}\frac{z^{k}}{k!}}} \\
\end{split}
\end{displaymath}
yielding equation
\begin{equation}
  g(z) = \sum_{j=1}^{m}{\sum_{k=0}^{j-1}{(-1)^{j-1}{{r}\choose{j-1}}{{j-1}\choose{k}}(-z)^{k}}} 
\end{equation}
We swap summations holding the same argument explained in previous section:
\begin{displaymath}
  g(z) = \sum_{k=0}^{m-1}{\left(\sum_{j=k+1}^{m}{(-1)^{j-1}{{r}\choose{j-1}}{{j-1}\choose{k}}}\right)(-z)^{k}}
\end{displaymath}
yielding equation
\begin{equation}
  g(z) = \sum_{k=0}^{m-1}{\left(\sum_{j=k}^{m-1}{(-1)^{j}{{r}\choose{j}}{{j}\choose{k}}}\right)(-z)^{k}}
\end{equation}
The above expression holds without conditions on both $r$ and $m$; however it is possible 
to find a closed expression for the inner sum:
\begin{eqnarray}
  g(z) &= \sum_{k=0}^{m-1}{\left(\left(-1\right)^{m}\frac{ k - m }{r-k}{\binom{m}{k}} {\binom{r}{m}}\right)(-z)^{k}}\\
       &= \sum_{k=0}^{m-1}{\left(\frac{ k - m }{r-k}{\binom{m}{k}} {\binom{m-r-1}{m}}\right)(-z)^{k}}
\end{eqnarray}
Last expression is defined unless $r=k$, namely $\mathcal{R}_{m}^{r}=g(\mathcal{R}_{m})$ for any proper, 
finite Riordan matrix $\mathcal{R}_{m}$, where $m\leq r$. On the other hand, for $m>r$ we have 
$g(z)=z^{r}$, which agrees with intuition because computing $g(\mathcal{R}_{m})$, for $m$ much bigger than $r$, 
requires computing combination of powers $\mathcal{R}_{m}^{j}$ for $j\in \lbrace 0,\ldots,r \rbrace$ at least,
which outperforms computing $\mathcal{R}_{m}^{r}$ as a whole.

For the sake of clarity, let $m=8$ to define polynomial $g$:
\begin{displaymath}
\begin{split}
g{\left (z \right )} &= z^{7} {\binom{r}{7}} \\
&+ z^{6} \left({\binom{r}{6}} - 7 {\binom{r}{7}}\right) \\
&+ z^{5} \left({\binom{r}{5}} - 6 {\binom{r}{6}} + 21 {\binom{r}{7}}\right) \\
&+ z^{4} \left({\binom{r}{4}} - 5 {\binom{r}{5}} + 15 {\binom{r}{6}} - 35 {\binom{r}{7}}\right) \\
&+ z^{3} \left({\binom{r}{3}} - 4 {\binom{r}{4}} + 10 {\binom{r}{5}} - 20 {\binom{r}{6}} + 35 {\binom{r}{7}}\right) \\
&+ z^{2} \left({\binom{r}{2}} - 3 {\binom{r}{3}} + 6 {\binom{r}{4}} - 10 {\binom{r}{5}} + 15 {\binom{r}{6}} - 21 {\binom{r}{7}}\right) \\
&+ z \left({\binom{r}{1}} - 2 {\binom{r}{2}} + 3 {\binom{r}{3}} - 4 {\binom{r}{4}} + 5 {\binom{r}{5}} - 6 {\binom{r}{6}} + 7 {\binom{r}{7}}\right) \\
&- {\binom{r}{1}} + {\binom{r}{2}} - {\binom{r}{3}} + {\binom{r}{4}} - {\binom{r}{5}} + {\binom{r}{6}} - {\binom{r}{7}} + 1 \\
\end{split}
\end{displaymath}

\iffalse
expansion of inner binomial coefficients yields
\begin{displaymath}
\begin{split}
g{\left (z \right )} &= - \frac{r^{7}}{5040} + \frac{r^{6}}{180} - \frac{23 r^{5}}{360} + \frac{7 r^{4}}{18} - \frac{967 r^{3}}{720} + \frac{469 r^{2}}{180} - \frac{363 r}{140} \\
&+ z^{7} \left(\frac{r^{7}}{5040} - \frac{r^{6}}{240} + \frac{5 r^{5}}{144} - \frac{7 r^{4}}{48} + \frac{29 r^{3}}{90} - \frac{7 r^{2}}{20} + \frac{r}{7}\right) \\
&+ z^{6} \left(- \frac{r^{7}}{720} + \frac{11 r^{6}}{360} - \frac{19 r^{5}}{72} + \frac{41 r^{4}}{36} - \frac{1849 r^{3}}{720} + \frac{1019 r^{2}}{360} - \frac{7 r}{6}\right) \\
&+ z^{5} \left(\frac{r^{7}}{240} - \frac{23 r^{6}}{240} + \frac{69 r^{5}}{80} - \frac{185 r^{4}}{48} + \frac{134 r^{3}}{15} - \frac{201 r^{2}}{20} + \frac{21 r}{5}\right) \\
&+ z^{4} \left(- \frac{r^{7}}{144} + \frac{r^{6}}{6} - \frac{113 r^{5}}{72} + \frac{22 r^{4}}{3} - \frac{2545 r^{3}}{144} + \frac{41 r^{2}}{2} - \frac{35 r}{4}\right) \\
&+ z^{3} \left(\frac{r^{7}}{144} - \frac{25 r^{6}}{144} + \frac{247 r^{5}}{144} - \frac{1219 r^{4}}{144} + \frac{389 r^{3}}{18} - \frac{949 r^{2}}{36} + \frac{35 r}{3}\right) \\
&+ z^{2} \left(- \frac{r^{7}}{240} + \frac{13 r^{6}}{120} - \frac{9 r^{5}}{8} + \frac{71 r^{4}}{12} - \frac{3929 r^{3}}{240} + \frac{879 r^{2}}{40} - \frac{21 r}{2}\right) \\
&+ z \left(\frac{r^{7}}{720} - \frac{3 r^{6}}{80} + \frac{59 r^{5}}{144} - \frac{37 r^{4}}{16} + \frac{319 r^{3}}{45} - \frac{223 r^{2}}{20} + 7 r\right) + 1
\end{split}
\end{displaymath}
\fi

Moreover, using last two equations and requiring $r \geq 8$, we have:
\begin{displaymath}
\begin{split}
g{\left (z \right )} &= 8 {\binom{r}{8}} \left( \frac{z^{7}}{r - 7} - \frac{7 z^{6}}{r - 6} + \frac{21 z^{5}}{r - 5}\right. \left. - \frac{35 z^{4}}{r - 4} + \frac{35 z^{3}}{r - 3} - \frac{21 z^{2}}{r - 2} + \frac{7 z}{r - 1} - \frac{1}{r} \right) \\
&= 8 {\binom{7-r}{8}} \left( \frac{z^{7}}{r - 7} - \frac{7 z^{6}}{r - 6} + \frac{21 z^{5}}{r - 5}\right. \left. - \frac{35 z^{4}}{r - 4} + \frac{35 z^{3}}{r - 3} - \frac{21 z^{2}}{r - 2} + \frac{7 z}{r - 1} - \frac{1}{r} \right) \\
\end{split}
\end{displaymath}
respectively.

Using Riordan array characterization we have 
\begin{displaymath}
D_{{z}^{r}}E_{\lambda_{1}} = \left[\begin{matrix}\frac{{\left(r\right)}_{0} \lambda_{1}^{r}}{{\left(0\right)}_{0}} & 0 & 0 & 0 & 0 & 0 & 0 & 0\\- \frac{{\left(r\right)}_{1} \lambda_{1}^{r}}{{\left(1\right)}_{1}} & \frac{{\left(r\right)}_{1}}{{\left(0\right)}_{0}} \lambda_{1}^{r - 1} & 0 & 0 & 0 & 0 & 0 & 0\\\frac{{\left(r\right)}_{2} \lambda_{1}^{r}}{{\left(2\right)}_{2}} & - \frac{{\left(r\right)}_{2}}{{\left(1\right)}_{1}} \lambda_{1}^{r - 1} & \frac{{\left(r\right)}_{2}}{{\left(0\right)}_{0}} \lambda_{1}^{r - 2} & 0 & 0 & 0 & 0 & 0\\- \frac{{\left(r\right)}_{3} \lambda_{1}^{r}}{{\left(3\right)}_{3}} & \frac{{\left(r\right)}_{3}}{{\left(2\right)}_{2}} \lambda_{1}^{r - 1} & - \frac{{\left(r\right)}_{3}}{{\left(1\right)}_{1}} \lambda_{1}^{r - 2} & \frac{{\left(r\right)}_{3}}{{\left(0\right)}_{0}} \lambda_{1}^{r - 3} & 0 & 0 & 0 & 0\\\frac{{\left(r\right)}_{4} \lambda_{1}^{r}}{{\left(4\right)}_{4}} & - \frac{{\left(r\right)}_{4}}{{\left(3\right)}_{3}} \lambda_{1}^{r - 1} & \frac{{\left(r\right)}_{4}}{{\left(2\right)}_{2}} \lambda_{1}^{r - 2} & - \frac{{\left(r\right)}_{4}}{{\left(1\right)}_{1}} \lambda_{1}^{r - 3} & \frac{{\left(r\right)}_{4}}{{\left(0\right)}_{0}} \lambda_{1}^{r - 4} & 0 & 0 & 0\\- \frac{{\left(r\right)}_{5} \lambda_{1}^{r}}{{\left(5\right)}_{5}} & \frac{{\left(r\right)}_{5}}{{\left(4\right)}_{4}} \lambda_{1}^{r - 1} & - \frac{{\left(r\right)}_{5}}{{\left(3\right)}_{3}} \lambda_{1}^{r - 2} & \frac{{\left(r\right)}_{5}}{{\left(2\right)}_{2}} \lambda_{1}^{r - 3} & - \frac{{\left(r\right)}_{5}}{{\left(1\right)}_{1}} \lambda_{1}^{r - 4} & \frac{{\left(r\right)}_{5}}{{\left(0\right)}_{0}} \lambda_{1}^{r - 5} & 0 & 0\\\frac{{\left(r\right)}_{6} \lambda_{1}^{r}}{{\left(6\right)}_{6}} & - \frac{{\left(r\right)}_{6}}{{\left(5\right)}_{5}} \lambda_{1}^{r - 1} & \frac{{\left(r\right)}_{6}}{{\left(4\right)}_{4}} \lambda_{1}^{r - 2} & - \frac{{\left(r\right)}_{6}}{{\left(3\right)}_{3}} \lambda_{1}^{r - 3} & \frac{{\left(r\right)}_{6}}{{\left(2\right)}_{2}} \lambda_{1}^{r - 4} & - \frac{{\left(r\right)}_{6}}{{\left(1\right)}_{1}} \lambda_{1}^{r - 5} & \frac{{\left(r\right)}_{6}}{{\left(0\right)}_{0}} \lambda_{1}^{r - 6} & 0\\- \frac{{\left(r\right)}_{7} \lambda_{1}^{r}}{{\left(7\right)}_{7}} & \frac{{\left(r\right)}_{7}}{{\left(6\right)}_{6}} \lambda_{1}^{r - 1} & - \frac{{\left(r\right)}_{7}}{{\left(5\right)}_{5}} \lambda_{1}^{r - 2} & \frac{{\left(r\right)}_{7}}{{\left(4\right)}_{4}} \lambda_{1}^{r - 3} & - \frac{{\left(r\right)}_{7}}{{\left(3\right)}_{3}} \lambda_{1}^{r - 4} & \frac{{\left(r\right)}_{7}}{{\left(2\right)}_{2}} \lambda_{1}^{r - 5} & - \frac{{\left(r\right)}_{7}}{{\left(1\right)}_{1}} \lambda_{1}^{r - 6} & \frac{{\left(r\right)}_{7}}{{\left(0\right)}_{0}} \lambda_{1}^{r - 7}\end{matrix}\right]
\end{displaymath}
generated by the production matrix
\begin{displaymath}
\left[\begin{matrix}- r & \frac{r}{\lambda_{1}} & 0 & 0 & 0 & 0 & 0\\- \frac{\lambda_{1}}{2} \left(r + 1\right) & 1 & \frac{1}{\lambda_{1}} \left(r - 1\right) & 0 & 0 & 0 & 0\\- \frac{\lambda_{1}^{2}}{6} \left(r + 1\right) & 0 & 1 & \frac{1}{\lambda_{1}} \left(r - 2\right) & 0 & 0 & 0\\- \frac{\lambda_{1}^{3}}{24} \left(r + 1\right) & 0 & 0 & 1 & \frac{1}{\lambda_{1}} \left(r - 3\right) & 0 & 0\\- \frac{\lambda_{1}^{4}}{120} \left(r + 1\right) & 0 & 0 & 0 & 1 & \frac{1}{\lambda_{1}} \left(r - 4\right) & 0\\- \frac{\lambda_{1}^{5}}{720} \left(r + 1\right) & 0 & 0 & 0 & 0 & 1 & \frac{1}{\lambda_{1}} \left(r - 5\right)\\- \frac{\lambda_{1}^{6}}{5040} \left(r + 1\right) & 0 & 0 & 0 & 0 & 0 & 1\end{matrix}\right]
\end{displaymath}
so the matrix satisfies the recurrence relation 
\begin{displaymath}
\begin{split}
d_{0,0}&=\lambda_{1}^{r}\\
d_{n,0}&=-\left(r d_{n-1, 0} + (r+1)\sum_{k=1}^{n-1}{d_{n-1, k}\frac{\lambda_{1}^{k}}{(k+1)!}}\right), \quad n>0 \\
d_{n,k}&=\frac{r+1-k}{\lambda_{1}}d_{n-1, k-1} + d_{n-1,k}, \quad n,k > 0\\
\end{split}
\end{displaymath}
finally,
\begin{displaymath}
D_{{z}^{r}}E_{\lambda_{1}}\boldsymbol{z} = \left[\begin{matrix}\frac{{\left(r\right)}_{0} \lambda_{1}^{r}}{{\left(0\right)}_{0}}\\\frac{{\left(r\right)}_{1}}{{\left(1\right)}_{1}} \left(z - \lambda_{1}\right) \lambda_{1}^{r - 1}\\\frac{{\left(r\right)}_{2}}{{\left(2\right)}_{2}} \left(z - \lambda_{1}\right)^{2} \lambda_{1}^{r - 2}\\\frac{{\left(r\right)}_{3}}{{\left(3\right)}_{3}} \left(z - \lambda_{1}\right)^{3} \lambda_{1}^{r - 3}\\\frac{{\left(r\right)}_{4}}{{\left(4\right)}_{4}} \left(z - \lambda_{1}\right)^{4} \lambda_{1}^{r - 4}\\\frac{{\left(r\right)}_{5}}{{\left(5\right)}_{5}} \left(z - \lambda_{1}\right)^{5} \lambda_{1}^{r - 5}\\\frac{{\left(r\right)}_{6}}{{\left(6\right)}_{6}} \left(z - \lambda_{1}\right)^{6} \lambda_{1}^{r - 6}\\\frac{{\left(r\right)}_{7}}{{\left(7\right)}_{7}} \left(z - \lambda_{1}\right)^{7} \lambda_{1}^{r - 7}\end{matrix}\right]
 = \left[\begin{matrix}{\binom{r}{0}} \lambda_{1}^{r}\\\left(z - \lambda_{1}\right) {\binom{r}{1}} \lambda_{1}^{r - 1}\\\left(z - \lambda_{1}\right)^{2} {\binom{r}{2}} \lambda_{1}^{r - 2}\\\left(z - \lambda_{1}\right)^{3} {\binom{r}{3}} \lambda_{1}^{r - 3}\\\left(z - \lambda_{1}\right)^{4} {\binom{r}{4}} \lambda_{1}^{r - 4}\\\left(z - \lambda_{1}\right)^{5} {\binom{r}{5}} \lambda_{1}^{r - 5}\\\left(z - \lambda_{1}\right)^{6} {\binom{r}{6}} \lambda_{1}^{r - 6}\\\left(z - \lambda_{1}\right)^{7} {\binom{r}{7}} \lambda_{1}^{r - 7}\end{matrix}\right]
\end{displaymath}
therefore restoring $\lambda_{1}=1$ yields the polynomial
\begin{displaymath}
\begin{split}
g{\left (z \right )} = \boldsymbol{1}^{T}D_{{z}^{r}}E_{\lambda_{1}}\boldsymbol{z} &= \left(z - 1\right)^{7} {\binom{r}{7}} + \left(z - 1\right)^{6} {\binom{r}{6}} + \left(z - 1\right)^{5} {\binom{r}{5}} + \left(z - 1\right)^{4} {\binom{r}{4}} \\
                    &+ \left(z - 1\right)^{3} {\binom{r}{3}} + \left(z - 1\right)^{2} {\binom{r}{2}} + \left(z - 1\right) {\binom{r}{1}} + {\binom{r}{0}}
\end{split}
\end{displaymath}
hence we generalize for $m\in\mathbb{N}$:
\begin{displaymath}
\mathcal{R}_{m}^{r} = g{\left (\mathcal{R}_{m} \right )} = \sum_{j=0}^{m-1}{\binom{r}{j}}{\left(Z_{1,2}^{[\mathcal{R}_{m}]}\right)^{j} }
\end{displaymath}


In the following two section we study two specializations $r=-1$ and
$r=\frac{1}{2}$, corresponding to the \textit{inverse} and \textit{square root}
functions, respectively.

\subsubsection{$f(z)=\frac{1}{z}$}


The general form of $j$th derivative of function $f$ is 
$$\frac{\partial^{(j)}{f}(z)}{\partial{z}} = \frac{(-1)^{j}j!}{z^{j+1}}$$ 
therefore
\begin{displaymath}
\begin{split}
  g(z) &= \sum_{j=1}^{m}{ \left. \frac{\partial^{(j-1)}{f}}{\partial{z}} \right|_{z=\lambda_{1}}\Phi_{1,j}(z)} \\
       &= \sum_{j=1}^{m}{ \left. \frac{(-1)^{j-1}(j-1)!}{z^{j}} \right|_{z=1}\Phi_{1,j}(z)} \\
       %&= \sum_{j=1}^{m}{ \left. \frac{1}{z^{j}} \right|_{z=1}(-1)^{j-1}(j-1)!\Phi_{1,j}(z)} \\
       &= \sum_{j=1}^{m}{\sum_{k=0}^{j-1}{(-1)^{j-1}(j-1)!\frac{(-1)^{j-1-k}}{(j-1-k)!}\frac{z^{k}}{k!}}} \\
\end{split}
\end{displaymath}
yielding equation
\begin{equation}
  g(z) = \sum_{j=1}^{m}{\sum_{k=0}^{j-1}{{{j-1}\choose{k}}(-z)^{k}}} 
\end{equation}

To swap summations we reason on the relation $k=j-1$, which holds by the inner summation upper limit:
$k\in \lbrace 0,\ldots,m-1 \rbrace$ because last value for $j$ is $m$. Finally, to cover the same set 
of pairs $(j, k)$, when $k=0$ then $j\in \lbrace 1,\ldots,m \rbrace$ and when $k=m-1$ then 
$j\in \lbrace m \rbrace$, therefore $j\in \lbrace k+1, \ldots, m \rbrace$ is required, in the general case.
Binomial manipulations yield:
\begin{displaymath}
  g(z) = \sum_{k=0}^{m-1}{\left(\sum_{j=k+1}^{m}{{{j-1}\choose{k}}}\right)(-z)^{k}}
       = \sum_{k=0}^{m-1}{\left(\sum_{j=k}^{m-1}{{{j}\choose{k}}}\right)(-z)^{k}}
\end{displaymath}
the inner sum admits a closed expression, yielding equation
\begin{equation}
  g(z) = \sum_{k=0}^{m-1}{{{m}\choose{k+1}}(-z)^{k}} \\
\end{equation}
namely, $\mathcal{R}_{m}^{-1}=g(\mathcal{R}_{m})$ for any proper, finite Riordan matrix $\mathcal{R}_{m}$.

\iffalse
Polynomial $g$ can also be written in closed form, assuming convergence condition $|z|\leq 1$:
\begin{displaymath}
g(z) = \frac{1- \left(1- z \right)^{m} }{z}
\end{displaymath}
\fi

For the sake of clarity, polynomial $g$ when $m=8$ and relaxing the condition $\lambda=1$, is defined according to 
\begin{displaymath}
\begin{split}
g{\left (z \right )} &= - \frac{z^{7}}{\lambda^{8}} \\
&+ z^{6} \left(\frac{1}{\lambda^{7}} + \frac{7}{\lambda^{8}}\right) \\
&+ z^{5} \left(- \frac{1}{\lambda^{6}} - \frac{6}{\lambda^{7}} - \frac{21}{\lambda^{8}}\right) \\
&+ z^{4} \left(\frac{1}{\lambda^{5}} + \frac{5}{\lambda^{6}} + \frac{15}{\lambda^{7}} + \frac{35}{\lambda^{8}}\right) \\
&+ z^{3} \left(- \frac{1}{\lambda^{4}} - \frac{4}{\lambda^{5}} - \frac{10}{\lambda^{6}} - \frac{20}{\lambda^{7}} - \frac{35}{\lambda^{8}}\right) \\
&+ z^{2} \left(\frac{1}{\lambda^{3}} + \frac{3}{\lambda^{4}} + \frac{6}{\lambda^{5}} + \frac{10}{\lambda^{6}} + \frac{15}{\lambda^{7}} + \frac{21}{\lambda^{8}}\right) \\
&+ z \left(- \frac{1}{\lambda^{2}} - \frac{2}{\lambda^{3}} - \frac{3}{\lambda^{4}} - \frac{4}{\lambda^{5}} - \frac{5}{\lambda^{6}} - \frac{6}{\lambda^{7}} - \frac{7}{\lambda^{8}}\right) \\
&+ \frac{1}{\lambda} + \frac{1}{\lambda^{2}} + \frac{1}{\lambda^{3}} + \frac{1}{\lambda^{4}} + \frac{1}{\lambda^{5}} + \frac{1}{\lambda^{6}} + \frac{1}{\lambda^{7}} + \frac{1}{\lambda^{8}}
\end{split}
\end{displaymath}
restoring $\lambda=1$ yields \[g{\left (z \right )} = - z^{7} + 8 z^{6} - 28 z^{5} + 56 z^{4} - 70 z^{3} + 56 z^{2} - 28 z + 8\]

A computation observation concerns the evaluation of $g(\mathcal{R}_{m})$,
which should be carried out as
\begin{displaymath}
g(z) = z \left(z \left(z \left(z \left(z \left(z \left(- z + 8\right) - 28\right) + 56\right) - 70\right) + 56\right) - 28\right) + 8
\end{displaymath}
namely, according to the Horner rule for polynomials, interpreting each
coefficient $c\in\mathbb{R}$ as $cI$ where $I\in\mathbb{C}^{m\times m}$ is the
identity matrix. Such approach requires $m-2$ matrix products and $m-1$
additions; finally, we implicitly use this scheme in all subsequent evaluation
of a polynomial $g$ to a matrix $A$.

Using Riordan array characterization we have 
\begin{displaymath}
D_{\frac{1}{z}}E_{\lambda_{1}} = \left[\begin{matrix}\frac{1}{\lambda_{1}} & 0 & 0 & 0 & 0 & 0 & 0 & 0\\\frac{1}{\lambda_{1}} & - \frac{1}{\lambda_{1}^{2}} & 0 & 0 & 0 & 0 & 0 & 0\\\frac{1}{\lambda_{1}} & - \frac{2}{\lambda_{1}^{2}} & \frac{2}{\lambda_{1}^{3}} & 0 & 0 & 0 & 0 & 0\\\frac{1}{\lambda_{1}} & - \frac{3}{\lambda_{1}^{2}} & \frac{6}{\lambda_{1}^{3}} & - \frac{6}{\lambda_{1}^{4}} & 0 & 0 & 0 & 0\\\frac{1}{\lambda_{1}} & - \frac{4}{\lambda_{1}^{2}} & \frac{12}{\lambda_{1}^{3}} & - \frac{24}{\lambda_{1}^{4}} & \frac{24}{\lambda_{1}^{5}} & 0 & 0 & 0\\\frac{1}{\lambda_{1}} & - \frac{5}{\lambda_{1}^{2}} & \frac{20}{\lambda_{1}^{3}} & - \frac{60}{\lambda_{1}^{4}} & \frac{120}{\lambda_{1}^{5}} & - \frac{120}{\lambda_{1}^{6}} & 0 & 0\\\frac{1}{\lambda_{1}} & - \frac{6}{\lambda_{1}^{2}} & \frac{30}{\lambda_{1}^{3}} & - \frac{120}{\lambda_{1}^{4}} & \frac{360}{\lambda_{1}^{5}} & - \frac{720}{\lambda_{1}^{6}} & \frac{720}{\lambda_{1}^{7}} & 0\\\frac{1}{\lambda_{1}} & - \frac{7}{\lambda_{1}^{2}} & \frac{42}{\lambda_{1}^{3}} & - \frac{210}{\lambda_{1}^{4}} & \frac{840}{\lambda_{1}^{5}} & - \frac{2520}{\lambda_{1}^{6}} & \frac{5040}{\lambda_{1}^{7}} & - \frac{5040}{\lambda_{1}^{8}}\end{matrix}\right]
\end{displaymath}
generated by the production matrix
\begin{displaymath}
\left[\begin{matrix}1 & - \frac{1}{\lambda_{1}} & 0 & 0 & 0 & 0 & 0\\0 & 1 & - \frac{2}{\lambda_{1}} & 0 & 0 & 0 & 0\\0 & 0 & 1 & - \frac{3}{\lambda_{1}} & 0 & 0 & 0\\0 & 0 & 0 & 1 & - \frac{4}{\lambda_{1}} & 0 & 0\\0 & 0 & 0 & 0 & 1 & - \frac{5}{\lambda_{1}} & 0\\0 & 0 & 0 & 0 & 0 & 1 & - \frac{6}{\lambda_{1}}\\0 & 0 & 0 & 0 & 0 & 0 & 1\end{matrix}\right]
\end{displaymath}
so the matrix satisfies the recurrence relation 
\begin{displaymath}
\begin{split}
d_{0,0}&=\frac{1}{\lambda_{1}}\\
d_{n,0}&=d_{n-1, 0}, \quad n>0 \\
d_{n,k}&=-\frac{k}{\lambda_{1}}d_{n-1, k-1} + d_{n-1,k}, \quad n,k > 0\\
\end{split}
\end{displaymath}
finally,
\begin{displaymath}
D_{\frac{1}{z}}E_{\lambda_{1}}\boldsymbol{z} = \left[\begin{matrix}\frac{1}{\lambda_{1}}\\- \frac{1}{\lambda_{1}^{2}} \left(z - \lambda_{1}\right)\\\frac{1}{\lambda_{1}^{3}} \left(z - \lambda_{1}\right)^{2}\\- \frac{1}{\lambda_{1}^{4}} \left(z - \lambda_{1}\right)^{3}\\\frac{1}{\lambda_{1}^{5}} \left(z - \lambda_{1}\right)^{4}\\- \frac{1}{\lambda_{1}^{6}} \left(z - \lambda_{1}\right)^{5}\\\frac{1}{\lambda_{1}^{7}} \left(z - \lambda_{1}\right)^{6}\\- \frac{1}{\lambda_{1}^{8}} \left(z - \lambda_{1}\right)^{7}\end{matrix}\right]
\end{displaymath}
therefore restoring $\lambda_{1}=1$ yields the polynomial
\[g{\left (z \right )} = \boldsymbol{1}^{T}D_{\frac{1}{z}}E_{\lambda_{1}}\boldsymbol{z} = - \left(z - 1\right)^{7} + \left(z - 1\right)^{6} - \left(z - 1\right)^{5} + \left(z - 1\right)^{4} - \left(z - 1\right)^{3} + \left(z - 1\right)^{2} - (z-1) + 1\]
hence we generalize for $m\in\mathbb{N}$:
\begin{displaymath}
\mathcal{R}_{m}^{-1} = g{\left (\mathcal{R}_{m} \right )} = \sum_{j=0}^{m-1}{\left(-Z_{1,2}^{[\mathcal{R}_{m}]}\right)^{j}} = \frac{1}{1+Z_{1,2}^{[\mathcal{R}_{m}]}}
\end{displaymath}
moreover, the limit for $m \rightarrow \infty$ yields $ g{\left (\mathcal{R} \right )} = \frac{1}{\mathcal{R}} $ for the whole Riordan array $\mathcal{R}$.



\subsubsection{$f(z)=\sqrt{z}$}


The general form of $j$th derivative of function $f$ is 
$$\frac{\partial^{(j)}{f}(z)}{\partial{z}} =\frac{(-1)^{j-1}}{2}\frac{(j-1)!}{4^{j-1}}{{2(j-1)}\choose{j-1}}\frac{1}{z^{\frac{2(j-1)+1}{2}}} $$ 
for $j \geq 1$, therefore
\begin{displaymath}
\begin{split}
  g(z) &= \sum_{j=1}^{m}{ \left. \frac{\partial^{(j-1)}{f}}{\partial{z}} \right|_{z=\lambda_{1}}\Phi_{1,j}(z)} \\
       &= \sum_{j=0}^{m-1}{ \left. \frac{\partial^{(j)}{f}}{\partial{z}} \right|_{z=1}\Phi_{1,j+1}(z)} \\
       &= f(1)\Phi_{1,1}(z) + \sum_{j=1}^{m-1}{ \left. \frac{\partial^{(j)}{f}}{\partial{z}} \right|_{z=1}\Phi_{1,j+1}(z)} \\
\end{split}
\end{displaymath}
Observing that $\Phi_{1,1}(z)=1$ and
\begin{displaymath}
  \Phi_{1,j+1}(z) = \sum_{k=0}^{j}{\frac{(-1)^{j-k}}{(j-k)!}\frac{z^{k}}{k!}}
\end{displaymath}
it allows us to state the equation
\begin{equation}
  g(z) = 1 - \frac{1}{2} \sum_{j=1}^{m-1}{\sum_{k=0}^{j}{\frac{1}{j 4^{j-1}} {{2(j-1)}\choose{j-1}}{{j}\choose{k}} (-z)^{k}}}
\end{equation}
To swap summations, for covering the same set of pairs $(j, k)$ it is required that 
$j\in \lbrace 1, \ldots, m-1 \rbrace$ for both $k=0$ and $k=1$, therefore we split
the outer sum over $k$ extracting the very first term due to $k=0$:
\begin{equation}
  g(z) = 1 - \frac{1}{2}\sum_{j=1}^{m-1}{\frac{1}{j 4^{j-1}} {{2(j-1)}\choose{j-1}}} 
       - \frac{1}{2} \sum_{k=1}^{m-1}{\left(\sum_{j=k}^{m-1}{\frac{1}{j 4^{j-1}} {{2(j-1)}\choose{j-1}}{{j}\choose{k}}}\right)(-z)^{k}} 
\end{equation}
compactly, using Kroneker delta in the inner sum's starting:
\begin{equation}
  g(z) = 1 - \frac{1}{2} \sum_{k=0}^{m-1}{\left(\sum_{j=k+\delta_{k,0}}^{m-1}{\frac{1}{j 4^{j-1}} {{2(j-1)}\choose{j-1}}{{j}\choose{k}}}\right)(-z)^{k}} \\
\end{equation}

For the sake of clarity, here is $g$ polynomial where $m=8$ and relaxing $\lambda=1$ hypothesis:
\begin{displaymath}
\begin{split}
g{\left (z \right )} &= \frac{33 z^{7}}{2048 \lambda^{\frac{13}{2}}} \\
&+ z^{6} \left(- \frac{21}{1024 \lambda^{\frac{11}{2}}} - \frac{231}{2048 \lambda^{\frac{13}{2}}}\right) \\
&+ z^{5} \left(\frac{7}{256 \lambda^{\frac{9}{2}}} + \frac{63}{512 \lambda^{\frac{11}{2}}} + \frac{693}{2048 \lambda^{\frac{13}{2}}}\right) \\
&+ z^{4} \left(- \frac{5}{128 \lambda^{\frac{7}{2}}} - \frac{35}{256 \lambda^{\frac{9}{2}}} - \frac{315}{1024 \lambda^{\frac{11}{2}}} - \frac{1155}{2048 \lambda^{\frac{13}{2}}}\right) \\
&+ z^{3} \left(\frac{1}{16 \lambda^{\frac{5}{2}}} + \frac{5}{32 \lambda^{\frac{7}{2}}} + \frac{35}{128 \lambda^{\frac{9}{2}}} + \frac{105}{256 \lambda^{\frac{11}{2}}} + \frac{1155}{2048 \lambda^{\frac{13}{2}}}\right) \\
&+ z^{2} \left(- \frac{1}{8 \lambda^{\frac{3}{2}}} - \frac{3}{16 \lambda^{\frac{5}{2}}} - \frac{15}{64 \lambda^{\frac{7}{2}}} - \frac{35}{128 \lambda^{\frac{9}{2}}} - \frac{315}{1024 \lambda^{\frac{11}{2}}} - \frac{693}{2048 \lambda^{\frac{13}{2}}}\right) \\
&+ z \left(\frac{1}{2 \sqrt{\lambda}} + \frac{1}{4 \lambda^{\frac{3}{2}}} + \frac{3}{16 \lambda^{\frac{5}{2}}} + \frac{5}{32 \lambda^{\frac{7}{2}}} + \frac{35}{256 \lambda^{\frac{9}{2}}} \right. + \left. \frac{63}{512 \lambda^{\frac{11}{2}}} + \frac{231}{2048 \lambda^{\frac{13}{2}}}\right) \\
&+ \sqrt{\lambda} - \frac{1}{2 \sqrt{\lambda}} - \frac{1}{8 \lambda^{\frac{3}{2}}} - \frac{1}{16 \lambda^{\frac{5}{2}}} - \frac{5}{128 \lambda^{\frac{7}{2}}} - \frac{7}{256 \lambda^{\frac{9}{2}}} - \frac{21}{1024 \lambda^{\frac{11}{2}}} - \frac{33}{2048 \lambda^{\frac{13}{2}}}
\end{split}
\end{displaymath}
restoring $\lambda=1$:
\begin{displaymath}
g{\left (z \right )} = \frac{33 z^{7}}{2048} - \frac{273 z^{6}}{2048} + \frac{1001 z^{5}}{2048} - \frac{2145 z^{4}}{2048} + \frac{3003 z^{3}}{2048} - \frac{3003 z^{2}}{2048} + \frac{3003 z}{2048} + \frac{429}{2048}
\end{displaymath}

Using Riordan array characterization we have 
\begin{displaymath}
D_{\sqrt{z}}E_{\lambda_{1}} = \left[\begin{matrix}\sqrt{\lambda_{1}} & 0 & 0 & 0 & 0 & 0 & 0 & 0\\- \frac{\sqrt{\lambda_{1}}}{2} & \frac{1}{2 \sqrt{\lambda_{1}}} & 0 & 0 & 0 & 0 & 0 & 0\\- \frac{\sqrt{\lambda_{1}}}{8} & \frac{1}{4 \sqrt{\lambda_{1}}} & - \frac{1}{4 \lambda_{1}^{\frac{3}{2}}} & 0 & 0 & 0 & 0 & 0\\- \frac{\sqrt{\lambda_{1}}}{16} & \frac{3}{16 \sqrt{\lambda_{1}}} & - \frac{3}{8 \lambda_{1}^{\frac{3}{2}}} & \frac{3}{8 \lambda_{1}^{\frac{5}{2}}} & 0 & 0 & 0 & 0\\- \frac{5 \sqrt{\lambda_{1}}}{128} & \frac{5}{32 \sqrt{\lambda_{1}}} & - \frac{15}{32 \lambda_{1}^{\frac{3}{2}}} & \frac{15}{16 \lambda_{1}^{\frac{5}{2}}} & - \frac{15}{16 \lambda_{1}^{\frac{7}{2}}} & 0 & 0 & 0\\- \frac{7 \sqrt{\lambda_{1}}}{256} & \frac{35}{256 \sqrt{\lambda_{1}}} & - \frac{35}{64 \lambda_{1}^{\frac{3}{2}}} & \frac{105}{64 \lambda_{1}^{\frac{5}{2}}} & - \frac{105}{32 \lambda_{1}^{\frac{7}{2}}} & \frac{105}{32 \lambda_{1}^{\frac{9}{2}}} & 0 & 0\\- \frac{21 \sqrt{\lambda_{1}}}{1024} & \frac{63}{512 \sqrt{\lambda_{1}}} & - \frac{315}{512 \lambda_{1}^{\frac{3}{2}}} & \frac{315}{128 \lambda_{1}^{\frac{5}{2}}} & - \frac{945}{128 \lambda_{1}^{\frac{7}{2}}} & \frac{945}{64 \lambda_{1}^{\frac{9}{2}}} & - \frac{945}{64 \lambda_{1}^{\frac{11}{2}}} & 0\\- \frac{33 \sqrt{\lambda_{1}}}{2048} & \frac{231}{2048 \sqrt{\lambda_{1}}} & - \frac{693}{1024 \lambda_{1}^{\frac{3}{2}}} & \frac{3465}{1024 \lambda_{1}^{\frac{5}{2}}} & - \frac{3465}{256 \lambda_{1}^{\frac{7}{2}}} & \frac{10395}{256 \lambda_{1}^{\frac{9}{2}}} & - \frac{10395}{128 \lambda_{1}^{\frac{11}{2}}} & \frac{10395}{128 \lambda_{1}^{\frac{13}{2}}}\end{matrix}\right]
\end{displaymath}
generated by the production matrix
\begin{displaymath}
\left[\begin{matrix}- \frac{1}{2} & \frac{1}{2 \lambda_{1}} & 0 & 0 & 0 & 0 & 0\\- \frac{3 \lambda_{1}}{4} & 1 & - \frac{1}{2 \lambda_{1}} & 0 & 0 & 0 & 0\\- \frac{\lambda_{1}^{2}}{4} & 0 & 1 & - \frac{3}{2 \lambda_{1}} & 0 & 0 & 0\\- \frac{\lambda_{1}^{3}}{16} & 0 & 0 & 1 & - \frac{5}{2 \lambda_{1}} & 0 & 0\\- \frac{\lambda_{1}^{4}}{80} & 0 & 0 & 0 & 1 & - \frac{7}{2 \lambda_{1}} & 0\\- \frac{\lambda_{1}^{5}}{480} & 0 & 0 & 0 & 0 & 1 & - \frac{9}{2 \lambda_{1}}\\- \frac{\lambda_{1}^{6}}{3360} & 0 & 0 & 0 & 0 & 0 & 1\end{matrix}\right]
\end{displaymath}
so the matrix satisfies the recurrence relation
\begin{displaymath}
\begin{split}
d_{0,0}&=\sqrt{\lambda_{1}}\\
d_{n,0}&=-\left(\frac{1}{2} d_{n-1, 0} + \frac{3}{2}\sum_{k=1}^{n-1}{d_{n-1, k}\frac{\lambda_{1}^{k}}{(k+1)!}}\right), \quad n>0 \\
d_{n,k}&=\frac{3-2k}{2\lambda_{1}}d_{n-1, k-1} + d_{n-1,k}, \quad n,k > 0\\
\end{split}
\end{displaymath}
finally,
\begin{displaymath}
D_{\sqrt{z}}E_{\lambda_{1}}\boldsymbol{z} = \left[\begin{matrix}\sqrt{\lambda_{1}}\\\frac{z - \lambda_{1}}{2 \sqrt{\lambda_{1}}}\\- \frac{\left(z - \lambda_{1}\right)^{2}}{8 \lambda_{1}^{\frac{3}{2}}}\\\frac{\left(z - \lambda_{1}\right)^{3}}{16 \lambda_{1}^{\frac{5}{2}}}\\- \frac{5 \left(z - \lambda_{1}\right)^{4}}{128 \lambda_{1}^{\frac{7}{2}}}\\\frac{7 \left(z - \lambda_{1}\right)^{5}}{256 \lambda_{1}^{\frac{9}{2}}}\\- \frac{21 \left(z - \lambda_{1}\right)^{6}}{1024 \lambda_{1}^{\frac{11}{2}}}\\\frac{33 \left(z - \lambda_{1}\right)^{7}}{2048 \lambda_{1}^{\frac{13}{2}}}\end{matrix}\right]
\end{displaymath}
therefore restoring $\lambda_{1}=1$ yields the polynomial
\begin{displaymath}
\begin{split}
g{\left (z \right )} = \boldsymbol{1}^{T}D_{\sqrt{z}}E_{\lambda_{1}}\boldsymbol{z} &= \frac{33}{2048} \left(z - 1\right)^{7} - \frac{21}{1024} \left(z - 1\right)^{6} + \frac{7}{256} \left(z - 1\right)^{5} - \frac{5}{128} \left(z - 1\right)^{4} \\
    &+ \frac{1}{16} \left(z - 1\right)^{3} - \frac{1}{8} \left(z - 1\right)^{2} + \frac{1}{2}(z-1) + 1
\end{split}
\end{displaymath}
hence we generalize for $m\in\mathbb{N}$:
\begin{displaymath}
\sqrt{\mathcal{R}_{m}} = g{\left (\mathcal{R}_{m} \right )} = \sum_{j=0}^{m-1}{\left(\left[t^{j}\right]\sqrt{1+t}\right){\left(Z_{1,2}^{[\mathcal{R}_{m}]}\right)^{j} }} = \sqrt{1+Z_{1,2}^{[\mathcal{R}_{m}]}}
\end{displaymath}
moreover, the limit for $m \rightarrow \infty$ yields $ g{\left (\mathcal{R} \right )} = \sqrt{\mathcal{R}} $ for the whole Riordan array $\mathcal{R}$.



\subsection{$f(z)=e^{\alpha z}$}


The general form of $j$th derivative of function $f$ is 
$$\frac{\partial^{(j)}{f}(z)}{\partial{z}} = \alpha^{j} e^{\alpha z}$$ 
therefore
\begin{displaymath}
\begin{split}
  g(z) &= \sum_{j=1}^{m}{ \left. \frac{\partial^{(j-1)}{f}}{\partial{z}} \right|_{z=\lambda_{1}}\Phi_{1,j}(z)} \\
       &= \sum_{j=1}^{m}{ \left. \alpha^{j-1} e^{\alpha z} \right|_{z=1}\Phi_{1,j}(z)} \\
       %&= \sum_{j=1}^{m}{ \left. e^{\alpha z} \right|_{z=\lambda_{1}}\alpha^{j-1} \Phi_{1,j}(z)} \\
       &= e^{\alpha}\sum_{j=1}^{m}{\sum_{k=0}^{j-1}{\frac{\alpha^{j-1}}{(j-1)!}  \frac{(j-1)!(-1)^{j-1-k}}{(j-1-k)!}\frac{z^{k}}{k!}}}\\
\end{split}
\end{displaymath}
yielding equation
\begin{equation}
  g(z) = e^{\alpha}\sum_{j=1}^{m}{\sum_{k=0}^{j-1}{\frac{(-\alpha)^{j-1}}{(j-1)!}{{j-1}\choose{k}}(-z)^{k}}} 
\end{equation}
We swap summations holding the same argument explained in previous section:
\begin{displaymath}
  g(z) = e^{\alpha}\sum_{k=0}^{m-1}{\left(\sum_{j=k+1}^{m}{\frac{(-\alpha)^{j-1}}{(j-1)!}{{j-1}\choose{k}}}\right)(-z)^{k}}
\end{displaymath}
yielding equation
\begin{equation}
  g(z) = e^{\alpha}\sum_{k=0}^{m-1}{\left(\sum_{j=k}^{m-1}{\frac{(-\alpha)^{j}}{j!}{{j}\choose{k}}}\right)(-z)^{k}}
\end{equation}
namely $e^{\alpha\mathcal{R}_{m}}=g(\mathcal{R}_{m})$ for any proper, 
finite Riordan matrix $\mathcal{R}_{m}$.

For the sake of clarity, let $m=8$ to define polynomial $g$:
\begin{displaymath}
\hspace{-3cm}
\begin{split}
g{\left (z \right )} = e^{\alpha}&\left(\frac{\alpha^{7} z^{7}}{5040}\right. \\
&+ \frac{\alpha^{6} z^{6}}{720} \left(- \alpha + 1\right) \\
&+ \frac{\alpha^{5} z^{5}}{240} \left(\alpha^{2} - 2 \alpha + 2\right) \\
&+ \frac{\alpha^{4} z^{4}}{144} \left(- \alpha^{3} + 3 \alpha^{2} - 6 \alpha + 6\right) \\
&+ \frac{\alpha^{3} z^{3}}{144} \left(\alpha^{4} - 4 \alpha^{3} + 12 \alpha^{2} - 24 \alpha + 24\right) \\
&+ \frac{\alpha^{2} z^{2}}{240} \left(- \alpha^{5} + 5 \alpha^{4} - 20 \alpha^{3} + 60 \alpha^{2} - 120 \alpha + 120\right) \\
&+ \frac{\alpha z}{720} \left(\alpha^{6} - 6 \alpha^{5} + 30 \alpha^{4} - 120 \alpha^{3} + 360 \alpha^{2} - 720 \alpha + 720\right) \\
&- \left.\frac{\alpha^{7}}{5040} + \frac{\alpha^{6}}{720} - \frac{\alpha^{5}}{120} + \frac{\alpha^{4}}{24} - \frac{\alpha^{3}}{6} + \frac{\alpha^{2}}{2} -\alpha + 1\right) 
\end{split}
\end{displaymath}
choosing $\alpha=1$ yields:
\begin{displaymath}
\hspace{-1cm}
%\begin{split}
g{\left (z \right )} = e \left(\frac{z^{7}}{5040} + \frac{z^{5}}{240} + \frac{z^{4}}{72} + \frac{z^{3}}{16} + \frac{11 z^{2}}{60} + \frac{53 z}{144} + \frac{103}{280}\right)
%\end{split}
\end{displaymath}
on the other hand, choosing $\alpha=-1$ yields:
\begin{displaymath}
g{\left (z \right )} = \frac{1}{e} \left( - \frac{z^{7}}{5040} + \frac{z^{6}}{360} - \frac{z^{5}}{48} + \frac{z^{4}}{9}\right. - \left.\frac{65 z^{3}}{144} + \frac{163 z^{2}}{120} - \frac{1957 z}{720} + \frac{685}{252}\right)
\end{displaymath}

Using Riordan array characterization we have 
\begin{displaymath}
D_{e^{\alpha z}}E_{\lambda_{1}} = e^{\alpha \lambda_{1}} \left[\begin{matrix}1 & 0 & 0 & 0 & 0 & 0 & 0 & 0\\- \alpha  \lambda_{1} & \alpha  & 0 & 0 & 0 & 0 & 0 & 0\\\frac{\alpha^{2} \lambda_{1}^{2}}{2}  & - \alpha^{2}  \lambda_{1} & \alpha^{2}  & 0 & 0 & 0 & 0 & 0\\- \frac{\alpha^{3} \lambda_{1}^{3}}{6}  & \frac{\alpha^{3} \lambda_{1}^{2}}{2}  & - \alpha^{3}  \lambda_{1} & \alpha^{3}  & 0 & 0 & 0 & 0\\\frac{\alpha^{4} \lambda_{1}^{4}}{24}  & - \frac{\alpha^{4} \lambda_{1}^{3}}{6}  & \frac{\alpha^{4} \lambda_{1}^{2}}{2}  & - \alpha^{4}  \lambda_{1} & \alpha^{4}  & 0 & 0 & 0\\- \frac{\alpha^{5} \lambda_{1}^{5}}{120}  & \frac{\alpha^{5} \lambda_{1}^{4}}{24}  & - \frac{\alpha^{5} \lambda_{1}^{3}}{6}  & \frac{\alpha^{5} \lambda_{1}^{2}}{2}  & - \alpha^{5}  \lambda_{1} & \alpha^{5}  & 0 & 0\\\frac{\alpha^{6} \lambda_{1}^{6}}{720}  & - \frac{\alpha^{6} \lambda_{1}^{5}}{120}  & \frac{\alpha^{6} \lambda_{1}^{4}}{24}  & - \frac{\alpha^{6} \lambda_{1}^{3}}{6}  & \frac{\alpha^{6} \lambda_{1}^{2}}{2}  & - \alpha^{6}  \lambda_{1} & \alpha^{6}  & 0\\- \frac{\alpha^{7} \lambda_{1}^{7}}{5040}  & \frac{\alpha^{7} \lambda_{1}^{6}}{720}  & - \frac{\alpha^{7} \lambda_{1}^{5}}{120}  & \frac{\alpha^{7} \lambda_{1}^{4}}{24}  & - \frac{\alpha^{7} \lambda_{1}^{3}}{6}  & \frac{\alpha^{7} \lambda_{1}^{2}}{2}  & - \alpha^{7}  \lambda_{1} & \alpha^{7} \end{matrix}\right]
\end{displaymath}
generated by the production matrix
\begin{displaymath}
\left[\begin{matrix}- \alpha \lambda_{1} & \alpha & 0 & 0 & 0 & 0 & 0\\- \frac{\alpha \lambda_{1}^{2}}{2} & 0 & \alpha & 0 & 0 & 0 & 0\\- \frac{\alpha \lambda_{1}^{3}}{6} & 0 & 0 & \alpha & 0 & 0 & 0\\- \frac{\alpha \lambda_{1}^{4}}{24} & 0 & 0 & 0 & \alpha & 0 & 0\\- \frac{\alpha \lambda_{1}^{5}}{120} & 0 & 0 & 0 & 0 & \alpha & 0\\- \frac{\alpha \lambda_{1}^{6}}{720} & 0 & 0 & 0 & 0 & 0 & \alpha\\- \frac{\alpha \lambda_{1}^{7}}{5040} & 0 & 0 & 0 & 0 & 0 & 0\end{matrix}\right]
\end{displaymath}
so the matrix satisfies the recurrence relation
\begin{displaymath}
\begin{split}
d_{0,0}&=e^{\alpha \lambda_{1}}\\
d_{n,0}&=\alpha\sum_{k=0}^{n-1}{d_{n-1, k}\frac{\lambda_{1}^{k+1}}{(k+1)!}}, \quad n>0 \\
d_{n,k}&=\alpha d_{n-1, k-1}, \quad n,k > 0\\
\end{split}
\end{displaymath}
finally,
\begin{displaymath}
D_{e^{\alpha z}}E_{\lambda_{1}}\boldsymbol{z} = e^{\alpha \lambda_{1}}\left[\begin{matrix}1\\\alpha \left(z - \lambda_{1}\right) \\\frac{\alpha^{2}}{2} \left(z - \lambda_{1}\right)^{2} \\\frac{\alpha^{3}}{6} \left(z - \lambda_{1}\right)^{3} \\\frac{\alpha^{4}}{24} \left(z - \lambda_{1}\right)^{4} \\\frac{\alpha^{5}}{120} \left(z - \lambda_{1}\right)^{5} \\\frac{\alpha^{6}}{720} \left(z - \lambda_{1}\right)^{6} \\\frac{\alpha^{7}}{5040} \left(z - \lambda_{1}\right)^{7} \end{matrix}\right]
\end{displaymath}
therefore restoring $\lambda_{1}=1$ yields the polynomial
\begin{displaymath}
\begin{split}
g{\left (z \right )} = \boldsymbol{1}^{T}D_{e^{\alpha z}}E_{\lambda_{1}}\boldsymbol{z} = e^{\alpha} &\left(\frac{\alpha^{7} }{5040} \left(z - 1\right)^{7} + \frac{\alpha^{6} }{720} \left(z - 1\right)^{6} + \frac{\alpha^{5} }{120} \left(z - 1\right)^{5} + \frac{\alpha^{4} }{24} \left(z - 1\right)^{4}\right.\\
    &+ \left. \frac{\alpha^{3} }{6} \left(z - 1\right)^{3} + \frac{\alpha^{2} }{2} \left(z - 1\right)^{2} + \alpha \left(z - 1\right)  + 1\right)
\end{split}
\end{displaymath}
hence we generalize for $m\in\mathbb{N}$
\begin{displaymath}
e^{\alpha \mathcal{R}_{m}} = g{\left (\mathcal{R}_{m} \right )} =e^{\alpha} \sum_{j=0}^{m-1}{\frac{\alpha^{j}}{j!}{\left(Z_{1,2}^{[\mathcal{R}_{m}]}\right)^{j} }} = e^{\alpha\left(1+Z_{1,2}^{[\mathcal{R}_{m}]}\right)}
\end{displaymath}
moreover, the limit for $m \rightarrow \infty$ yields $ g{\left (\mathcal{R} \right )} = e^{\alpha \mathcal{R}} $ for the whole Riordan array $\mathcal{R}$.


\subsection{$f(z)=\log{z}$}


The general form of $j$th derivative of function $f$ is 
$$\frac{\partial^{(j)}{f}(z)}{\partial{z}} =\frac{(-1)^{j-1}(j-1)!}{z^{j}}  $$ 
for $j \geq 1$, therefore
\begin{displaymath}
\begin{split}
  g(z) &= \sum_{j=1}^{m}{ \left. \frac{\partial^{(j-1)}{f}}{\partial{z}} \right|_{z=\lambda_{1}}\Phi_{1,j}(z)} \\
       &= \sum_{j=0}^{m-1}{ \left. \frac{\partial^{(j)}{f}}{\partial{z}} \right|_{z=1}\Phi_{1,j+1}(z)} \\
       &= f(1)\Phi_{1,1}(z) + \sum_{j=1}^{m-1}{ \left. \frac{\partial^{(j)}{f}}{\partial{z}} \right|_{z=1}\Phi_{1,j+1}(z)} \\
\end{split}
\end{displaymath}
substitution of known quantities allows us to state the equation
\begin{equation}
  g(z) = - \sum_{j=1}^{m-1}{\sum_{k=0}^{j}{\frac{1}{j}{{j}\choose{k}}(-z)^{k}}}
\end{equation}
To swap summations, we repeat the same argument given for function $\sqrt{z}$, splitting
a term due to $k=0$ from the whole summation:
\begin{equation}
  g(z) = - \sum_{k=1}^{m-1}{\left(\sum_{j=k}^{m-1}{\frac{1}{j}{{j}\choose{k}}}\right)}(-z)^{k} - H_{m-1} 
\end{equation}
where $H_{n}$ is the $n$-th harmonic number; finally, the inner summation admits
a closed form:
\begin{equation}
  g(z) = - \sum_{k=1}^{m-1}\frac{1}{k}{{m-1}\choose{k}}{(-z)^{k}}- H_{m-1} 
\end{equation}

For the sake of clarity, here is $g$ polynomial where $m=8$ and relaxing $\lambda=1$ hypothesis:
\begin{displaymath}
\begin{split}
g{\left (z \right )} &= \frac{z^{7}}{7 \lambda^{7}} \\
&+ z^{6} \left(- \frac{1}{6 \lambda^{6}} - \frac{1}{\lambda^{7}}\right) \\
&+ z^{5} \left(\frac{1}{5 \lambda^{5}} + \frac{1}{\lambda^{6}} + \frac{3}{\lambda^{7}}\right) \\
&+ z^{4} \left(- \frac{1}{4 \lambda^{4}} - \frac{1}{\lambda^{5}} - \frac{5}{2 \lambda^{6}} - \frac{5}{\lambda^{7}}\right) \\
&+ z^{3} \left(\frac{1}{3 \lambda^{3}} + \frac{1}{\lambda^{4}} + \frac{2}{\lambda^{5}} + \frac{10}{3 \lambda^{6}} + \frac{5}{\lambda^{7}}\right) \\
&+ z^{2} \left(- \frac{1}{2 \lambda^{2}} - \frac{1}{\lambda^{3}} - \frac{3}{2 \lambda^{4}} - \frac{2}{\lambda^{5}} - \frac{5}{2 \lambda^{6}} - \frac{3}{\lambda^{7}}\right) \\
&+ z \left(\frac{1}{\lambda} + \frac{1}{\lambda^{2}} + \frac{1}{\lambda^{3}} + \frac{1}{\lambda^{4}} + \frac{1}{\lambda^{5}} + \frac{1}{\lambda^{6}} + \frac{1}{\lambda^{7}}\right) \\
&+ \log{\left (\lambda \right )} - \frac{1}{\lambda} - \frac{1}{2 \lambda^{2}} - \frac{1}{3 \lambda^{3}} - \frac{1}{4 \lambda^{4}} - \frac{1}{5 \lambda^{5}} - \frac{1}{6 \lambda^{6}} - \frac{1}{7 \lambda^{7}}
\end{split}
\end{displaymath}
restoring $\lambda=1$:
\begin{displaymath}
%\hspace{-1cm}
%\begin{split}
g{\left (z \right )} = \frac{z^{7}}{7} - \frac{7 z^{6}}{6} + \frac{21 z^{5}}{5} - \frac{35 z^{4}}{4} + \frac{35 z^{3}}{3} - \frac{21 z^{2}}{2} + 7 z - \frac{363}{140}
%\end{split}
\end{displaymath}


Using Riordan array characterization we have 
\begin{displaymath}
D_{\log{z}}E_{\lambda_{1}} = \left[\begin{matrix}\log{\left (\lambda_{1} \right )} & 0 & 0 & 0 & 0 & 0 & 0 & 0\\-1 & \frac{1}{\lambda_{1}} & 0 & 0 & 0 & 0 & 0 & 0\\- \frac{1}{2} & \frac{1}{\lambda_{1}} & - \frac{1}{\lambda_{1}^{2}} & 0 & 0 & 0 & 0 & 0\\- \frac{1}{3} & \frac{1}{\lambda_{1}} & - \frac{2}{\lambda_{1}^{2}} & \frac{2}{\lambda_{1}^{3}} & 0 & 0 & 0 & 0\\- \frac{1}{4} & \frac{1}{\lambda_{1}} & - \frac{3}{\lambda_{1}^{2}} & \frac{6}{\lambda_{1}^{3}} & - \frac{6}{\lambda_{1}^{4}} & 0 & 0 & 0\\- \frac{1}{5} & \frac{1}{\lambda_{1}} & - \frac{4}{\lambda_{1}^{2}} & \frac{12}{\lambda_{1}^{3}} & - \frac{24}{\lambda_{1}^{4}} & \frac{24}{\lambda_{1}^{5}} & 0 & 0\\- \frac{1}{6} & \frac{1}{\lambda_{1}} & - \frac{5}{\lambda_{1}^{2}} & \frac{20}{\lambda_{1}^{3}} & - \frac{60}{\lambda_{1}^{4}} & \frac{120}{\lambda_{1}^{5}} & - \frac{120}{\lambda_{1}^{6}} & 0\\- \frac{1}{7} & \frac{1}{\lambda_{1}} & - \frac{6}{\lambda_{1}^{2}} & \frac{30}{\lambda_{1}^{3}} & - \frac{120}{\lambda_{1}^{4}} & \frac{360}{\lambda_{1}^{5}} & - \frac{720}{\lambda_{1}^{6}} & \frac{720}{\lambda_{1}^{7}}\end{matrix}\right]
\end{displaymath}
generated by the production matrix
\begin{displaymath}
\left[\begin{matrix}- \frac{1}{\log{\left (\lambda_{1} \right )}} & \frac{1}{\log{\left (\lambda_{1} \right )} \lambda_{1}} & 0 & 0 & 0 & 0 & 0\\- \frac{\lambda_{1}}{2} - \frac{\lambda_{1}}{\log{\left (\lambda_{1} \right )}} & 1 + \frac{1}{\log{\left (\lambda_{1} \right )}} & - \frac{1}{\lambda_{1}} & 0 & 0 & 0 & 0\\- \frac{\left(\log{\left (\lambda_{1} \right )} + 3\right) \lambda_{1}^{2}}{6 \log{\left (\lambda_{1} \right )}} & \frac{\lambda_{1}}{2 \log{\left (\lambda_{1} \right )}} & 1 & - \frac{2}{\lambda_{1}} & 0 & 0 & 0\\- \frac{\left(\log{\left (\lambda_{1} \right )} + 4\right) \lambda_{1}^{3}}{24 \log{\left (\lambda_{1} \right )}} & \frac{\lambda_{1}^{2}}{6 \log{\left (\lambda_{1} \right )}} & 0 & 1 & - \frac{3}{\lambda_{1}} & 0 & 0\\- \frac{\left(\log{\left (\lambda_{1} \right )} + 5\right) \lambda_{1}^{4}}{120 \log{\left (\lambda_{1} \right )}} & \frac{\lambda_{1}^{3}}{24 \log{\left (\lambda_{1} \right )}} & 0 & 0 & 1 & - \frac{4}{\lambda_{1}} & 0\\- \frac{\left(\log{\left (\lambda_{1} \right )} + 6\right) \lambda_{1}^{5}}{720 \log{\left (\lambda_{1} \right )}} & \frac{\lambda_{1}^{4}}{120 \log{\left (\lambda_{1} \right )}} & 0 & 0 & 0 & 1 & - \frac{5}{\lambda_{1}}\\- \frac{\left(\log{\left (\lambda_{1} \right )} + 7\right) \lambda_{1}^{6}}{5040 \log{\left (\lambda_{1} \right )}} & \frac{\lambda_{1}^{5}}{720 \log{\left (\lambda_{1} \right )}} & 0 & 0 & 0 & 0 & 1\end{matrix}\right]
\end{displaymath}
so the matrix satisfies the recurrence relation (\textbf{to be fix})
\begin{displaymath}
\begin{split}
d_{0,0}&=\lambda_{1}^{r}\\
d_{n,0}&=-\left(r d_{n-1, 0} + (r+1)\sum_{k=1}^{n-1}{d_{n-1, k}\frac{\lambda_{1}^{k}}{(k+1)!}}\right), \quad n>0 \\
d_{n,k}&=\frac{r+1-k}{\lambda_{1}}d_{n-1, k-1} + d_{n-1,k}, \quad n,k > 0\\
\end{split}
\end{displaymath}
finally,
\begin{displaymath}
D_{\log{z}}E_{\lambda_{1}}\boldsymbol{z} = \left[\begin{matrix}\log{\left (\lambda_{1} \right )}\\\frac{1}{\lambda_{1}} \left(z - \lambda_{1}\right)\\- \frac{\left(z - \lambda_{1}\right)^{2}}{2 \lambda_{1}^{2}}\\\frac{\left(z - \lambda_{1}\right)^{3}}{3 \lambda_{1}^{3}}\\- \frac{\left(z - \lambda_{1}\right)^{4}}{4 \lambda_{1}^{4}}\\\frac{\left(z - \lambda_{1}\right)^{5}}{5 \lambda_{1}^{5}}\\- \frac{\left(z - \lambda_{1}\right)^{6}}{6 \lambda_{1}^{6}}\\\frac{\left(z - \lambda_{1}\right)^{7}}{7 \lambda_{1}^{7}}\end{matrix}\right]
\end{displaymath}
therefore restoring $\lambda_{1}=1$ yields the polynomial
\begin{displaymath}
g{\left (z \right )} = \boldsymbol{1}^{T}D_{\log{z}}E_{\lambda_{1}}\boldsymbol{z} = \frac{1}{7} \left(z - 1\right)^{7} - \frac{1}{6} \left(z - 1\right)^{6} + \frac{1}{5} \left(z - 1\right)^{5} - \frac{1}{4} \left(z - 1\right)^{4} + \frac{1}{3} \left(z - 1\right)^{3} - \frac{1}{2} \left(z - 1\right)^{2} + (z - 1)
\end{displaymath}
hence we generalize for $m\in\mathbb{N}$:
\begin{displaymath}
\log{\mathcal{R}_{m}} = g{\left (\mathcal{R}_{m} \right )} = \sum_{j=1}^{m-1}{\frac{(-1)^{j+1}}{j}{\left(Z_{1,2}^{[\mathcal{R}_{m}]}\right)^{j} }} = \log{\left(1 + Z_{1,2}^{[\mathcal{R}_{m}]}\right)}
\end{displaymath}
moreover, the limit for $m \rightarrow \infty$ yields $ g{\left (\mathcal{R} \right )} = \log{\mathcal{R}} $ for the whole Riordan array $\mathcal{R}$.



\section{Case studies}

\subsection{Pascal array $\mathcal{P}$}


Let $m=8$ and define
\begin{displaymath}
%\mathcal{P}_{m}=\left[\begin{matrix}1 & 0 & 0 & 0 & 0 & 0 & 0 & 0\\1 & 1 & 0 & 0 & 0 & 0 & 0 & 0\\1 & 2 & 1 & 0 & 0 & 0 & 0 & 0\\1 & 3 & 3 & 1 & 0 & 0 & 0 & 0\\1 & 4 & 6 & 4 & 1 & 0 & 0 & 0\\1 & 5 & 10 & 10 & 5 & 1 & 0 & 0\\1 & 6 & 15 & 20 & 15 & 6 & 1 & 0\\1 & 7 & 21 & 35 & 35 & 21 & 7 & 1\end{matrix}\right]
\mathcal{P}_{m}=\left[\begin{matrix}1 &   &   &   &   &   &   &  \\1 & 1 &   &   &   &   &   &  \\1 & 2 & 1 &   &   &   &   &  \\1 & 3 & 3 & 1 &   &   &   &  \\1 & 4 & 6 & 4 & 1 &   &   &  \\1 & 5 & 10 & 10 & 5 & 1 &   &  \\1 & 6 & 15 & 20 & 15 & 6 & 1 &  \\1 & 7 & 21 & 35 & 35 & 21 & 7 & 1\end{matrix}\right]
\end{displaymath}
therefore $\mathcal{P}_{m}^{-1}$ is
\begin{displaymath}
%\left[\begin{matrix}1 & 0 & 0 & 0 & 0 & 0 & 0 & 0\\-1 & 1 & 0 & 0 & 0 & 0 & 0 & 0\\1 & -2 & 1 & 0 & 0 & 0 & 0 & 0\\-1 & 3 & -3 & 1 & 0 & 0 & 0 & 0\\1 & -4 & 6 & -4 & 1 & 0 & 0 & 0\\-1 & 5 & -10 & 10 & -5 & 1 & 0 & 0\\1 & -6 & 15 & -20 & 15 & -6 & 1 & 0\\-1 & 7 & -21 & 35 & -35 & 21 & -7 & 1\end{matrix}\right]
\left[\begin{matrix}1 &   &   &   &   &   &   &  \\-1 & 1 &   &   &   &   &   &  \\1 & -2 & 1 &   &   &   &   &  \\-1 & 3 & -3 & 1 &   &   &   &  \\1 & -4 & 6 & -4 & 1 &   &   &  \\-1 & 5 & -10 & 10 & -5 & 1 &   &  \\1 & -6 & 15 & -20 & 15 & -6 & 1 &  \\-1 & 7 & -21 & 35 & -35 & 21 & -7 & 1\end{matrix}\right]
\end{displaymath}
$\mathcal{P}_{m}^{r}$ is
\begin{displaymath}
%\left[\begin{matrix}1 & 0 & 0 & 0 & 0 & 0 & 0 & 0\\r & 1 & 0 & 0 & 0 & 0 & 0 & 0\\r^{2} & 2 r & 1 & 0 & 0 & 0 & 0 & 0\\r^{3} & 3 r^{2} & 3 r & 1 & 0 & 0 & 0 & 0\\r^{4} & 4 r^{3} & 6 r^{2} & 4 r & 1 & 0 & 0 & 0\\r^{5} & 5 r^{4} & 10 r^{3} & 10 r^{2} & 5 r & 1 & 0 & 0\\r^{6} & 6 r^{5} & 15 r^{4} & 20 r^{3} & 15 r^{2} & 6 r & 1 & 0\\r^{7} & 7 r^{6} & 21 r^{5} & 35 r^{4} & 35 r^{3} & 21 r^{2} & 7 r & 1\end{matrix}\right]
\left[\begin{matrix}1 &   &   &   &   &   &   &  \\r & 1 &   &   &   &   &   &  \\r^{2} & 2 r & 1 &   &   &   &   &  \\r^{3} & 3 r^{2} & 3 r & 1 &   &   &   &  \\r^{4} & 4 r^{3} & 6 r^{2} & 4 r & 1 &   &   &  \\r^{5} & 5 r^{4} & 10 r^{3} & 10 r^{2} & 5 r & 1 &   &  \\r^{6} & 6 r^{5} & 15 r^{4} & 20 r^{3} & 15 r^{2} & 6 r & 1 &  \\r^{7} & 7 r^{6} & 21 r^{5} & 35 r^{4} & 35 r^{3} & 21 r^{2} & 7 r & 1\end{matrix}\right]
\end{displaymath}
$\sqrt{\mathcal{P}_{m}}$ is
\begin{displaymath}
%\left[\begin{matrix}1 & 0 & 0 & 0 & 0 & 0 & 0 & 0\\\frac{1}{2} & 1 & 0 & 0 & 0 & 0 & 0 & 0\\\frac{1}{4} & 1 & 1 & 0 & 0 & 0 & 0 & 0\\\frac{1}{8} & \frac{3}{4} & \frac{3}{2} & 1 & 0 & 0 & 0 & 0\\\frac{1}{16} & \frac{1}{2} & \frac{3}{2} & 2 & 1 & 0 & 0 & 0\\\frac{1}{32} & \frac{5}{16} & \frac{5}{4} & \frac{5}{2} & \frac{5}{2} & 1 & 0 & 0\\\frac{1}{64} & \frac{3}{16} & \frac{15}{16} & \frac{5}{2} & \frac{15}{4} & 3 & 1 & 0\\\frac{1}{128} & \frac{7}{64} & \frac{21}{32} & \frac{35}{16} & \frac{35}{8} & \frac{21}{4} & \frac{7}{2} & 1\end{matrix}\right]
\left[\begin{matrix}1 &   &   &   &   &   &   &  \\\frac{1}{2} & 1 &   &   &   &   &   &  \\\frac{1}{4} & 1 & 1 &   &   &   &   &  \\\frac{1}{8} & \frac{3}{4} & \frac{3}{2} & 1 &   &   &   &  \\\frac{1}{16} & \frac{1}{2} & \frac{3}{2} & 2 & 1 &   &   &  \\\frac{1}{32} & \frac{5}{16} & \frac{5}{4} & \frac{5}{2} & \frac{5}{2} & 1 &   &  \\\frac{1}{64} & \frac{3}{16} & \frac{15}{16} & \frac{5}{2} & \frac{15}{4} & 3 & 1 &  \\\frac{1}{128} & \frac{7}{64} & \frac{21}{32} & \frac{35}{16} & \frac{35}{8} & \frac{21}{4} & \frac{7}{2} & 1\end{matrix}\right]
\end{displaymath}
finally, $e^{\mathcal{P}_{m}}$ is
\begin{displaymath}
%e \left[\begin{matrix}1 & 0 & 0 & 0 & 0 & 0 & 0 & 0\\1 & 1 & 0 & 0 & 0 & 0 & 0 & 0\\2 & 2 & 1 & 0 & 0 & 0 & 0 & 0\\5 & 6 & 3 & 1 & 0 & 0 & 0 & 0\\15 & 20 & 12 & 4 & 1 & 0 & 0 & 0\\52 & 75 & 50 & 20 & 5 & 1 & 0 & 0\\203 & 312 & 225 & 100 & 30 & 6 & 1 & 0\\877 & 1421 & 1092 & 525 & 175 & 42 & 7 & 1\end{matrix}\right]
e \left[\begin{matrix}1 &   &   &   &   &   &   &  \\1 & 1 &   &   &   &   &   &  \\2 & 2 & 1 &   &   &   &   &  \\5 & 6 & 3 & 1 &   &   &   &  \\15 & 20 & 12 & 4 & 1 &   &   &  \\52 & 75 & 50 & 20 & 5 & 1 &   &  \\203 & 312 & 225 & 100 & 30 & 6 & 1 &  \\877 & 1421 & 1092 & 525 & 175 & 42 & 7 & 1\end{matrix}\right]
\end{displaymath}
such matrix is known as $A056857$ in the OEIS, there we found
the interesting relation $e^{\mathcal{P}_{m}}=e\cdot\left(\mathcal{S}_{m}\cdot \mathcal{P}_{m}\cdot \mathcal{S}_{m}^{-1}\right)$,
where $\mathcal{S}_{m}$ is the matrix of Stirling numbers of the second kind, defined in a later case study.
Also, it is interesting the application of function $f(z)=e^{\alpha z}$ to the matrix $H_{m}$ defined as
\begin{displaymath}
%H = \left[\begin{matrix}0 & 0 & 0 & 0 & 0 & 0 & 0 & 0\\1 & 0 & 0 & 0 & 0 & 0 & 0 & 0\\0 & 2 & 0 & 0 & 0 & 0 & 0 & 0\\0 & 0 & 3 & 0 & 0 & 0 & 0 & 0\\0 & 0 & 0 & 4 & 0 & 0 & 0 & 0\\0 & 0 & 0 & 0 & 5 & 0 & 0 & 0\\0 & 0 & 0 & 0 & 0 & 6 & 0 & 0\\0 & 0 & 0 & 0 & 0 & 0 & 7 & 0\end{matrix}\right]
H_{m} = \left[\begin{matrix} 0 &   &   &   &   &   &   &  \\1 & 0   &   &   &   &   &   &  \\  & 2 &  0  &   &   &   &   &  \\  &   & 3 &  0  &   &   &   &  \\  &   &   & 4 &  0  &   &   &  \\  &   &   &   & 5 &  0  &   &  \\  &   &   &   &   & 6 &  0  &  \\  &   &   &   &   &   & 7 &  0 \end{matrix}\right]
\end{displaymath}
It requires the generalized Lagrange base
\begin{displaymath}
\begin{split}
&\left\{\Phi_{ 1, 1 }{\left (z \right )} = 1, \Phi_{ 1, 2 }{\left (z \right )} = z, \Phi_{ 1, 3 }{\left (z \right )} = \frac{z^{2}}{2},\right.  \Phi_{ 1, 4 }{\left (z \right )} = \frac{z^{3}}{6}, \\
&\Phi_{ 1, 5 }{\left (z \right )} = \frac{z^{4}}{24}, \Phi_{ 1, 6 }{\left (z \right )} = \frac{z^{5}}{120}, \left.\Phi_{ 1, 7 }{\left (z \right )} = \frac{z^{6}}{720}, \Phi_{ 1, 8 }{\left (z \right )} = \frac{z^{7}}{5040}\right\}
\end{split}
\end{displaymath}
to define the interpolating polynomial $g$ as
\begin{displaymath}
g{\left (z \right )} = \frac{\alpha^{7} z^{7}}{5040} + \frac{\alpha^{6} z^{6}}{720} + \frac{\alpha^{5} z^{5}}{120} + \frac{\alpha^{4} z^{4}}{24} + \frac{\alpha^{3} z^{3}}{6} + \frac{\alpha^{2} z^{2}}{2} + \alpha z + 1
\end{displaymath}
therefore $g(H_{m})=e^{\alpha H_{m}}$ is the matrix
\begin{displaymath}
%\left[\begin{matrix}1 & 0 & 0 & 0 & 0 & 0 & 0 & 0\\\alpha & 1 & 0 & 0 & 0 & 0 & 0 & 0\\\alpha^{2} & 2 \alpha & 1 & 0 & 0 & 0 & 0 & 0\\\alpha^{3} & 3 \alpha^{2} & 3 \alpha & 1 & 0 & 0 & 0 & 0\\\alpha^{4} & 4 \alpha^{3} & 6 \alpha^{2} & 4 \alpha & 1 & 0 & 0 & 0\\\alpha^{5} & 5 \alpha^{4} & 10 \alpha^{3} & 10 \alpha^{2} & 5 \alpha & 1 & 0 & 0\\\alpha^{6} & 6 \alpha^{5} & 15 \alpha^{4} & 20 \alpha^{3} & 15 \alpha^{2} & 6 \alpha & 1 & 0\\\alpha^{7} & 7 \alpha^{6} & 21 \alpha^{5} & 35 \alpha^{4} & 35 \alpha^{3} & 21 \alpha^{2} & 7 \alpha & 1\end{matrix}\right]
\left[\begin{matrix}1 &   &   &   &   &   &   &  \\\alpha & 1 &   &   &   &   &   &  \\\alpha^{2} & 2 \alpha & 1 &   &   &   &   &  \\\alpha^{3} & 3 \alpha^{2} & 3 \alpha & 1 &   &   &   &  \\\alpha^{4} & 4 \alpha^{3} & 6 \alpha^{2} & 4 \alpha & 1 &   &   &  \\\alpha^{5} & 5 \alpha^{4} & 1  \alpha^{3} & 1  \alpha^{2} & 5 \alpha & 1 &   &  \\\alpha^{6} & 6 \alpha^{5} & 15 \alpha^{4} & 2  \alpha^{3} & 15 \alpha^{2} & 6 \alpha & 1 &  \\\alpha^{7} & 7 \alpha^{6} & 21 \alpha^{5} & 35 \alpha^{4} & 35 \alpha^{3} & 21 \alpha^{2} & 7 \alpha & 1\end{matrix}\right]
\end{displaymath}
yielding the identity $e^{\alpha H_{m}} = \mathcal{P}_{m}^{\alpha}$;
moreover, $e^{(\alpha+\beta) H_{m}} = \mathcal{P}_{m}^{\alpha+\beta} = \mathcal{P}_{m}^{\alpha}\mathcal{P}_{m}^{\beta} $ 
also holds, as well as $\log{\mathcal{P}_{m}}=H_{m}$.

We define a new matrix $\hat{H}_{m}$ having the same shape of $H_{m}$, making
coefficients symbolic values:
\begin{displaymath}
%\left[\begin{matrix}0 & 0 & 0 & 0 & 0 & 0 & 0 & 0\\h_{1} & 0 & 0 & 0 & 0 & 0 & 0 & 0\\0 & h_{2} & 0 & 0 & 0 & 0 & 0 & 0\\0 & 0 & h_{3} & 0 & 0 & 0 & 0 & 0\\0 & 0 & 0 & h_{4} & 0 & 0 & 0 & 0\\0 & 0 & 0 & 0 & h_{5} & 0 & 0 & 0\\0 & 0 & 0 & 0 & 0 & h_{6} & 0 & 0\\0 & 0 & 0 & 0 & 0 & 0 & h_{7} & 0\end{matrix}\right]
\hat{H}_{m} = \left[\begin{matrix}  &   &   &   &   &   &   &  \\h_{1} &   &   &   &   &   &   &  \\  & h_{2} &   &   &   &   &   &  \\  &   & h_{3} &   &   &   &   &  \\  &   &   & h_{4} &   &   &   &  \\  &   &   &   & h_{5} &   &   &  \\  &   &   &   &   & h_{6} &   &  \\  &   &   &   &   &   & h_{7} &  \end{matrix}\right]
\end{displaymath}
such that $e^{\alpha\hat{H}_{m}}=\mathcal{P}^{\alpha} \leftrightarrow h_{i}=i$, 
for $i\in \lbrace 1,\ldots,7 \rbrace$, generalizing previous argument; 
for completeness, we report the whole matrix $e^{\alpha\hat{H}_{m}}$:
\begin{displaymath}
\footnotesize
e^{\alpha\hat{H}_{m}} = \left[\begin{matrix}1 &   &   &   &   &   &   & \\\alpha h_{1} & 1 &   &   &   &   &   & \\\frac{\alpha^{2} }{2}h_{1} h_{2} & \alpha h_{2} & 1 &   &   &   &   & \\\frac{\alpha^{3} }{6}h_{1} h_{2} h_{3} & \frac{\alpha^{2} }{2}h_{2} h_{3} & \alpha h_{3} & 1 &   &   &   & \\\frac{\alpha^{4} }{24}h_{1} h_{2} h_{3} h_{4} & \frac{\alpha^{3} }{6}h_{2} h_{3} h_{4} & \frac{\alpha^{2} }{2}h_{3} h_{4} & \alpha h_{4} & 1 &   &   & \\\frac{\alpha^{5} }{120}h_{1} h_{2} h_{3} h_{4} h_{5} & \frac{\alpha^{4} }{24}h_{2} h_{3} h_{4} h_{5} & \frac{\alpha^{3} }{6}h_{3} h_{4} h_{5} & \frac{\alpha^{2} }{2}h_{4} h_{5} & \alpha h_{5} & 1 &   & \\\frac{\alpha^{6} }{720}h_{1} h_{2} h_{3} h_{4} h_{5} h_{6} & \frac{\alpha^{5} }{120}h_{2} h_{3} h_{4} h_{5} h_{6} & \frac{\alpha^{4} }{24}h_{3} h_{4} h_{5} h_{6} & \frac{\alpha^{3} }{6}h_{4} h_{5} h_{6} & \frac{\alpha^{2} }{2}h_{5} h_{6} & \alpha h_{6} & 1 & \\\frac{\alpha^{7} }{5040}h_{1} h_{2} h_{3} h_{4} h_{5} h_{6} h_{7} & \frac{\alpha^{6} }{720}h_{2} h_{3} h_{4} h_{5} h_{6} h_{7} & \frac{\alpha^{5} }{120}h_{3} h_{4} h_{5} h_{6} h_{7} & \frac{\alpha^{4} }{24}h_{4} h_{5} h_{6} h_{7} & \frac{\alpha^{3} }{6}h_{5} h_{6} h_{7} & \frac{\alpha^{2} }{2}h_{6} h_{7} & \alpha h_{7} & 1\end{matrix}\right]
\end{displaymath}



\subsection{Catalan array $\mathcal{C}$}


Let $m=8$ and define
\begin{displaymath}
%\mathcal{C}_{m}=\left[\begin{matrix}1 & 0 & 0 & 0 & 0 & 0 & 0 & 0\\1 & 1 & 0 & 0 & 0 & 0 & 0 & 0\\2 & 2 & 1 & 0 & 0 & 0 & 0 & 0\\5 & 5 & 3 & 1 & 0 & 0 & 0 & 0\\14 & 14 & 9 & 4 & 1 & 0 & 0 & 0\\42 & 42 & 28 & 14 & 5 & 1 & 0 & 0\\132 & 132 & 90 & 48 & 20 & 6 & 1 & 0\\429 & 429 & 297 & 165 & 75 & 27 & 7 & 1\end{matrix}\right]
\mathcal{C}_{m}=\left[\begin{matrix}1 &   &   &   &   &   &   &  \\1 & 1 &   &   &   &   &   &  \\2 & 2 & 1 &   &   &   &   &  \\5 & 5 & 3 & 1 &   &   &   &  \\14 & 14 & 9 & 4 & 1 &   &   &  \\42 & 42 & 28 & 14 & 5 & 1 &   &  \\132 & 132 & 9  & 48 & 2  & 6 & 1 &  \\429 & 429 & 297 & 165 & 75 & 27 & 7 & 1\end{matrix}\right]
\end{displaymath}
therefore $\mathcal{C}_{m}^{-1}$ is
\begin{displaymath}
%\left[\begin{matrix}1 & 0 & 0 & 0 & 0 & 0 & 0 & 0\\-1 & 1 & 0 & 0 & 0 & 0 & 0 & 0\\0 & -2 & 1 & 0 & 0 & 0 & 0 & 0\\0 & 1 & -3 & 1 & 0 & 0 & 0 & 0\\0 & 0 & 3 & -4 & 1 & 0 & 0 & 0\\0 & 0 & -1 & 6 & -5 & 1 & 0 & 0\\0 & 0 & 0 & -4 & 10 & -6 & 1 & 0\\0 & 0 & 0 & 1 & -10 & 15 & -7 & 1\end{matrix}\right]
\left[\begin{matrix}1 &   &   &   &   &   &   &  \\-1 & 1 &   &   &   &   &   &  \\  & -2 & 1 &   &   &   &   &  \\  & 1 & -3 & 1 &   &   &   &  \\  &   & 3 & -4 & 1 &   &   &  \\  &   & -1 & 6 & -5 & 1 &   &  \\  &   &   & -4 & 1  & -6 & 1 &  \\  &   &   & 1 & -1  & 15 & -7 & 1\end{matrix}\right]
\end{displaymath}
since $\mathcal{C}_{m}^{r}$ is too big we report $\mathcal{C}_{m}^{r}\boldsymbol{e}_{1}$ 
\begin{displaymath}
\hspace{-1.5cm}
%the following is the complete matrix expression
%\left[\begin{matrix}1 & 0 & 0 & 0 & 0 & 0 & 0 & 0\\r & 1 & 0 & 0 & 0 & 0 & 0 & 0\\r \left(r + 1\right) & 2 r & 1 & 0 & 0 & 0 & 0 & 0\\\frac{r}{2} \left(2 r^{2} + 5 r + 3\right) & r \left(3 r + 2\right) & 3 r & 1 & 0 & 0 & 0 & 0\\\frac{r}{3} \left(3 r^{3} + 13 r^{2} + 18 r + 8\right) & r \left(4 r^{2} + 7 r + 3\right) & 3 r \left(2 r + 1\right) & 4 r & 1 & 0 & 0 & 0\\\frac{r}{12} \left(12 r^{4} + 77 r^{3} + 178 r^{2} + 175 r + 62\right) & \frac{r}{3} \left(15 r^{3} + 47 r^{2} + 48 r + 16\right) & \frac{r}{2} \left(20 r^{2} + 27 r + 9\right) & 2 r \left(5 r + 2\right) & 5 r & 1 & 0 & 0\\\frac{r}{30} \left(30 r^{5} + 261 r^{4} + 875 r^{3} + 1405 r^{2} + 1075 r + 314\right) & \frac{r}{6} \left(36 r^{4} + 171 r^{3} + 298 r^{2} + 225 r + 62\right) & r \left(15 r^{3} + 37 r^{2} + 30 r + 8\right) & 2 r \left(10 r^{2} + 11 r + 3\right) & 5 r \left(3 r + 1\right) & 6 r & 1 & 0\\\frac{r}{60} \left(60 r^{6} + 669 r^{5} + 3002 r^{4} + 6900 r^{3} + 8510 r^{2} + 5301 r + 1298\right) & \frac{r}{60} \left(420 r^{5} + 2754 r^{4} + 7075 r^{3} + 8860 r^{2} + 5375 r + 1256\right) & \frac{r}{4} \left(84 r^{4} + 319 r^{3} + 448 r^{2} + 275 r + 62\right) & \frac{r}{3} \left(105 r^{3} + 214 r^{2} + 144 r + 32\right) & \frac{5 r}{2} \left(14 r^{2} + 13 r + 3\right) & 3 r \left(7 r + 2\right) & 7 r & 1\end{matrix}\right]
\left[\begin{matrix}1 \\r \\r \left(r + 1\right) \\\frac{r}{2} \left(2 r^{2} + 5 r + 3\right) \\\frac{r}{3} \left(3 r^{3} + 13 r^{2} + 18 r + 8\right) \\\frac{r}{12} \left(12 r^{4} + 77 r^{3} + 178 r^{2} + 175 r + 62\right) \\\frac{r}{30} \left(30 r^{5} + 261 r^{4} + 875 r^{3} + 1405 r^{2} + 1075 r + 314\right) \\\frac{r}{60} \left(60 r^{6} + 669 r^{5} + 3002 r^{4} + 6900 r^{3} + 8510 r^{2} + 5301 r + 1298\right) \end{matrix}\right]
\end{displaymath}
$\sqrt{\mathcal{C}_{m}}$ is
\begin{displaymath}
%\left[\begin{matrix}1 & 0 & 0 & 0 & 0 & 0 & 0 & 0\\\frac{1}{2} & 1 & 0 & 0 & 0 & 0 & 0 & 0\\\frac{3}{4} & 1 & 1 & 0 & 0 & 0 & 0 & 0\\\frac{3}{2} & \frac{7}{4} & \frac{3}{2} & 1 & 0 & 0 & 0 & 0\\\frac{55}{16} & \frac{15}{4} & 3 & 2 & 1 & 0 & 0 & 0\\\frac{545}{64} & \frac{143}{16} & \frac{55}{8} & \frac{9}{2} & \frac{5}{2} & 1 & 0 & 0\\\frac{709}{32} & \frac{727}{32} & \frac{273}{16} & 11 & \frac{25}{4} & 3 & 1 & 0\\\frac{15249}{256} & \frac{3855}{64} & \frac{2853}{64} & \frac{455}{16} & \frac{65}{4} & \frac{33}{4} & \frac{7}{2} & 1\end{matrix}\right]
\left[\begin{matrix}1 &   &   &   &   &   &   &  \\\frac{1}{2} & 1 &   &   &   &   &   &  \\\frac{3}{4} & 1 & 1 &   &   &   &   &  \\\frac{3}{2} & \frac{7}{4} & \frac{3}{2} & 1 &   &   &   &  \\\frac{55}{16} & \frac{15}{4} & 3 & 2 & 1 &   &   &  \\\frac{545}{64} & \frac{143}{16} & \frac{55}{8} & \frac{9}{2} & \frac{5}{2} & 1 &   &  \\\frac{7 9}{32} & \frac{727}{32} & \frac{273}{16} & 11 & \frac{25}{4} & 3 & 1 &  \\\frac{15249}{256} & \frac{3855}{64} & \frac{2853}{64} & \frac{455}{16} & \frac{65}{4} & \frac{33}{4} & \frac{7}{2} & 1\end{matrix}\right]
\end{displaymath}
$e^{\mathcal{C}_{m}}$ is
\begin{displaymath}
%e \left[\begin{matrix}1 & 0 & 0 & 0 & 0 & 0 & 0 & 0\\1 & 1 & 0 & 0 & 0 & 0 & 0 & 0\\3 & 2 & 1 & 0 & 0 & 0 & 0 & 0\\\frac{23}{2} & 8 & 3 & 1 & 0 & 0 & 0 & 0\\\frac{154}{3} & 37 & 15 & 4 & 1 & 0 & 0 & 0\\\frac{1027}{4} & \frac{572}{3} & \frac{163}{2} & 24 & 5 & 1 & 0 & 0\\\frac{7046}{5} & \frac{6439}{6} & 478 & 150 & 35 & 6 & 1 & 0\\\frac{502481}{60} & \frac{390899}{60} & \frac{12005}{4} & \frac{2965}{3} & \frac{495}{2} & 48 & 7 & 1\end{matrix}\right]
e \left[\begin{matrix}1 &   &   &   &   &   &   &  \\1 & 1 &   &   &   &   &   &  \\3 & 2 & 1 &   &   &   &   &  \\\frac{23}{2} & 8 & 3 & 1 &   &   &   &  \\\frac{154}{3} & 37 & 15 & 4 & 1 &   &   &  \\\frac{1 27}{4} & \frac{572}{3} & \frac{163}{2} & 24 & 5 & 1 &   &  \\\frac{7 46}{5} & \frac{6439}{6} & 478 & 15  & 35 & 6 & 1 &  \\\frac{5 2481}{6 } & \frac{39 899}{6 } & \frac{12  5}{4} & \frac{2965}{3} & \frac{495}{2} & 48 & 7 & 1\end{matrix}\right]
\end{displaymath}
finally, $\log{\mathcal{C}_{m}}$ is
\begin{displaymath}
%\left[\begin{matrix}0 & 0 & 0 & 0 & 0 & 0 & 0 & 0\\1 & 0 & 0 & 0 & 0 & 0 & 0 & 0\\1 & 2 & 0 & 0 & 0 & 0 & 0 & 0\\\frac{3}{2} & 2 & 3 & 0 & 0 & 0 & 0 & 0\\\frac{8}{3} & 3 & 3 & 4 & 0 & 0 & 0 & 0\\\frac{31}{6} & \frac{16}{3} & \frac{9}{2} & 4 & 5 & 0 & 0 & 0\\\frac{157}{15} & \frac{31}{3} & 8 & 6 & 5 & 6 & 0 & 0\\\frac{649}{30} & \frac{314}{15} & \frac{31}{2} & \frac{32}{3} & \frac{15}{2} & 6 & 7 & 0\end{matrix}\right]
\left[\begin{matrix}  &   &   &   &   &   &   &  \\1 &   &   &   &   &   &   &  \\1 & 2 &   &   &   &   &   &  \\\frac{3}{2} & 2 & 3 &   &   &   &   &  \\\frac{8}{3} & 3 & 3 & 4 &   &   &   &  \\\frac{31}{6} & \frac{16}{3} & \frac{9}{2} & 4 & 5 &   &   &  \\\frac{157}{15} & \frac{31}{3} & 8 & 6 & 5 & 6 &   &  \\\frac{649}{3 } & \frac{314}{15} & \frac{31}{2} & \frac{32}{3} & \frac{15}{2} & 6 & 7 &  \end{matrix}\right]
\end{displaymath}
evaluating corresponding $g$ polynomials on $\mathcal{C}_{m}$, as required.


\subsection{Stirling array $\mathcal{S}$}


Let $m=8$ and define
\begin{displaymath}
%\mathcal{S}_{m} = \left[\begin{matrix}1 & 0 & 0 & 0 & 0 & 0 & 0 & 0\\0 & 1 & 0 & 0 & 0 & 0 & 0 & 0\\0 & 1 & 1 & 0 & 0 & 0 & 0 & 0\\0 & 1 & 3 & 1 & 0 & 0 & 0 & 0\\0 & 1 & 7 & 6 & 1 & 0 & 0 & 0\\0 & 1 & 15 & 25 & 10 & 1 & 0 & 0\\0 & 1 & 31 & 90 & 65 & 15 & 1 & 0\\0 & 1 & 63 & 301 & 350 & 140 & 21 & 1\end{matrix}\right]
\mathcal{S}_{m} = \left[\begin{matrix}1 &   &   &   &   &   &   &  \\  & 1 &   &   &   &   &   &  \\  & 1 & 1 &   &   &   &   &  \\  & 1 & 3 & 1 &   &   &   &  \\  & 1 & 7 & 6 & 1 &   &   &  \\  & 1 & 15 & 25 & 1  & 1 &   &  \\  & 1 & 31 & 9  & 65 & 15 & 1 &  \\  & 1 & 63 & 3 1 & 35  & 14  & 21 & 1\end{matrix}\right]
\end{displaymath}
therefore $\mathcal{S}_{m}^{-1}$ is
\begin{displaymath}
%\left[\begin{matrix}1 & 0 & 0 & 0 & 0 & 0 & 0 & 0\\0 & 1 & 0 & 0 & 0 & 0 & 0 & 0\\0 & -1 & 1 & 0 & 0 & 0 & 0 & 0\\0 & 2 & -3 & 1 & 0 & 0 & 0 & 0\\0 & -6 & 11 & -6 & 1 & 0 & 0 & 0\\0 & 24 & -50 & 35 & -10 & 1 & 0 & 0\\0 & -120 & 274 & -225 & 85 & -15 & 1 & 0\\0 & 720 & -1764 & 1624 & -735 & 175 & -21 & 1\end{matrix}\right]
\left[\begin{matrix}1 &   &   &   &   &   &   &  \\  & 1 &   &   &   &   &   &  \\  & -1 & 1 &   &   &   &   &  \\  & 2 & -3 & 1 &   &   &   &  \\  & -6 & 11 & -6 & 1 &   &   &  \\  & 24 & -5  & 35 & -1  & 1 &   &  \\  & -12  & 274 & -225 & 85 & -15 & 1 &  \\  & 72  & -1764 & 1624 & -735 & 175 & -21 & 1\end{matrix}\right]
\end{displaymath}
since $\mathcal{S}_{m}^{r}$ is too big we report $\mathcal{S}_{m}^{r}\boldsymbol{e}_{2}$ 
\begin{displaymath}
\hspace{-1cm}
%\left[\begin{matrix}1 & 0 & 0 & 0 & 0 & 0 & 0 & 0\\0 & 1 & 0 & 0 & 0 & 0 & 0 & 0\\0 & r & 1 & 0 & 0 & 0 & 0 & 0\\0 & \frac{r}{2} \left(3 r - 1\right) & 3 r & 1 & 0 & 0 & 0 & 0\\0 & \frac{r}{2} \left(6 r^{2} - 5 r + 1\right) & r \left(9 r - 2\right) & 6 r & 1 & 0 & 0 & 0\\0 & \frac{r}{6} \left(45 r^{3} - 65 r^{2} + 30 r - 4\right) & \frac{5 r}{2} \left(12 r^{2} - 7 r + 1\right) & 5 r \left(6 r - 1\right) & 10 r & 1 & 0 & 0\\0 & \frac{r}{24} \left(540 r^{4} - 1155 r^{3} + 890 r^{2} - 273 r + 22\right) & \frac{r}{2} \left(225 r^{3} - 235 r^{2} + 80 r - 8\right) & \frac{15 r}{2} \left(20 r^{2} - 9 r + 1\right) & 5 r \left(15 r - 2\right) & 15 r & 1 & 0\\0 & \frac{r}{24} \left(1890 r^{5} - 5481 r^{4} + 6125 r^{3} - 3129 r^{2} + 637 r - 18\right) & \frac{7 r}{24} \left(1620 r^{4} - 2565 r^{3} + 1490 r^{2} - 351 r + 22\right) & \frac{7 r}{2} \left(225 r^{3} - 185 r^{2} + 50 r - 4\right) & \frac{35 r}{2} \left(30 r^{2} - 11 r + 1\right) & \frac{35 r}{2} \left(9 r - 1\right) & 21 r & 1\end{matrix}\right]
\left[\begin{matrix} 0 \\ 1 \\ r \\ \frac{r}{2} \left(3 r - 1\right) \\ \frac{r}{2} \left(6 r^{2} - 5 r + 1\right) \\ \frac{r}{6} \left(45 r^{3} - 65 r^{2} + 30 r - 4\right) \\ \frac{r}{24} \left(540 r^{4} - 1155 r^{3} + 890 r^{2} - 273 r + 22\right) \\ \frac{r}{24} \left(1890 r^{5} - 5481 r^{4} + 6125 r^{3} - 3129 r^{2} + 637 r - 18\right) \end{matrix}\right]
\end{displaymath}
$\sqrt{\mathcal{S}_{m}}$ is
\begin{displaymath}
%\left[\begin{matrix}1 & 0 & 0 & 0 & 0 & 0 & 0 & 0\\0 & 1 & 0 & 0 & 0 & 0 & 0 & 0\\0 & \frac{1}{2} & 1 & 0 & 0 & 0 & 0 & 0\\0 & \frac{1}{8} & \frac{3}{2} & 1 & 0 & 0 & 0 & 0\\0 & 0 & \frac{5}{4} & 3 & 1 & 0 & 0 & 0\\0 & \frac{1}{32} & \frac{5}{8} & 5 & 5 & 1 & 0 & 0\\0 & - \frac{7}{128} & \frac{11}{32} & \frac{45}{8} & \frac{55}{4} & \frac{15}{2} & 1 & 0\\0 & \frac{1}{128} & - \frac{7}{128} & \frac{161}{32} & \frac{105}{4} & \frac{245}{8} & \frac{21}{2} & 1\end{matrix}\right]
\left[\begin{matrix}1 &   &   &   &   &   &   &  \\  & 1 &   &   &   &   &   &  \\  & \frac{1}{2} & 1 &   &   &   &   &  \\  & \frac{1}{8} & \frac{3}{2} & 1 &   &   &   &  \\  & 0   & \frac{5}{4} & 3 & 1 &   &   &  \\  & \frac{1}{32} & \frac{5}{8} & 5 & 5 & 1 &   &  \\  & - \frac{7}{128} & \frac{11}{32} & \frac{45}{8} & \frac{55}{4} & \frac{15}{2} & 1 &  \\  & \frac{1}{128} & - \frac{7}{128} & \frac{161}{32} & \frac{1 5}{4} & \frac{245}{8} & \frac{21}{2} & 1\end{matrix}\right]
\end{displaymath}
$e^{\mathcal{S}_{m}}$ is
\begin{displaymath}
%e \left[\begin{matrix}1 & 0 & 0 & 0 & 0 & 0 & 0 & 0\\0 & 1 & 0 & 0 & 0 & 0 & 0 & 0\\0 & 1 & 1 & 0 & 0 & 0 & 0 & 0\\0 & \frac{5}{2} & 3 & 1 & 0 & 0 & 0 & 0\\0 & \frac{21}{2} & 16 & 6 & 1 & 0 & 0 & 0\\0 & \frac{203}{3} & \frac{235}{2} & 55 & 10 & 1 & 0 & 0\\0 & \frac{14681}{24} & 1176 & \frac{1245}{2} & 140 & 15 & 1 & 0\\0 & \frac{22018}{3} & \frac{367745}{24} & 8911 & \frac{4515}{2} & \frac{595}{2} & 21 & 1\end{matrix}\right]
e \left[\begin{matrix}1 &   &   &   &   &   &   &  \\  & 1 &   &   &   &   &   &  \\  & 1 & 1 &   &   &   &   &  \\  & \frac{5}{2} & 3 & 1 &   &   &   &  \\  & \frac{21}{2} & 16 & 6 & 1 &   &   &  \\  & \frac{2 3}{3} & \frac{235}{2} & 55 & 1  & 1 &   &  \\  & \frac{14681}{24} & 1176 & \frac{1245}{2} & 14  & 15 & 1 &  \\  & \frac{22 18}{3} & \frac{367745}{24} & 8911 & \frac{4515}{2} & \frac{595}{2} & 21 & 1\end{matrix}\right]
\end{displaymath}
finally, $\log{\mathcal{S}_{m}}$ is
\begin{displaymath}
%\left[\begin{matrix}0 & 0 & 0 & 0 & 0 & 0 & 0 & 0\\0 & 0 & 0 & 0 & 0 & 0 & 0 & 0\\0 & 1 & 0 & 0 & 0 & 0 & 0 & 0\\0 & - \frac{1}{2} & 3 & 0 & 0 & 0 & 0 & 0\\0 & \frac{1}{2} & -2 & 6 & 0 & 0 & 0 & 0\\0 & - \frac{2}{3} & \frac{5}{2} & -5 & 10 & 0 & 0 & 0\\0 & \frac{11}{12} & -4 & \frac{15}{2} & -10 & 15 & 0 & 0\\0 & - \frac{3}{4} & \frac{77}{12} & -14 & \frac{35}{2} & - \frac{35}{2} & 21 & 0\end{matrix}\right]
\left[\begin{matrix}  &   &   &   &   &   &   &  \\  &   &   &   &   &   &   &  \\  & 1 &   &   &   &   &   &  \\  & - \frac{1}{2} & 3 &   &   &   &   &  \\  & \frac{1}{2} & -2 & 6 &   &   &   &  \\  & - \frac{2}{3} & \frac{5}{2} & -5 & 1  &   &   &  \\  & \frac{11}{12} & -4 & \frac{15}{2} & -1  & 15 &   &  \\  & - \frac{3}{4} & \frac{77}{12} & -14 & \frac{35}{2} & - \frac{35}{2} & 21 &  \end{matrix}\right]
\end{displaymath}
evaluating corresponding $g$ polynomials on $\mathcal{S}_{m}$, as required.


\subsection{Fibonacci numbers}


In order to have a comparison with a matrix $\mathcal{F}$ having two
eigenvalues $\lambda_{1}\neq \lambda_{2}$, defined as
\begin{displaymath}
\mathcal{F} = \left[\begin{matrix}1 & 1\\1 & 0\end{matrix}\right],
\quad  \lambda_{1} =  \frac{1}{2}- \frac{\sqrt{5}}{2} ,
\quad \lambda_{2} = \frac{1}{2} + \frac{\sqrt{5}}{2}
\end{displaymath}
respectively, we need to use an augmented base composed of polynomials
\begin{displaymath}
 \Phi_{ 1, 1 }{\left (z \right )} = \frac{z}{\lambda_{1} - \lambda_{2}} - \frac{\lambda_{2}}{\lambda_{1} - \lambda_{2}}, \quad  \Phi_{ 2, 1 }{\left (z \right )} = - \frac{z}{\lambda_{1} - \lambda_{2}} + \frac{\lambda_{1}}{\lambda_{1} - \lambda_{2}}
\end{displaymath}
to define polynomial $g$ representing $f(z)=z^{r}$:
\begin{displaymath}
g{\left (z \right )} = z \left(\frac{\lambda_{1}^{r}}{\lambda_{1} - \lambda_{2}} - \frac{\lambda_{2}^{r}}{\lambda_{1} - \lambda_{2}}\right) + \frac{\lambda_{1} \lambda_{2}^{r}}{\lambda_{1} - \lambda_{2}} - \frac{\lambda_{1}^{r} \lambda_{2}}{\lambda_{1} - \lambda_{2}}
\end{displaymath}
therefore $\mathcal{F}^{r}$ is $g(\mathcal{F})$, in matrix notation
\begin{displaymath}
\left[\begin{matrix}\frac{1}{\lambda_{1} - \lambda_{2}} \left(\lambda_{1} \lambda_{2}^{r} - \lambda_{1}^{r} \lambda_{2} + \lambda_{1}^{r} - \lambda_{2}^{r}\right) & \frac{\lambda_{1}^{r} - \lambda_{2}^{r}}{\lambda_{1} - \lambda_{2}}\\\frac{\lambda_{1}^{r} - \lambda_{2}^{r}}{\lambda_{1} - \lambda_{2}} & \frac{\lambda_{1} \lambda_{2}^{r} - \lambda_{1}^{r} \lambda_{2}}{\lambda_{1} - \lambda_{2}}\end{matrix}\right]
\end{displaymath}

For the sake of clarity, choose $r=8$ to obtain
    \begin{displaymath}
    \mathcal{F}^{8} = \left[\begin{matrix}f_{9} & f_{8}\\f_{8} & f_{7}\end{matrix}\right] = \left[\begin{matrix}34 & 21\\21 & 13\end{matrix}\right]
    \end{displaymath}
where $f_{n}$ is the $n$-th Fibonacci number within sequence A000045 in the OEIS.


\section{Appendix}


\begin{description}

\item[Facts about Riordan matrices] $\quad$

\begin{itemize} 
\item matrix-vector product $\mathcal{R}\cdot \boldsymbol{a}$ is defined as 
    $ d(t)A(h(t))$, where $A$ is a formal power series
    with coefficients belonging to vector $\boldsymbol{a}$ and $\mathcal{R}=(d, h)$;
\item matrix-matrix product $\mathcal{R}\cdot \mathcal{W}$ is defined as 
    $(d(t)g(h(t)), l(h(t)))$, where $\mathcal{R}=(d, h)$ and $\mathcal{W}=(g, l)$;
\item identity matrix $I$ is $(1, t)$;
\item the inverse $\mathcal{R}^{-1}$ of $\mathcal{R}=(d, h)$ is defined as 
    $\left(\frac{1}{d(\bar{h}(t))}, \bar{h}\right)$, where $\bar{h}$ is the \emph{compositional inverse} 
    of $h$ satisfying $\bar{h}(h(t))=t$;
\item the set of these matrices is a \emph{group} with matrix-matrix product as operation.    
\end{itemize}

\item[Facts about component matrices] $\quad$

\begin{itemize}
\item they are linearly independent and don't depend on function $f$;
\item they commute respect the product, $Z_{ij}Z_{kr}= Z_{kr}Z_{ij}$;
\item $Z_{ij}Z_{kr}=O$ if $i\neq k$;
\item $Z_{i1}Z_{ij}=Z_{ij}$ if $j > 0 $;
\item $Z_{i2}Z_{ij}=jZ_{i,j+1}$ if $j > 0 $;
\item $Z_{ij}=\frac{Z_{i2}^{j-1}}{(j-1)!}$ if $j > 1 $;
\item $Z_{i2}^{m_{i}}=O$ for $i\in\lbrace 1,\ldots,\nu\rbrace$;
\item $I = \sum_{i=1}^{\nu}{Z_{i1}}$ from $f(z)=1$;
\item $Z_{i2} = Z_{i1}(A-\lambda_{i}I)$ from $f(z)=z-\lambda_{i}$, for
        $i\in\lbrace 1,\ldots,\nu\rbrace$.
\end{itemize}

\end{description}



\subsection*{Matrices}


\rotatebox{90}{$
e^{\alpha\hat{H}_{m}} = \left[\begin{matrix}1 &   &   &   &   &   &   & \\\alpha h_{1} & 1 &   &   &   &   &   & \\\frac{\alpha^{2} }{2}h_{1} h_{2} & \alpha h_{2} & 1 &   &   &   &   & \\\frac{\alpha^{3} }{6}h_{1} h_{2} h_{3} & \frac{\alpha^{2} }{2}h_{2} h_{3} & \alpha h_{3} & 1 &   &   &   & \\\frac{\alpha^{4} }{24}h_{1} h_{2} h_{3} h_{4} & \frac{\alpha^{3} }{6}h_{2} h_{3} h_{4} & \frac{\alpha^{2} }{2}h_{3} h_{4} & \alpha h_{4} & 1 &   &   & \\\frac{\alpha^{5} }{120}h_{1} h_{2} h_{3} h_{4} h_{5} & \frac{\alpha^{4} }{24}h_{2} h_{3} h_{4} h_{5} & \frac{\alpha^{3} }{6}h_{3} h_{4} h_{5} & \frac{\alpha^{2} }{2}h_{4} h_{5} & \alpha h_{5} & 1 &   & \\\frac{\alpha^{6} }{720}h_{1} h_{2} h_{3} h_{4} h_{5} h_{6} & \frac{\alpha^{5} }{120}h_{2} h_{3} h_{4} h_{5} h_{6} & \frac{\alpha^{4} }{24}h_{3} h_{4} h_{5} h_{6} & \frac{\alpha^{3} }{6}h_{4} h_{5} h_{6} & \frac{\alpha^{2} }{2}h_{5} h_{6} & \alpha h_{6} & 1 & \\\frac{\alpha^{7} }{5040}h_{1} h_{2} h_{3} h_{4} h_{5} h_{6} h_{7} & \frac{\alpha^{6} }{720}h_{2} h_{3} h_{4} h_{5} h_{6} h_{7} & \frac{\alpha^{5} }{120}h_{3} h_{4} h_{5} h_{6} h_{7} & \frac{\alpha^{4} }{24}h_{4} h_{5} h_{6} h_{7} & \frac{\alpha^{3} }{6}h_{5} h_{6} h_{7} & \frac{\alpha^{2} }{2}h_{6} h_{7} & \alpha h_{7} & 1\end{matrix}\right]
$}





\bibliographystyle{plainnat}
\bibliography{biblio}
\end{document}
